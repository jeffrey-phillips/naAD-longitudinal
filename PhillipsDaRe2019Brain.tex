\documentclass[]{article}
\usepackage{lmodern}
\usepackage{amssymb,amsmath}
\usepackage{ifxetex,ifluatex}
\usepackage{fixltx2e} % provides \textsubscript
\ifnum 0\ifxetex 1\fi\ifluatex 1\fi=0 % if pdftex
  \usepackage[T1]{fontenc}
  \usepackage[utf8]{inputenc}
\else % if luatex or xelatex
  \ifxetex
    \usepackage{mathspec}
  \else
    \usepackage{fontspec}
  \fi
  \defaultfontfeatures{Ligatures=TeX,Scale=MatchLowercase}
\fi
% use upquote if available, for straight quotes in verbatim environments
\IfFileExists{upquote.sty}{\usepackage{upquote}}{}
% use microtype if available
\IfFileExists{microtype.sty}{%
\usepackage{microtype}
\UseMicrotypeSet[protrusion]{basicmath} % disable protrusion for tt fonts
}{}
\usepackage[margin=1in]{geometry}
\usepackage{hyperref}
\hypersetup{unicode=true,
            pdftitle={Longitudinal progression of grey matter atrophy in non-amnestic Alzheimer's disease},
            pdfauthor={*Jeffrey S. Phillips, PhD1,2, *Fulvio Da Re, MD1,3,4, David J. Irwin, MD1,2, Corey T. McMillan, PhD1,2, Sanjeev N. Vaishnavi, MD, PhD2,5, Sharon X. Xie, PhD6, Edward B. Lee, MD, PhD7, Philip A. Cook, PhD8, James C. Gee, PhD8, Leslie M. Shaw, PhD7, John Q. Trojanowski, MD, PhD7, David A. Wolk, MD2,5 and Murray Grossman, MD, EdD1,2},
            pdfborder={0 0 0},
            breaklinks=true}
\urlstyle{same}  % don't use monospace font for urls
\usepackage{graphicx,grffile}
\makeatletter
\def\maxwidth{\ifdim\Gin@nat@width>\linewidth\linewidth\else\Gin@nat@width\fi}
\def\maxheight{\ifdim\Gin@nat@height>\textheight\textheight\else\Gin@nat@height\fi}
\makeatother
% Scale images if necessary, so that they will not overflow the page
% margins by default, and it is still possible to overwrite the defaults
% using explicit options in \includegraphics[width, height, ...]{}
\setkeys{Gin}{width=\maxwidth,height=\maxheight,keepaspectratio}
\IfFileExists{parskip.sty}{%
\usepackage{parskip}
}{% else
\setlength{\parindent}{0pt}
\setlength{\parskip}{6pt plus 2pt minus 1pt}
}
\setlength{\emergencystretch}{3em}  % prevent overfull lines
\providecommand{\tightlist}{%
  \setlength{\itemsep}{0pt}\setlength{\parskip}{0pt}}
\setcounter{secnumdepth}{5}
% Redefines (sub)paragraphs to behave more like sections
\ifx\paragraph\undefined\else
\let\oldparagraph\paragraph
\renewcommand{\paragraph}[1]{\oldparagraph{#1}\mbox{}}
\fi
\ifx\subparagraph\undefined\else
\let\oldsubparagraph\subparagraph
\renewcommand{\subparagraph}[1]{\oldsubparagraph{#1}\mbox{}}
\fi

%%% Use protect on footnotes to avoid problems with footnotes in titles
\let\rmarkdownfootnote\footnote%
\def\footnote{\protect\rmarkdownfootnote}

%%% Change title format to be more compact
\usepackage{titling}

% Create subtitle command for use in maketitle
\newcommand{\subtitle}[1]{
  \posttitle{
    \begin{center}\large#1\end{center}
    }
}

\setlength{\droptitle}{-2em}

  \title{Longitudinal progression of grey matter atrophy in non-amnestic
Alzheimer's disease}
    \pretitle{\vspace{\droptitle}\centering\huge}
  \posttitle{\par}
    \author{*Jeffrey S. Phillips, PhD\textsuperscript{1,2}, *Fulvio Da Re,
MD\textsuperscript{1,3,4}, David J. Irwin, MD\textsuperscript{1,2},
Corey T. McMillan, PhD\textsuperscript{1,2}, Sanjeev N. Vaishnavi, MD,
PhD\textsuperscript{2,5}, Sharon X. Xie, PhD\textsuperscript{6}, Edward
B. Lee, MD, PhD\textsuperscript{7}, Philip A. Cook,
PhD\textsuperscript{8}, James C. Gee, PhD\textsuperscript{8}, Leslie M.
Shaw, PhD\textsuperscript{7}, John Q. Trojanowski, MD,
PhD\textsuperscript{7}, David A. Wolk, MD\textsuperscript{2,5} and
Murray Grossman, MD, EdD\textsuperscript{1,2}}
    \preauthor{\centering\large\emph}
  \postauthor{\par}
      \predate{\centering\large\emph}
  \postdate{\par}
    \date{February 13, 2019}

\usepackage{graphicx} \usepackage{morefloats} \usepackage{caption}
\usepackage{lscape} \usepackage{siunitx} \usepackage{textcomp}
\usepackage{xcolor, colortbl}
\newcommand{\blandscape}{\begin{landscape}}
\newcommand{\elandscape}{\end{landscape}}

\begin{document}
\maketitle

\pagenumbering{gobble}

* These authors contributed equally to this work.

\textbf{Short title:} Longitudinal Atrophy in Non-Amnestic AD

\textbf{Affiliations:}

\textsuperscript{1}Penn Frontotemporal Degeneration Center, University
of Pennsylvania, Philadelphia, PA, USA, \textsuperscript{2}Department of
Neurology, Perelman School of Medicine, University of Pennsylvania,
Philadelphia, PA, USA, \textsuperscript{3}PhD Program in Neuroscience,
University of Milano-Bicocca, Milan, Italy, \textsuperscript{4}School of
Medicine and Surgery, Milan Center for Neuroscience (NeuroMI),
University of Milano-Bicocca, Milan, Italy, \textsuperscript{5}Penn
Memory Center, University of Pennsylvania, Philadelphia, PA, USA,
\textsuperscript{6}Department of Biostatistics, Epidemiology, and
Informatics, Perelman School of Medicine, University of Pennsylvania,
Philadelphia, PA, USA, \textsuperscript{7}Center for Neurodegenerative
Disease Research, University of Pennsylvania, Philadelphia, PA, USA,
\textsuperscript{8}Penn Image Computing and Science Laboratory,
Department of Radiology, University of Pennsylvania, Philadelphia, PA,
USA

\textbf{Corresponding author:}

Jeffrey S. Phillips, PhD\\
\href{mailto:jefphi@pennmedicine.upenn.edu}{\nolinkurl{jefphi@pennmedicine.upenn.edu}}\\
Penn Frontotemporal Degeneration Center\\
Department of Neurology\\
University of Pennsylvania\\
3400 Spruce St., 3 Gibson\\
Philadelphia, PA 19104

\newpage

\pagenumbering{arabic}

\section*{Abstract}\label{abstract}
\addcontentsline{toc}{section}{Abstract}

Recent models of Alzheimer's disease progression propose that disease
may be transmitted between brain areas either via local diffusion or
long-distance transport via white-matter fiber pathways. However, it is
unclear whether such models are applicable in non-amnestic Alzheimer's
disease, which is associated with domain-specific cognitive deficits and
relatively spared episodic memory. To date, the anatomical progression
of disease in non-amnestic patients remains understudied. We used
longitudinal imaging to differentiate earlier atrophy and later disease
spread in three non-amnestic variants, including logopenic-variant
primary progressive aphasia (n=25), posterior cortical atrophy (n=20),
and frontal-variant Alzheimer's disease (n=12), as well as 17 amnestic
Alzheimer's disease patients. Patients were compared to 37 matched
controls. All patients had autopsy (n=7) or cerebrospinal fluid (n=67)
evidence of Alzheimer's disease pathology. We first assessed atrophy in
suspected sites of disease origin, adjusting for age, sex, and severity
of cognitive impairment; we then performed exploratory whole-brain
analysis to investigate longitudinal disease spread both within and
outside these regions. Additionally, we asked whether each phenotype
exhibited more rapid change in its associated disease foci than other
phenotypes. Finally, we investigated whether atrophy was related to
structural brain connectivity. Each non-amnestic phenotype displayed
unique patterns of initial atrophy and subsequent neocortical change
that correlated with cognitive decline. Longitudinal atrophy included
areas both proximal to and distant from sites of initial atrophy,
suggesting heterogeneous mechanisms of disease spread. Moreover,
regional rates of neocortical change differed by phenotype.
Logopenic-variant patients exhibited greater initial atrophy and more
rapid longitudinal change in left lateral temporal areas than other
groups. Frontal-variant patients had pronounced atrophy in left insula
and middle frontal gyrus, combined with more rapid atrophy of left
insula than other non-amnestic patients. In the medial temporal lobes,
non-amnestic patients had less atrophy at their initial scan than
amnestic patients, but longitudinal rate of change did not differ
between patient groups. Medial temporal sparing in non-amnestic
Alzheimer's disease may thus be due in part to later onset of medial
temporal degeneration than in amnestic patients rather than different
rates of atrophy over time. Finally, the magnitude of longitudinal
atrophy was predicted by structural connectivity, measured in terms of
node degree; this result provides indirect support for the role of
long-distance fiber pathways in the spread of neurodegenerative disease.

\section*{Keywords}\label{keywords}
\addcontentsline{toc}{section}{Keywords}

Non-amnestic Alzheimer's disease, logopenic-variant primary progressive
aphasia, posterior cortical atrophy, frontal-variant Alzheimer's
disease, longitudinal progression, neuroimaging

\section*{Abbreviations}\label{abbreviations}
\addcontentsline{toc}{section}{Abbreviations}

A\(\beta\) = \(\beta\)-amyloid; aAD= amnestic Alzheimer's disease; naAD
= non-amnestic Alzheimer's disease; CBS = corticobasal syndrome; fvAD =
frontal-variant Alzheimer's disease; lvPPA = logopenic-variant primary
progressive aphasia; PCA = posterior cortical atrophy; GM = grey matter;
MTL = medial temporal lobes; ROI = region of interest; MMSE = Mini
Mental Status Exam; CSF = cerebrospinal fluid; LME = linear mixed
effects; PVLT = Philadelphia Verbal Learning Test; PBAC = Philadelphia
Brief Assessment of Cognition

\newpage

\section*{Introduction}\label{introduction}
\addcontentsline{toc}{section}{Introduction}

Recent theories of neurodegenerative disease progression have raised the
possibility that pathogenic protein aggregates do not arise
spontaneously throughout the brain; rather, they may be transmitted from
areas of existing pathology through one or more mechanisms, including
local diffusion of pathogenic proteins through the extracellular medium
as well as long-distance transmission along white-matter pathways
between brain areas (Guo and Lee, 2014). This transmission model of
neurodegenerative disease is supported by a diverse body of research
ranging from rodent models (Liu \emph{et al.}, 2012; Iba \emph{et al.},
2013; Ahmed \emph{et al.}, 2014; Mezias \emph{et al.}, 2017) to
computational modeling of human neuroimaging data (Raj \emph{et al.},
2012, 2015; Iturria-Medina \emph{et al.}, 2014; Hu \emph{et al.}, 2016).
This research particularly supports the relevance of the transmission
model in typical, amnestic Alzheimer's disease (AD), which spreads from
the transentorhinal cortex and hippocampus to the rest of the medial
temporal lobes and ultimately to the neocortex (Braak and Braak, 1991).
This stereotyped progression provides detailed expectations against
which to test models of interregional transmission.

However, it is unclear whether the transmission hypothesis has equal
relevance in atypical presentations of AD, which demonstrate a number of
pathological and clinical differences from amnestic AD (aAD), including
relatively prominent neocortical disease and relative hippocampal
sparing (Galton \emph{et al.}, 2000; Murray \emph{et al.}, 2011;
Whitwell \emph{et al.}, 2012; Mesulam \emph{et al.}, 2014a). Clinically,
atypical AD includes at least four non-amnestic phenotypes:
logopenic-variant primary progressive aphasia (lvPPA), characterized by
primary language deficits (Gorno-Tempini \emph{et al.}, 2011); posterior
cortical atrophy (PCA), characterized by visuospatial deficits (Crutch
\emph{et al.}, 2017); a variant defined by deficits in executive
function and/or social behavior (Dubois \emph{et al.}, 2014), frequently
referred to as frontal-variant Alzheimer's disease (fvAD); and
corticobasal syndrome (CBS), which can present with a constellation of
lateralized motor and cognitive deficits (Medaglia \emph{et al.}, 2017).
These syndromes are marked by different rates of clinical progression
than in aAD (Duara \emph{et al.}, 2013; Byun \emph{et al.}, 2015;
Poulakis \emph{et al.}, 2018). Moreover, each phenotype is associated
with distinct anatomical distributions of disease, particularly in early
stages. LvPPA patients have strongly left-lateralized disease with
pathological accumulations in left superior temporal and inferior
parietal cortex (Mesulam \emph{et al.}, 2014b; Giannini \emph{et al.},
2017); additional disease is commonly observed in left dorsolateral
prefrontal cortex (Rogalski \emph{et al.}, 2016; Giannini \emph{et al.},
2017). PCA is marked by involvement of the parietal and/or occipital
lobes (Tang-Wai \emph{et al.}, 2004; Crutch \emph{et al.}, 2017);
depending on the patient, disease in PCA may or may not have a
right-hemisphere bias (Crutch \emph{et al.}, 2012; Ossenkoppele \emph{et
al.}, 2015a). In fvAD, elevated pathology has been most commonly
reported in the frontal lobes (Johnson \emph{et al.}, 1999;
Blennerhassett \emph{et al.}, 2014), although a recent report based on a
larger sample of these patients also highlights disease in posterior
cortical areas (Ossenkoppele \emph{et al.}, 2015b). In CBS due to AD,
disease is distributed widely and often asymmetrically throughout the
temporal and parietal lobes, sometimes extending into the posterior
portion of the frontal lobes (Lee \emph{et al.}, 2011; McMillan \emph{et
al.}, 2016). Each of these naAD phenotypes shares clinical similarities
to presentations of frontotemporal lobar degeneration (FTLD) spectrum
pathology, making it imperative to corroborate clinical diagnosis
through AD biomarkers.

Non-amnestic syndromes with underlying Alzheimer's disease pathology may
be more prevalent than previously thought (Peter \emph{et al.}, 2014;
Dickerson \emph{et al.}, 2017). However, relatively little research has
examined the anatomical spread of non-amnestic Alzheimer's disease
(naAD). In a previous MRI analysis (Phillips \emph{et al.}, 2018), we
inferred patterns of atrophy spread in naAD phenotypes from
cross-sectional MRI data: following the logic of postmortem pathology
studies, we explicitly assumed that the relative frequency of atrophy in
different brain areas could be used to infer the anatomical progression
of disease over time. These results corroborated the hypothesis that
each naAD phenotype has a distinct neocortical origin with relative
sparing of the medial temporal lobes (MTL). Moreover, this study
suggested that each naAD phenotype has a distinct pattern of disease
spread that differs from aAD.

In the present study, we sought to validate and extend this previous
work, using longitudinal MRI to differentiate patterns of earlier
atrophy from subsequent disease spread in each phenotype. Additionally,
we investigated whether aAD and naAD differ in the anatomic distribution
and longitudinal rate of grey matter (GM) atrophy over time. We reasoned
that such differences could explain phenotype-specific patterns of
clinical progression across amnestic and non-amnestic Alzheimer's
disease variants. In a hypothesis-driven analysis based on our previous
cross-sectional study (Phillips \emph{et al.}, 2018), we investigated
differences in GM volume at the time of initial MRI as well as volume
change over time in regions-of-interest (ROIs) associated with lvPPA,
PCA, fvAD, and aAD. Exploratory, whole-brain, voxelwise analysis of
cortical thickness was performed to map patterns of disease spread
beyond these initial ROIs. We sought to identify group differences in
atrophy distribution and progression independent of age, which has been
previously reported to differ between typical and atypical forms of
Alzheimer's disease (Murray \emph{et al.}, 2011). Based on the high
neocortical disease burden and domain-specific cognitive deficits that
we previously observed in naAD, we predicted that naAD patients would
exhibit faster rates of atrophy in phenotype-specific neocortical ROIs
relative to aAD. Additionally, we tested the hypothesis that naAD
patients would exhibit slower atrophy than aAD patients in the
hippocampus and surrounding MTL areas, as a possible explanation for the
relative memory sparing associated with these structures in naAD.
Finally, we compared longitudinal atrophy patterns to measures of
inter-regional structural connectivity estimated from a large population
of healthy controls; we predicted that connectivity would predict
longitudinal atrophy, consistent with the transmission hypothesis.

\section*{Materials and methods}\label{materials-and-methods}
\addcontentsline{toc}{section}{Materials and methods}

\subsection*{Patients}\label{patients}
\addcontentsline{toc}{subsection}{Patients}

\begin{table}[ht]
\centering
\caption{Participant characteristics at time of first scan. Median values and interquartile ranges (square brackets) are given for all continuous variables. Education, age at MRI, inter-scan interval, and disease duration are expressed in years. For each cognitive score, numbers in parentheses indicate the number of observations per group. P-values reflect the results of a chi-squared test for sex and Kruskal-Wallis tests for all other variables.} 
\scalebox{0.67}{
\begin{tabular}{lrrrrrr}
  \hline
 & Control & aAD & lvPPA & PCA & fvAD & P \\ 
  \hline
N &   37 &   17 &   25 &   20 &   12 &  \\ 
  Male &   16 (43.2\%)  &    6 (35.3\%)  &    9 (36.0\%)  &    7 (35.0\%)  &    7 (58.3\%)  &  0.672 \\ 
  Education & 16.0 [16.0, 18.0] & 16.0 [14.0, 18.0] & 16.0 [14.0, 19.0] & 16.0 [12.0, 16.0] & 16.0 [13.5, 18.0] &  0.421 \\ 
  Age at MRI (years) & 61.9 [57.9, 65.6] & 59.4 [53.5, 70.3] & 58.5 [56.9, 64.5] & 58.0 [55.1, 61.4] & 63.9 [59.7, 69.5] &  0.137 \\ 
  Inter-scan interval (years) & 1.2 [0.9, 1.7] & 1.2 [0.9, 1.5] & 1.1 [0.9, 1.3] & 1.0 [0.9, 1.2] & 1.0 [0.7, 1.1] & 0.162 \\ 
  Disease duration (years) & --- &  3.0 [1.9, 4.0] &  2.7 [1.7, 3.9] &  2.2 [1.3, 4.0] &  2.2 [1.8, 5.2] &  0.747 \\ 
  MMSE (0--30) & 29.0 [28.0, 30.0] (20) & 23.0 [20.0, 25.0] (17) & 25.0 [23.0, 28.0] (25) & 24.5 [18.8, 25.2] (20) & 23.0 [17.0, 26.0] (12) & <0.001 \\ 
  Recognition memory (discrimination, 0--1) &  1.0 [0.9, 1.0] (7) &  0.6 [0.5, 0.7] (10) &  0.8 [0.8, 1.0] (25) &  0.7 [0.6, 0.9] (19) &  0.6 [0.6, 0.8] (12) & <0.001 \\ 
  Speech (0--4) &  4.0 [4.0, 4.0] (3) &  2.5 [2.5, 3.0] (9) &  2.5 [2.0, 3.0] (19) &  3.0 [3.0, 4.0] (15) &  3.5 [2.2, 4.0] (11) &  0.004 \\ 
  Letter fluency (\# words/60 s) & 19.0 [17.5, 20.5] (7) &  9.0 [5.0, 13.0] (13) &  8.5 [5.2, 10.8] (22) & 10.0 [6.5, 15.5] (19) &  6.5 [3.0, 11.0] (12) &  0.001 \\ 
  Forward digit span (length correct) &  7.0 [7.0, 8.0] (11) &  5.0 [3.0, 6.0] (9) &  5.0 [4.0, 5.0] (25) &  6.0 [5.0, 7.0] (20) &  5.0 [4.0, 6.0] (12) &  0.005 \\ 
  Rey figure copy (0--12) & 12.0 [12.0, 12.0] (3) & 11.0 [4.0, 12.0] (9) & 12.0 [11.0, 12.0] (19) &  2.5 [0.0, 8.8] (12) &  9.5 [4.5, 11.0] (10) &  0.001 \\ 
  Judgment of line orientation (0--6) &  6.0 [6.0, 6.0] (3) &  3.0 [0.8, 5.0] (8) &  5.0 [4.0, 6.0] (19) &  2.0 [0.0, 4.0] (13) &  4.0 [3.0, 5.0] (9) &  0.004 \\ 
  Social behavior (0--18) & 17.0 [17.0, 17.0] (3) & 17.5 [16.8, 18.0] (8) & 18.0 [17.0, 18.0] (19) & 17.0 [16.0, 18.0] (15) & 13.0 [11.1, 16.5] (11) &  0.004 \\ 
  Oral trail-making test (0--6) &  6.0 [5.5, 6.0] (3) &  0.0 [0.0, 3.0] (5) &  2.0 [0.2, 3.0] (10) &  0.5 [0.0, 2.8] (10) &  2.0 [0.2, 3.8] (6) &  0.051 \\ 
  Reverse digit span (length correct) &  6.0 [4.5, 6.0] (11) &  3.0 [3.0, 3.0] (9) &  3.0 [3.0, 4.0] (25) &  3.0 [2.0, 3.0] (19) &  3.0 [2.0, 3.2] (12) & <0.001 \\ 
   \hline
\end{tabular}
}
\end{table}

The current study used a longitudinal case-control design based on data
retrospectively selected from the Integrated Neurodegenerative Disease
Database at the University of Pennsylvania. Participants were recruited
through the Penn Frontotemporal Degeneration Center (FTDC) and the Penn
Memory Center (PMC). All procedures were approved by the University of
Pennsylvania's Institutional Review Board, and all patients and/or their
caregivers gave written informed consent according to the principles
established by the Declaration of Helsinki. An initial database query
yielded 1897 patients scanned on the same 3-Tesla Siemens MRI scanner at
the Hospital of the University of Pennsylvania. Of these, 360 patients
had either autopsy or cerebrospinal fluid biomarker evidence of
underlying Alzheimer's disease pathology. An additional 58 patients were
excluded due to major cerebrovascular disease, stroke, head trauma, or
comorbid psychiatric, neurodegenerative, medical, or developmental
disorders apart from their primary diagnoses. Of the remainder, a total
of 90 patients had longitudinal data available and exhibited clinical
phenotypes of interest, as described below. At time of recruitment, MRI
scans for all patients were screened for signs of cerebrovascular
disease, hydrocephalus, or white matter lesions; those with a Fazekas
scale score\textgreater{}1 were excluded. Additionally, MRI scans were
visually inspected by two raters (JSP and FDR), and 16 patients were
excluded for poor quality data. The final sample included 181
T1-weighted MRI scans from 74 patients (25 with lvPPA, 20 PCA, 12 with
fvAD, and 17 with aAD) and 85 scans from 37 demographically-matched
controls. A majority of participants (48/74 patients and 29/37 controls)
had only 2 available scans; the remaining participants contributed 3--4
scans each. We included scans acquired with a minimum inter-scan
interval of 6 months up to 3.5 years from the initial MRI; beyond this
window, there were insufficient observations for a valid analysis. Seven
patients had primary neuropathologic diagnoses and 67 had CSF biomarkers
(total tau/beta-amyloid ratio greater than 0.34) indicative of
Alzheimer's disease pathology according to methods previously described
(Shaw \emph{et al.}, 2009; Irwin \emph{et al.}, 2012; Toledo \emph{et
al.}, 2012). APOE genotyping was performed on 66 of 74 patients. One
patient (white male, aAD, age 51 at onset) with an APOE
\(\epsilon3\)/\(\epsilon4\) genotype was found to have a mutation in the
PSEN1 gene; supplementary analyses indicated that excluding this patient
did not have substantive effects on the outcome of key analyses. All
patients were clinically diagnosed by experienced neurologists (MG, DJI,
DW, and SV), and diagnoses were confirmed by consensus after patients'
initial visit by clinicians with expertise in dementia. Clinical
criteria for each patient phenotype were as follows: for lvPPA, primary
language impairment including deficits in repetition and/or naming
(Gorno-Tempini \emph{et al.}, 2011; Giannini \emph{et al.}, 2017); for
PCA, visuospatial deficits (e.g., in object/spatial perception, neglect,
or oculomotor apraxia) (Crutch \emph{et al.}, 2017); for fvAD, clinical
evidence of a behavioral/dysexecutive syndrome per Rascovsky et al.`s
(2011) criteria for behavioral-variant frontotemporal dementia; and for
aAD, primary memory impairment plus deficits in one or more additional
cognitive domains (McKhann \emph{et al.}, 2011). NaAD patients had
relatively preserved episodic memory, as assessed through clinical
interviews and detailed mental status examinations; however, we note
that the label ``non-amnestic'' is used throughout this manuscript to
denote patients' initial presentation and does not preclude the
development of memory deficits in more advanced disease. NaAD patients
also had relatively spared abilities in other cognitive domains except
their domain of primary impairment at initial presentation. Due to the
challenges of clinically differentiating behavioral/dysexecutive
syndromes due to AD vs.~FTLD, we performed additional screening on the
fvAD group, as detailed in the Supplementary Material (``Patient
selection details''). The current study included 54 patients from our
previous, cross-sectional study (Phillips \emph{et al.}, 2018) (aAD,
n=8; lvPPA, n=24; PCA, n=16; and fvAD, n=6).

Shapiro-Wilks tests indicated non-normal distributions for education and
disease duration, age, and MMSE score at initial MRI (all
p\textless{}0.001). Kruskal-Wallis tests of group differences were
non-significant, with the exception of MMSE {[}\(\chi^2\)(4)=38.5,
p\textless{}0.001{]}, reflecting patients' cognitive deficits relative
to controls. Mann-Whitney tests confirmed that all patient groups
exhibited significantly lower MMSE scores than controls (all
U\textgreater{}=428, p\textless{}0.001); all other pairwise comparisons
were non-significant. To corroborate naAD patients' domain-specific
cognitive impairment, we analyzed neuropsychological performance on
assessments independent of those used in clinical diagnosis, including
performance on specific items of the Philadelphia Brief Assessment of
Cognition (PBAC) (Libon \emph{et al.}, 2011b). Only neuropsychological
observations acquired within 1 year of an MRI scan were included.
Language was assessed in terms of speech features (with lower scores
indicating speech and language impairment), forward digit span as a
measure of repetition (Giannini \emph{et al.}, 2017), and letter
fluency, which is sensitive to deficits in executive-mediated lexical
retrieval (Rascovsky \emph{et al.}, 2007; Ramanan \emph{et al.}, 2017).
Visuospatial function was assessed by patients' ability to copy a
modified version of the Rey complex figure as well as the judgment of
line orientation. Social behavior was assessed on an 18-point scale
evaluating social comportment, apathy, disinhibition, agitation,
empathy, and ritualistic behaviors. Executive function was evaluated
through an oral version of the trail-making test as well as backward
digit span. Finally, episodic memory was assessed by recognition on the
Philadelphia Verbal Learning Test (PVLT) (Libon \emph{et al.}, 2011a) or
the PBAC verbal memory test, as available. All neuropsychological
assessments were acquired within 1 year of the initial MRI scan (PVLT:
mean=0.19 years, SD=0.25; letter fluency: mean=0.14 years, SD=0.25;
PBAC: mean=0.21 years, SD=0.27; digit span: mean=0.11 years, SD=0.21).
Results were consistent with each phenotype's primary impairment in all
domains except for executive function (Table 1). Post-hoc comparisons
between patient groups for neuropsychological performance at initial MRI
are reported in Supplementary Table 11. The median and maximum follow-up
intervals for the MMSE were 1.4 and 3.8 years, respectively; for verbal
recognition memory, 1.7 and 4.2 years; for letter fluency, 1.6 and 4.6
years; for forward and reverse digit span, 1.5 and 4.0 years; and for
additional measures, which were derived from the PBAC, 1.6 and 4.6
years.

\subsection*{Neuroimaging methods}\label{neuroimaging-methods}
\addcontentsline{toc}{subsection}{Neuroimaging methods}

T1-weighted MR images were acquired axially with 0.98 mm x 0.98 mm x 1
mm voxels, a 256 x 192 matrix, a repetition time of 1620 ms, an
inversion time of 950 ms, and a flip angle of 15\(^\circ\). Scans were
visually inspected for quality by two authors (JP and FDR). Advanced
Normalization Tools (ANTs) (Avants \emph{et al.}, 2014; Tustison
\emph{et al.}, 2014) was used to process each image using a prior-based
approach. Images underwent intensity normalization (Tustison \emph{et
al.}, 2010) and were spatially normalized to a template based on healthy
controls from the Open Access Series of Imaging Studies (OASIS) dataset
(Marcus \emph{et al.}, 2007) using a symmetric diffeomorphic algorithm
(Klein \emph{et al.}, 2009; Avants \emph{et al.}, 2011). Images were
segmented into 6 tissue classes (cortical grey matter, subcortical grey
matter, deep white matter, CSF, brainstem, and cerebellum) using
template-based priors; this tissue segmentation was then used to
estimate cortical thickness; ANTs cortical thickness measurements have
been extensively validated relative to surface-based methods such as
FreeSurfer (Tustison \emph{et al.}, 2014; Klein \emph{et al.}, 2017). We
used a joint label fusion approach (Wang \emph{et al.}, 2013) to align
the Mindboggle-101 labels (based on the Desikan-Killainy-Tourville label
scheme) (Klein and Tourville, 2012) with each image using
pseudo-geodesic registration (Tustison and Avants, 2013) and calculated
the volume of GM voxels within each label, normalized by intracranial
volume and converted to a z-score relative to controls' initial scans.
To perform voxelwise group analyses, we warped cortical thickness images
to the template using the previously-computed spatial transforms; these
images were then spatially smoothed with a 2-sigma Gaussian kernel and
downsampled to 2 mm isotropic voxels.

\subsection*{Statistical analysis}\label{statistical-analysis}
\addcontentsline{toc}{subsection}{Statistical analysis}

In a hypothesis-driven analysis, we analyzed GM volumes in
phenotype-specific ROIs motivated by our previous study of disease
progression in naAD (Phillips \emph{et al.}, 2018). This study
identified the regions most commonly atrophied in each naAD phenotype,
reflecting the likely anatomical origin of disease. These ROIs included
left middle and superior temporal gyri in lvPPA; right precuneus,
superior parietal lobule, and angular, supramarginal, and middle
temporal gyri in PCA; and left anterior insula and middle frontal gyrus
as well as right middle temporal gyrus in fvAD (Table 2). Each ROI was
expected to exhibit lower volume at the time of participants' initial
MRI scan as well as more rapid volume loss over time in its associated
patient group(s) relative to other groups. We additionally hypothesized
that the aAD group would demonstrate selective atrophy in the MTL,
including bilateral hippocampi, parahippocampal gyri, and entorhinal
cortex. Atrophy at the time of initial MRI was analyzed using multiple
linear regression models with a factor of group and covariates for age,
sex, and MMSE score at the time of initial MRI; controls formed the
reference group in these models. Longitudinal atrophy was assessed using
linear mixed effects (LME) models with fixed factors of group, time
since first scan, and the interaction of group x time. As in the
baseline model, covariates included age, sex, and MMSE score at initial
MRI. A subject-specific random intercept was included to account for
intra-individual correlations in imaging measures. Post-hoc comparisons
were performed for the effect of group at initial MRI as well as the
group x time interaction in longitudinal models; values of
p\textless{}0.05, corrected using the false discovery rate method, were
considered significant.

We used LME models to relate GM volume change to neuropsychological
performance within 1 year of each imaging session. Due to the limited
number of observations, only linear associations between atrophy and
time were assessed. The mean interval between test and MRI was 0.30
years (SD=0.30) for recognition memory; 0.22 years (SD=0.29) for letter
fluency; 0.18 years (SD=0.29) for digit span; and 0.25 years (SD=0.33)
for all other longitudinal neuropsychological measures. Separate LME
models were computed for each measure and change in associated ROIs.
Thus, recognition performance was related to GM volume in each of the 6
MTL ROIs; language measures were compared to volume change in left
middle and superior temporal gyrus; visuospatial measures were related
to change in the right superior parietal lobule, precuneus, and angular,
supramarginal, and middle temporal gyri; and behavioral and executive
measures were related to left anterior insula and middle frontal gyrus
as well as right middle temporal gyrus. Neuropsychological performance
formed the outcome in each model; predictors treated as fixed effects
included regional GM volume at initial MRI and subsequent volume change,
as well as covariates of sex and education. Additionally, a
subject-specific random intercept was included in the LME model. Due to
limited neuropsychological data, controls were omitted from these
models. The association with regional volume change in each model was
assessed at a significance level of p\textless{}0.05, corrected for
false discovery rate.

Additionally, we performed exploratory, whole-brain, voxelwise analysis
to investigate differences in cortical thickness that were not assessed
by \emph{a priori} ROIs. ROI-based and voxelwise analyses both present
distinct advantages and weaknesses. Voxelwise analysis is not
constrained by the borders of anatomically-defined ROIs, and it allows
more precise anatomical localization of effects. However, ROI volume is
regarded as a more reliable measure of GM atrophy than cortical
thickness (Schwarz \emph{et al.}, 2016). Moreover, voxelwise parametric
tests depend on patients' displaying neurodegeneration at the same
precise point within a brain area. Thus, ROI-based volumetric analysis
may be more sensitive to atrophy if the precise focus of atrophy within
a region differs across individuals. Voxelwise analysis did not include
hippocampus, where cortical thickness is not well estimated (Han
\emph{et al.}, 2006; Gronenschild \emph{et al.}, 2012; Schwarz \emph{et
al.}, 2016), but did include entorhinal cortex and parahippocampal gyri.
As in ROI-based analysis, we used multiple regression to assess group
differences at initial MRI and an LME model to investigate longitudinal
atrophy. These voxelwise models used the same regression formulae as
ROI-based models, and the LME was implemented in the 3dLME (Chen
\emph{et al.}, 2013) function from the Analysis of Functional
NeuroImaging (AFNI) software suite. Multiple comparisons correction was
performed by first thresholding voxelwise results at p\textless{}0.001
(uncorrected), then applying a cluster extent threshold corresponding to
a cluster-wise alpha value of 0.05. To calculate cluster extent
thresholds, we first estimated spatial auto-correlation from the model
residuals using AFNI's 3dFWHMx. We then used the 3dClustSim function,
which is based on a Monte Carlo approach (Forman \emph{et al.}, 1995;
Cox \emph{et al.}, 2017), to determine the cluster size corresponding to
a false-positive rate of 0.05 at a voxelwise threshold of
p\textless{}0.001 (uncorrected). These simulations indicated a cluster
threshold of 73 voxels (i.e., \SI{584}{\micro\litre}) for the baseline
MRI model and a threshold of 75 voxels (\SI{600}{\micro\litre}) for the
longitudinal LME model. For both the baseline effect of group and the
group x time interaction, we performed post-hoc contrasts between all
groups, which were corrected to cluster-wise p\textless{}0.05 using the
same method. In the Supplementary Material, we additionally display
voxelwise contrasts vs.~controls at a lenient threshold of
p\textless{}0.01, uncorrected for multiple comparisons (Supplementary
Figure 3).

\subsection*{Structural connectivity}\label{structural-connectivity}
\addcontentsline{toc}{subsection}{Structural connectivity}

To investigate associations between atrophy progression and brain
connectivity, we related longitudinal atrophy to structural
population-average structural connectivity measures computed by Yeh et
al. (2018). The decision to use population-average connectivity measures
rather than estimating connectivity from patients was based on both
practical and conceptual considerations. First, constraining participant
selection by the availability of white-matter imaging data would have
further reduced sample sizes. Second, white-matter degeneration in
patients' brains might adversely affect fiber tractography, leading to
false negatives in estimating region-to-region brain connectivity.

Yeh and colleagues reported a whole-brain connectivity matrix (available
at \url{http://brain.labsolver.org/}) based on diffusion MRI data from
842 healthy participants in the Human Connectome Project; connectivity
values represent average anisotropy values for white-matter fiber tracts
connecting 65 regions in a modified version of the Automated Anatomical
Labeling (AAL) brain parcellation (Tzourio-Mazoyer \emph{et al.}, 2002).
Because label boundaries for major cortical structures vary between the
AAL and Mindboggle parcellations, we warped the modified AAL atlas into
the native acquisition space for each of the T1-weighted scans in the
current study and re-computed GM volumes based on this parcellation. An
example of the anisotropy-based structural connectivity values reported
by Yeh et al. (2018) is shown in Supplementary Table 6 for the areas of
greatest overlap with Mindboggle ROIs in the hypothesis-driven analysis
described above.

Using the igraph package for R (\url{https://igraph.org/r/}), we created
an unweighted, undirected graph of structural connectivity from Yeh et
al.`s (2018) connectivity matrix, omitting the cerebellum and brainstem
to yield a total of 62 nodes (i.e., brain areas). The degree of each
node was computed as the number of non-zero white-matter connections
with other regions. Self-connections were excluded; thus, the maximum
possible degree of a node was 61. As with Mindboggle labels, volumes
were normalized by each participants' intracranial volume and converted
to a z-score relative to the region-wise mean and standard deviation of
the control sample. We calculated annualized change in GM volume over
time for each region by subtracting these z-score volume measures from
the first and last available scans for each participant and dividing by
the time interval. We then computed a linear mixed effects model with
annualized change as the outcome and fixed effects of group, node
degree, and the group x degree interaction, covarying for the baseline
volume of each region, patients' age at initial MRI, and sex. The
average volume of each region (i.e., raw volume divided by intracranial
volume) among control participants was also included as a covariate to
ensure that variation in node degree did not simply reflect differences
in region size. A random intercept was estimated for each participant,
and a significance threshold of p\textless{}0.05 was used.

\subsection*{Data and availability}\label{data-and-availability}
\addcontentsline{toc}{subsection}{Data and availability}

Computer code for the current manuscript (including all text, analysis,
and visualization of results) is available in the form of Rmarkdown and
LaTeX scripts in a public GitHub repository
(\url{https://github.com/jeffrey-phillips/naAD-longitudinal.git}).
Rmarkdown code requires \href{https://cran.r-project.org/}{R} version
3.4.4 or higher. Investigators who wish to access imaging and clinical
data may submit a direct request to the corresponding author.

\section*{Results}\label{results}
\addcontentsline{toc}{section}{Results}

\begin{table}[ht]
\centering
\caption{Differences in grey matter volume at initial MRI and longitudinal atrophy in hypothesis-driven analysis of regional brain volumes, relative to matched controls. Hypotheses included selective atrophy of neocortical areas associated with early disease in naAD\textsuperscript{10} and of the MTL (hippocampus, entorhinal cortex, and parahippocampal gyrus) in aAD patients. The left and right precentral gyri are included to demonstrate the regional specificity of atrophy. F-statistics indicate the main effect of group at initial MRI scan and the group x time interaction across all scans. Additional columns report z-statistics for pairwise contrasts of each patient group vs. controls. Blue cells indicate significant differences in volume only at initial MRI; red cells indicate significant differences in longitudinal atrophy rates; and green cells indicate differences in both initial volume and longitudinal atrophy, based on a threshold of p<0.05, corrected using the false discovery rate method. n.s.=non-significant; *p<0.05; **p<0.01; ***p<0.001.} 
\scalebox{0.67}{
\begin{tabular}{rlrrrrrrrrrr}
  \hline
A priori association & Region & $F_{First~MRI}$(4,103) & aAD & lvPPA & PCA & fvAD & $F_{Group~x~Time}$(4,150) & aAD & lvPPA & PCA & fvAD \\ 
  \hline
aAD & L entorhinal & 7.2*** & \cellcolor{blue}\textcolor{white}{-3.4} & -1.5 & -1.5 & -0.1 & 6.7*** & -2.2 & \cellcolor{red}\textcolor{white}{-3.7} & \cellcolor{red}\textcolor{white}{-3.3} & \cellcolor{red}\textcolor{white}{-3.9} \\ 
   & R entorhinal & 5.5*** & \cellcolor{green}\textcolor{black}{-2.6} & 0.7 & -1.9 & -0.1 & 7.4*** & \cellcolor{green}\textcolor{black}{-4.1} & \cellcolor{red}\textcolor{white}{-2.9} & \cellcolor{red}\textcolor{white}{-4.1} & \cellcolor{red}\textcolor{white}{-3.0} \\ 
   & L hippocampus & 8.4*** & \cellcolor{green}\textcolor{black}{-4.6} & \cellcolor{green}\textcolor{black}{-3.2} & -2.1 & -1.6 & 5.4*** & \cellcolor{green}\textcolor{black}{-4.1} & \cellcolor{green}\textcolor{black}{-2.9} & \cellcolor{red}\textcolor{white}{-2.9} & -0.3 \\ 
   & R hippocampus & 7.4*** & \cellcolor{green}\textcolor{black}{-4.3} & -1.7 & \cellcolor{green}\textcolor{black}{-3.1} & -1.8 & 4.7** & \cellcolor{green}\textcolor{black}{-3.5} & -2.3 & \cellcolor{green}\textcolor{black}{-3.4} & -1.2 \\ 
   & L parahippocampal & 2.8** & -1.7 & -2.1 & 0.1 & 0.8 & 5.6*** & \cellcolor{red}\textcolor{white}{-3.5} & \cellcolor{red}\textcolor{white}{-4.0} & \cellcolor{red}\textcolor{white}{-2.6} & -2.1 \\ 
   & R parahippocampal & 1.6* & -2.2 & -0.5 & -1.3 & -0.3 & 5.9*** & \cellcolor{red}\textcolor{white}{-3.7} & \cellcolor{red}\textcolor{white}{-2.5} & \cellcolor{red}\textcolor{white}{-4.0} & -0.5 \\ 
  lvPPA & L middle temporal & 30.5*** & \cellcolor{green}\textcolor{black}{-3.0} & \cellcolor{green}\textcolor{black}{-7.3} & \cellcolor{green}\textcolor{black}{-3.2} & \cellcolor{green}\textcolor{black}{-3.0} & 34.7*** & \cellcolor{green}\textcolor{black}{-9.6} & \cellcolor{green}\textcolor{black}{-9.2} & \cellcolor{green}\textcolor{black}{-6.4} & \cellcolor{green}\textcolor{black}{-5.3} \\ 
   & L superior temporal & 21.5*** & \cellcolor{green}\textcolor{black}{-2.6} & \cellcolor{green}\textcolor{black}{-6.8} & -2.3 & -1.8 & 17.3*** & \cellcolor{green}\textcolor{black}{-4.4} & \cellcolor{green}\textcolor{black}{-8.1} & \cellcolor{red}\textcolor{white}{-3.7} & \cellcolor{red}\textcolor{white}{-2.8} \\ 
  PCA & R angular & 14.7*** & \cellcolor{blue}\textcolor{white}{-2.8} & -1.3 & \cellcolor{green}\textcolor{black}{-3.9} & \cellcolor{blue}\textcolor{white}{-3.0} & 2.4 & -1.4 & \cellcolor{red}\textcolor{white}{-2.6} & \cellcolor{green}\textcolor{black}{-2.5} & -1.0 \\ 
   & R precuneus & 9.8*** & -0.7 & -0.5 & \cellcolor{green}\textcolor{black}{-4.4} & -1.3 & 13.7*** & \cellcolor{red}\textcolor{white}{-6.5} & \cellcolor{red}\textcolor{white}{-4.8} & \cellcolor{green}\textcolor{black}{-4.5} & -0.9 \\ 
   & R superior parietal lobule & 16.7*** & 0.6 & 0.0 & \cellcolor{blue}\textcolor{white}{-5.5} & -1.7 & 4.2** & \cellcolor{red}\textcolor{white}{-3.6} & \cellcolor{red}\textcolor{white}{-2.7} & -2.1 & -0.1 \\ 
   & R supramarginal & 6.4*** & 0.2 & 0.4 & \cellcolor{green}\textcolor{black}{-2.6} & -0.6 & 11.2*** & \cellcolor{red}\textcolor{white}{-6.5} & \cellcolor{red}\textcolor{white}{-3.1} & \cellcolor{green}\textcolor{black}{-2.6} & -0.5 \\ 
  fvAD & L anterior insula & 10.3*** & -2.1 & \cellcolor{blue}\textcolor{white}{-2.6} & 0.6 & \cellcolor{green}\textcolor{black}{-3.4} & 3.5* & -2.0 & -1.6 & -0.4 & \cellcolor{green}\textcolor{black}{-3.3} \\ 
   & L middle frontal & 15.3*** & \cellcolor{green}\textcolor{black}{-2.6} & \cellcolor{green}\textcolor{black}{-3.8} & -2.4 & \cellcolor{blue}\textcolor{white}{-4.4} & 8.2*** & \cellcolor{green}\textcolor{black}{-3.9} & \cellcolor{green}\textcolor{black}{-4.9} & \cellcolor{red}\textcolor{white}{-3.8} & -2.1 \\ 
  PCA, fvAD & R middle temporal & 23.6*** & \cellcolor{green}\textcolor{black}{-2.8} & \cellcolor{green}\textcolor{black}{-2.7} & \cellcolor{green}\textcolor{black}{-5.6} & \cellcolor{green}\textcolor{black}{-4.2} & 44.8*** & \cellcolor{green}\textcolor{black}{-11.3} & \cellcolor{green}\textcolor{black}{-9.3} & \cellcolor{green}\textcolor{black}{-7.5} & \cellcolor{green}\textcolor{black}{-6.4} \\ 
  Reference ROI & L precentral & 3.4*** & 0.8 & -1.3 & -1.4 & 1.2 & 3.7** & \cellcolor{red}\textcolor{white}{-3.1} & \cellcolor{red}\textcolor{white}{-3.1} & -2.1 & -1.3 \\ 
   & R precentral & 5.4*** & 0.2 & -0.7 & \cellcolor{green}\textcolor{black}{-2.9} & -0.7 & 5.0** & \cellcolor{red}\textcolor{white}{-3.3} & \cellcolor{red}\textcolor{white}{-3.3} & \cellcolor{green}\textcolor{black}{-3.3} & -0.6 \\ 
   \hline
\end{tabular}
}
\end{table}

\begin{figure}

{\centering \includegraphics{./fig/base_diff_fig-1} 

}

\caption{Patient group differences at time of initial MRI in normalized volumes for a priori regions of interest. Box-and-whisker plots represent the distribution of regional grey matter volumes, expressed in z-score units relative to the healthy control reference group and adjusted for age, sex, and global cognition. More negative values on the x-axis indicate greater atrophy. The vertical bar in each box indicates the median volume; the edges of the box represent the interquartile range (IQR), i.e., the difference between the first and third quartiles. The whiskers extend to the most extreme point within $1.5\times IQR$ from the left or right edge of the box; observations falling outside this range are plotted individually. The notches in each box extend $1.58\times IQR\div{\sqrt{n}}$, displaying an approximate 95\% confidence interval for the median. Black brackets indicate significant pairwise group differences (p<0.05 after false discovery rate correction). Group is indicated by color as well as by the shape centered over the median line in each box.}\label{fig:base_diff_fig}
\end{figure}

\subsection*{Hypothesis-driven analysis of ROI
volumes}\label{hypothesis-driven-analysis-of-roi-volumes}
\addcontentsline{toc}{subsection}{Hypothesis-driven analysis of ROI
volumes}

\subsubsection*{Areas of earlier atrophy in each
phenotype}\label{areas-of-earlier-atrophy-in-each-phenotype}
\addcontentsline{toc}{subsubsection}{Areas of earlier atrophy in each
phenotype}

We first assessed both regional volume at initial MRI and longitudinal
volume change in each group relative to matched controls in ROIs
associated \emph{a priori} with each phenotype. The purpose of this
hypothesis-driven analysis was two-fold: first, to dissociate effects of
earlier vs.~later degeneration that are confounded in cross-sectional
studies; and second, to test hypotheses regarding differential rates of
atrophy between phenotypes. This analysis identified multiple regions
displaying initial atrophy in each phenotype (Table 2), which reflect
atrophy prior to patients' initial scans. While a subset of these
regions continued to degenerate over the follow-up period (Table 2,
green cells), others exhibited no further change (blue cells).
Additionally, we detected a number of regions that were not atrophied
relative to controls at the initial scan but demonstrated progression
over the follow-up period (red cells); these areas are interpreted as
areas of later disease spread in each phenotype. In cross-sectional
analysis of participants' initial MRI scans, all ROIs exhibited a main
effect of group, independent of age and MMSE.

Patterns of atrophy at initial MRI corroborated prior cross-sectional
studies of naAD, supporting the accuracy of clinical diagnoses. LvPPA
patients exhibited strong lateralization of disease, with early atrophy
relative to controls and other patient groups in left superior and
middle temporal gyri (Table 2, Figure 1). Additionally, they had
significant atrophy relative to controls in left anterior insula,
hippocampus, and middle frontal gyrus as well right middle temporal
gyrus. PCA patients, in turn, had significant atrophy at first MRI in
right angular gyrus, precuneus, superior parietal lobule, and
hippocampus as well as bilateral middle temporal gyri. FvAD patients had
significant atrophy in left anterior insula and middle frontal gyrus;
right angular gyrus; and bilateral middle temporal gyri. The precentral
gyrus, which comprises primary motor areas, exhibited early atrophy only
in the PCA group and was restricted to the right hemisphere, consistent
with the general right-lateralization of these patients' atrophy
pattern; the relative sparing of these structures is consistent with
patients' preserved motor function and demonstrates the regional
specificity of atrophy patterns. The aAD patients exhibited initial
atrophy relative to controls in bilateral hippocampi and entorhinal
cortex, left middle frontal gyrus, bilateral temporal cortex, and right
angular gyrus. These temporoparietal areas have been previously
characterized as nodes of the posterior default mode network in which
different AD phenotypes demonstrate convergence of atrophy patterns
(Ossenkoppele \emph{et al.}, 2015a). Additionally, aAD patients
demonstrated more severe atrophy than naAD groups in bilateral
hippocampi and entorhinal cortex (Figure 1). Non-amnestic patients
exhibited characteristic sparing of MTL structures, with initial atrophy
limited to left hippocampus in lvPPA and right hippocampus in PCA. In
longitudinal models, areas of significant early atrophy tended to
demonstrate further progression over the follow-up period relative to
controls (Table 2, green cells; see also Supplementary Figure 2).
However, a subset of brain areas had a non-significant slope of change
over time (Table 2, blue cells), suggesting a slowing of atrophy. These
areas included left entorhinal and right angular gyrus in aAD; left
anterior insula in lvPPA; right superior parietal lobule in PCA; and
left middle frontal gyrus in fvAD. Importantly, variation in sample
sizes should be taken into consideration when interpreting results for
different patient groups. For example, the fvAD group (the smallest
sample) may be more susceptible to false negatives than larger groups.
We caution against drawing conclusions about group differences by visual
comparisons of each group's results vs.~controls (Table 2, Figure 3).
Rather, group differences in atrophy are directly assessed in Figures
1--2 and 4--5.

\subsubsection*{Longitudinal analysis identifies areas of later change
in each
phenotype}\label{longitudinal-analysis-identifies-areas-of-later-change-in-each-phenotype}
\addcontentsline{toc}{subsubsection}{Longitudinal analysis identifies
areas of later change in each phenotype}

Additionally, multiple brain areas in each phenotype demonstrated
significant change over time despite an absence of atrophy at initial
MRI; these areas appear to represent disease spread in later stages. In
the neocortex, lvPPA patients exhibited longitudinal atrophy in right
temporoparietal areas, while PCA patients exhibited new left-hemisphere
atrophy in superior temporal and middle frontal gyrus. FvAD patients
exhibited new atrophy in left superior temporal gyrus, marking lateral
temporal cortex as one of the most consistent areas of longitudinal
change across patient groups. In the MTL, aAD, lvPPA, and PCA patients
all exhibited later atrophy in bilateral parahippocampal gyri; and all
three naAD groups demonstrated later atrophy in bilateral entorhinal
cortex. Additionally, PCA patients exhibited later-stage atrophy in
right hippocampus. Because a subset of PCA patients have a disease focus
in the ventral visual processing stream (Crutch \emph{et al.}, 2017), we
additionally evaluated longitudinal atrophy in bilateral inferior
occipital gyri (Supplementary Tables 7--10); while we observed
significant atrophy across patient groups, there were no between-group
differences in either mean atrophy or its rate of change. Finally, in
precentral gyrus reference regions, all patient groups except fvAD
exhibited longitudinal change relative to controls, consistent with
their more advanced disease status; however, in the PCA group this
change remained restricted to the right hemisphere.

\subsubsection*{Group differences in regional rates of
change}\label{group-differences-in-regional-rates-of-change}
\addcontentsline{toc}{subsubsection}{Group differences in regional rates
of change}

The longitudinal design allowed us to test the hypothesis that each naAD
phenotype would exhibit faster atrophy in its associated neocortical
ROIs than other patient groups, consistent with phenotype-specific
disease patterns. Additionally, we predicted that naAD patients would
exhibit more gradual rates of change in MTL structures than aAD
patients, providing a dynamic correlate of MTL sparing in naAD. These
hypotheses were tested through pairwise contrasts of group x time
interaction terms from linear mixed effects models of GM volume change.
Consistent with hypotheses, lvPPA patients had more rapid atrophy than
PCA patients in left superior temporal gyrus (z=2.8, p\textless{}0.02)
as well as marginally more rapid change than aAD patients (z=2.1,
p\textless{}0.09). Similarly, the fvAD group exhibited significantly
greater atrophy rates in left anterior insula than lvPPA (z=2.5,
p\textless{}0.04) and PCA patients (z=2.9, p\textless{}0.02). Contrary
to hypotheses, PCA patients did not exhibit faster neurodegeneration
during the follow-up period than other phenotypes. Because PCA is
associated with heterogeneous disease distributions including both
dorsal and ventral occipito-temporal variants (Crutch \emph{et al.},
2017), we performed supplementary analyses of longitudinal atrophy in
ventrolateral occipital cortex (i.e., bilateral inferior occipital
gyri). While PCA patients exhibited significantly lower GM volumes than
controls in both left and right inferior occipital gyri, there were no
significant differences in either mean volumes or rates of longitudinal
change with other patient groups (Supplementary Tables 7--10).

In addition, aAD patients had more rapid atrophy in right middle
temporal gyrus than lvPPA (z=3.5, p\textless{}0.01) and PCA patients
(z=3.2, p\textless{}0.01); in left middle temporal gyrus relative to
lvPPA (z=2.7, p\textless{}0.03); in right precuneus relative to lvPPA
(z=2.5, p\textless{}0.04) and fvAD (z=3.0, p\textless{}0.02); and in
right supramarginal gyrus relative to all three naAD groups (all
z\textgreater{}3.3, p\textless{}0.01). We had predicted that naAD
patients would exhibit more gradual atrophy than aAD patients in MTL
structures. However, all patient groups demonstrated significant atrophy
relative to controls in one or more MTL structures (Table 2), and we
found no significant differences between patient groups in atrophy rates
for bilateral hippocampi, entorhinal cortex, or parahippocampal gyri. To
address limitations in statistical power, we performed a supplementary
analysis on MTL ROIs in which all naAD phenotypes were combined into a
single group; while both the naAD and aAD groups had significantly
faster atrophy than controls in all 6 MTL regions, we again observed no
difference in atrophy rates between naAD and aAD (Supplementary Table
3).

\begin{figure}

{\centering \includegraphics{./fig/roi_time_diffs-1} 

}

\caption{Patient group differences in the effect of time for a priori regions of interest. The plot displays annualized change in regional grey matter volume in each group, expressed in z-score units relative to the healthy control reference group and adjusted for sex as well as age and global cognition at initial MRI. More negative values on the x-axis indicate more rapid atrophy over time. The vertical bar in each box indicates the median volume; the edges of the box represent the interquartile range (IQR), i.e., the difference between the first and third quartiles. The whiskers extend to the most extreme point within $1.5\times IQR$ from the left or right edge of the box; observations falling outside this range are plotted individually. The notches in each box extend $1.58\times IQR\div{\sqrt{n}}$, displaying an approximate 95\% confidence interval for the median. Black brackets indicate significant pairwise group differences (p<0.05 after false discovery rate correction). Group is indicated by color as well as by the shape centered over the median line in each box.}\label{fig:roi_time_diffs}
\end{figure}

\subsection*{Exploratory whole-brain
analysis}\label{exploratory-whole-brain-analysis}
\addcontentsline{toc}{subsection}{Exploratory whole-brain analysis}

\begin{figure}[]
% Voxelwise cortical thickness versus controls.

{\centering \includegraphics[width=7in]{./fig/me_time.pdf} 

}

\caption[GM time effects]{Voxelwise differences in cortical thickness relative to matched controls. Image overlays are binarized t-statistic maps for simple contrasts of controls minus each patient group. Blue: simple effect of group (patients<controls) from cross-sectional analysis of participants' initial MRI scans; red: group x time interaction from longitudinal mixed effects models, indicating where patients have more rapid cortical thinning than controls; green: overlap between group and group x time effects. All results were thresholded at voxelwise p<0.001 with a minimum cluster volume of \SI{584}{\micro\litre} for baseline results and \SI{600}{\micro\litre} longitudinal results, corresponding to a corrected cluster-wise threshold of p<0.05. \label{gm_me_time}}

\end{figure}

Exploratory, whole-brain analysis of cortical thickness was performed to
identify areas of early atrophy and later spread that were not captured
by \emph{a priori} ROIs. As in ROI analysis (Table 2), areas were
categorized by whether they exhibited significant atrophy at patients'
first MRI and whether they exhibited significant longitudinal change
during the follow-up period relative to controls. As mentioned above,
the hippocampi were excluded from voxelwise analysis due to the
difficulty of reliably segmenting and estimating cortical thickness for
this structure (Han \emph{et al.}, 2006; Gronenschild \emph{et al.},
2012; Schwarz \emph{et al.}, 2016).

\subsubsection*{Voxelwise cortical thickness differences at initial
MRI}\label{voxelwise-cortical-thickness-differences-at-initial-mri}
\addcontentsline{toc}{subsubsection}{Voxelwise cortical thickness
differences at initial MRI}

Whole-brain atrophy patterns at initial MRI corroborated ROI-based
analyses and indicated areas of earlier neurodegeneration that fell
outside of \emph{a priori} ROIs. At initial MRI, the lvPPA group
exhibited lower cortical thickness vs.~controls in left middle and
superior temporal gyri, our hypothesized disease focus for lvPPA,
corroborating ROI volume analysis (Figure 3A). In addition to these
regions, lvPPA patients exhibited early atrophy in multiple
left-hemisphere temporal, parietal, and frontal areas including central
and parietal opercula; planum temporale; planum polare; and inferior
temporal, fusiform, supramarginal, angular, inferior occipital, and
middle occipital gyri (Figure 3). In prefrontal cortex, lvPPA patients
had cortical thinning in left anterior insula and frontal operculum as
well as bilateral middle and superior frontal gyri. Moreover, nearly all
of these areas continued to exhibit longitudinal change during the
follow-up period (Figure 3, green areas). Peak t-statistics and cluster
volumes for these regions are reported in Supplementary Table 4.
Voxelwise analysis of the PCA group not only demonstrated expected
atrophy in right parietal, occipital, and posterior temporal areas, but
also in their left-hemisphere homologues (Figure 3B). Additionally, PCA
patients' baseline atrophy extended into right precentral, middle
frontal, and superior frontal gyri. Among these areas, the bilateral
precuneus/posterior cingulate gyrus and middle temporal gyrus continued
to demonstrate change during the follow-up period. Overall, baseline
results thus indicated that despite some right lateralization of
disease, PCA patients in the current sample had bilateral cortical
involvement consistent with recent consensus criteria for PCA (Crutch
\emph{et al.}, 2017). As in ROI-based analysis, fvAD patients exhibited
initial atrophy relative to controls in left anterior insula and middle
frontal gyrus, right angular gyrus, and bilateral middle temporal gyri.
However, areas of early atrophy extended far beyond these regions to
include right insula and middle frontal gyrus as well as bilateral
medial and ventral prefrontal cortex, inferior and superior frontal
gyri, temporal poles, and opercular cortex (Figure 3C). The fvAD group
also had initial atrophy relative to controls in the anterior and dorsal
portion of right entorhinal cortex, a finding that was not captured by
ROI-based analysis. In apparent contrast to the findings of Ossenkoppele
and colleagues (Ossenkoppele \emph{et al.}, 2015b), posterior atrophy
was limited, most notably including the right precuneus. Among areas of
initial atrophy in fvAD, only right anterior insula and bilateral
central opercula displayed significant cortical thinning over the
follow-up period. The aAD group exhibited expected atrophy in right
entorhinal cortex as well as bilateral middle and superior temporal
gyri, partially replicating ROI-based findings (Figure 3D). Outside
\emph{a priori} ROIs, aAD patients also exhibited early atrophy in
bilateral parietal areas including the precunei and middle cingulate,
posterior cingulate, angular, and supramarginal gyri; right insula; and
right frontal lobe areas including anterior orbital, middle frontal,
superior frontal, and medial precentral gyri (Supplementary Table 4). Of
these areas, only the right insula demonstrated continued atrophy
throughout the follow-up period. Figure 4 presents contrasts between
patient groups of initial cortical thickness. Consistent with
expectations from previous cross-sectional studies, these results
indicate left lateralized atrophy in lvPPA (Figure 4A); parietal and
occipitotemporal disease in PCA that exhibits some right-hemisphere bias
(Figures 4A, C, and F), and greater frontal lobe involvement in fvAD
than in other phenotypes (Figures 4C, E, and F). Collectively, these
results replicate initial volume differences from ROI-based analysis and
highlight additional phenotype-specific areas of atrophy reported in
prior studies of lvPPA (Rogalski \emph{et al.}, 2016), PCA (Lehmann
\emph{et al.}, 2012), and fvAD (Whitwell \emph{et al.}, 2011).

\begin{figure}[]
% Voxelwise group differences in cortical thickness.

{\centering \includegraphics[width=7in]{./fig/vox_base_diff.pdf} 

}

\caption[GM mean differences]{Voxelwise differences between patient groups in cortical thickness at time of initial MRI scan. Results are thresholded at voxelwise p<0.001 with a minimum cluster volume of \SI{584}{\micro\litre}, corresponding to a corrected cluster-wise threshold of p<0.05. Warm colors indicate thinner cortical grey matter in the second group than the first; cool colors indicate thinner cortical grey matter in the first group than the second. \label{gm_mean_diff}}

\end{figure}

\begin{figure}[]
% Voxelwise group differences in rates of longitudinal GM change.

{\centering \includegraphics[width=7in]{./fig/vox_group_time.pdf} 

}

\caption[GM group x time differences]{Voxelwise differences between patient groups in rates of cortical thinning over time. Image overlays are t-statistic maps for the interaction of each group with time, calculated from linear mixed effects (LME) models and thresholded at voxelwise p<0.001 with a minimum cluster volume of \SI{600}{\micro\litre}, corresponding to a corrected cluster-wise threshold of p<0.05. Warm colors indicate that cortical thinning over time is more rapid in the second group than the first; cool colors indicate that cortical thinning is more rapid in the first group than the second. \label{gm_group_time}}

\end{figure}

\subsubsection*{Voxelwise whole-brain analysis of longitudinal disease
spread}\label{voxelwise-whole-brain-analysis-of-longitudinal-disease-spread}
\addcontentsline{toc}{subsubsection}{Voxelwise whole-brain analysis of
longitudinal disease spread}

Longitudinal whole-brain analysis also allowed us to identify brain
areas that were not significantly atrophied at baseline but demonstrated
progressive atrophy over the follow-up period. As in ROI-based analysis,
we interpret these effects to indicate the spread of disease to brain
areas that were relatively spared in early disease stages. The lvPPA
group showed extensive new atrophy in right temporoparietal areas and
throughout bilateral prefrontal, medial parietal, and anterior temporal
cortex (Figure 3A, red regions), suggesting spread of disease to these
areas following patients' initial scans. In PCA, progressive atrophy was
observed in several areas unaffected at initial MRI, including the
temporal poles, bilateral superior frontal gyri, and bilateral
perisylvian cortex (Figure 3B, Supplementary Table 5). In contrast to
lvPPA and PCA patients, areas of newer atrophy progression were sparse
among fvAD patients, limited to portions of right anterior insula as
well as left opercular and perisylvian cortex (Figure 3C). Because the
small sample size of this group might have limited statistical
sensitivity, we also present voxelwise contrasts vs.~controls at a
liberal statistical threshold of p\textless{}0.01, without cluster-wise
correction for multiple comparisons (Supplementary Figure 3). While
these results must be interpreted with caution due to the potential for
false positive results, they suggest more extensive disease spread to
left posterior insula, left dorsolateral prefrontal cortex, and
bilateral anterior prefrontal areas. Finally, aAD patients showed new
longitudinal change during the follow-up period in bilateral parietal
cortex as well as right posterior temporal, anterior temporal,
opercular, and prefrontal areas (Figure 3D, Supplementary Table 5).

\subsubsection*{Voxelwise whole-brain differences in regional rates of
change}\label{voxelwise-whole-brain-differences-in-regional-rates-of-change}
\addcontentsline{toc}{subsubsection}{Voxelwise whole-brain differences
in regional rates of change}

As in ROI-based analysis, we assessed group differences in the regional
pace of cortical thinning over time. Consistent with ROI-based analysis
(Figure 2), lvPPA patients had significantly more rapid atrophy than aAD
patients in left anterior and posterior superior/middle temporal gyri
(Figure 5A). Additionally, aAD patients exhibited faster atrophy
progression than fvAD patients in right middle occipital gyrus and
superior parietal lobule (Figure 5B), consistent with parietal
differences observed between these groups in ROI-based analysis.
Similarly, lvPPA patients exhibited more rapid atrophy than fvAD
patients in left precuneus and bilateral middle occipital gyri (Figure
5C). These results corroborate ROI-based findings that suggest
neocortical rates of atrophy may vary by region according to patient
phenotype.

\subsubsection*{Degree of structural connectivity predicts longitudinal
atrophy}\label{degree-of-structural-connectivity-predicts-longitudinal-atrophy}
\addcontentsline{toc}{subsubsection}{Degree of structural connectivity
predicts longitudinal atrophy}

Nodes in the AAL region graph had a median degree of 19.5 (interquartile
range=17--25.75). Nodes in the top quartile corresponded to several
\emph{a priori} ROIs, including bilateral superior parietal lobules;
bilateral inferior, middle, and superior temporal gyri; and right
angular gyrus. Both left and right hippocampus labels (which encompassed
proximal MTL structures) had degrees of 17; left insula, 23; left middle
frontal gyrus, 21; right supramarginal gyrus, 21; right precuneus, 22;
and left and right precentral gyri, 21 and 18, respectively. Node degree
was positively associated with regions' average volume among controls
(Pearson's R=0.46, p\textless{}0.001); to account for this potential
confound, average control volume for each region was included as a
covariate. Linear mixed effects modeling showed that higher node degree
predicted greater annualized GM volume loss in each patient group
relative to controls (Figure 6), as evidenced by group x degree
interaction terms: for lvPPA, \(\beta\)=-0.011, t(6703)=-6.4,
p\textless{}0.001; for PCA, \(\beta\)=-0.0059, t(6703)=-3.1,
p\textless{}0.001; for fvAD, \(\beta\)=-0.0049, t(6703)=-2.2,
p\textless{}0.03; and for aAD, \(\beta\)=-0.010, t(6703)=-5.3,
p\textless{}0.001. The main effect of degree was marginally significant
{[}\(\beta\)=-0.0020, t(6703)=1.7, p\textless{}0.09{]}, reflecting the
lack of substantial GM volume loss in the control group (Figure 6).
Among covariates, volume at participants' first MRI was significantly
associated with annualized change {[}\(\beta\)=-0.024, t(6703)=-6.2,
p\textless{}0.001{]}, as was average region size among controls
{[}\(\beta\)=-13.2, t(6703)=-8.3, p\textless{}0.001{]}. Simple effects
of group were not significant, although the fvAD group had marginally
lower volumes relative to controls {[}\(\beta\)=-0.15, t(103)=-1.9,
p\textless{}0.06{]}. Effects of age and sex were also non-significant
(both p\textgreater{}0.12). In pairwise post-hoc contrasts, the aAD
group had significantly greater overall volume loss than the lvPPA group
(z=2.3, p\textless{}0.02). Additionally, the aAD group had a more
negative slope of association between degree and longitudinal change
than the PCA group (z=2.0, p\textless{}0.05) and the fvAD group (z=2.1,
p\textless{}0.04), suggesting that brain connectivity predicted greater
atrophy in aAD than these other phenotypes. Other contrasts of group and
group x degree interaction terms were non-significant (all
z\textless{}0.6, p\textgreater{}0.5). Importantly, these associations
were based on GM volumes estimated for AAL labels and are thus
unaffected by differences in the AAL and Mindboggle parcellation
schemes. Furthermore, because the present analysis relies on
population-averaged connectivity values for all groups, it does not
address potential connectivity differences between naAD and aAD in
networks associated with language, visuospatial function, social
behavior, executive control, and memory.

\begin{figure}

{\centering \includegraphics{./fig/aal_transmission-1} 

}

\caption{Node degree predicts annualized grey matter volume decline among study participants in regions defined by the Automated Anatomical Labeling (AAL) atlas. Node degree is based on structural connectivity measures computed by Yeh et al. (2018) and reflects the number of white matter connections that each AAL region has with other regions. Shaded areas show the pointwise 95\% confidence interval for each regression line.}\label{fig:aal_transmission}
\end{figure}

\subsubsection*{Effects of global cognition and
age}\label{effects-of-global-cognition-and-age}
\addcontentsline{toc}{subsubsection}{Effects of global cognition and
age}

Initial MMSE score (which was included as a measure of global cognitive
impairment) was positively associated with GM volume in the majority of
ROIs {[}all t(103)\textgreater{}=2.3, p\textless{}0.03{]}, with the
exception of bilateral hippocampi and parahippocampal gyri {[}all
t(103)\textless{}1.0, p\textgreater{}0.3{]}. In contrast, age at initial
MRI was inversely associated with volume in all 6 MTL regions
investigated, including left and right hippocampus {[}t(103)=-2.5,
p\textless{}0.02 and t(103)=-3.2, p\textless{}0.02, respectively, after
false discovery rate correction{]}, left and right parahippocampal gyri
{[}t(103)=-3.7, p\textless{}0.001 and t(103)=-4.9, p\textless{}0.001{]},
and left and right entorhinal cortex {[}t(103)=-3.2, p\textless{}0.002
and t(103)=-3.8, p\textless{}0.001{]}. In addition, age effects were
observed in bilateral precentral gyri {[}both t(103)\textless{}-2.9,
p\textless{}0.01{]}, suggesting age-related atrophy in motor cortex. No
other ROIs displayed an effect of age. To determine whether this age
effect differed by group, we performed secondary analyses on MTL volumes
at the time of first scan using multiple regression models with
predictors of group, age, and their interaction, covarying for MMSE
score and the interval between MMSE and MRI. After false discovery rate
correction, no MTL regions showed a significant group x age interaction
{[}all F(4,99)\textless{}2.3, p\textgreater{}0.2{]}, suggesting that the
association of increased age with MTL atrophy was similar across groups.
Age and MMSE effects for the exploratory voxelwise analysis are shown in
Supplementary Figure 1. Consistent with ROI-based results, voxelwise
associations with baseline MMSE score were distributed throughout all
lobes of the brain (Supplementary Figure 1, bottom). Voxelwise analysis
further showed robust age effects in the MTL as well as the precentral
gyri, anterior temporal lobes, and ventral prefrontal cortex.
Conversely, age was positively associated with cortical thickness in the
precuneus, which exhibits greater atrophy in earlier-onset than
later-onset Alzheimer's disease (Möller \emph{et al.}, 2013). No
significant effects of sex were observed in either ROI-based or
voxelwise analysis.

\subsubsection*{Effects of APOE
genotype}\label{effects-of-apoe-genotype}
\addcontentsline{toc}{subsubsection}{Effects of APOE genotype}

We additionally assessed the distribution of APOE genotypes among aAD
and naAD patients. Genotyping data were unavailable for one lvPPA, one
PCA, one fvAD, and two aAD patients. One aAD, three PCA, and three fvAD
patients each carried one copy of the \(\epsilon2\) allele, which is
associated with lower risk for Alzheimer's disease (Corder \emph{et
al.}, 1994). The proportions of lvPPA, PCA, fvAD, and aAD patients
carrying 1--2 copies of the APOE \(\epsilon4\) allele were 29.2\%,
36.8\%, 63.6\%, and 73.3\%, respectively. The frequency of individuals
carrying zero, one, or two copies of the \(\epsilon4\) allele
significantly differed across patient groups {[}\(\chi^2\)(6)=14.9,
p\textless{}0.02{]}. In post-hoc comparisons, these frequencies differed
between the lvPPA and PCA groups {[}\(\chi^2\)(1)=8.3,
p\textless{}0.02{]} and between the PCA and fvAD groups
{[}\(\chi^2\)(1)=6.1, p\textless{}0.05{]}. Because APOE genotypes were
unavailable for control participants, we recomputed LMEs for \emph{a
priori} ROIs using lvPPA (the largest group) as the reference group and
number of \(\epsilon4\) alleles as a covariate. No ROIs exhibited an
association with number of \(\epsilon4\) alleles independent of group
(all p\textgreater{}0.2).

\subsubsection*{Longitudinal associations between neuropsychological
performance and grey matter
volume}\label{longitudinal-associations-between-neuropsychological-performance-and-grey-matter-volume}
\addcontentsline{toc}{subsubsection}{Longitudinal associations between
neuropsychological performance and grey matter volume}

Associations between longitudinal neuropsychological performance and
concurrent GM volume loss were evaluated in patients that had two or
more assessments, each within one year of a structural MRI scan. This
longitudinal analysis contrasts with previous studies that have inferred
associations by correlating brain imaging data from a single timepoint
with cognitive change. For recognition memory, this yielded 121
observations from 51 patients, including 21 lvPPA, 13 PCA, eight aAD,
and nine fvAD patients. For letter fluency, 129 observations were
available from 20 lvPPA, 14 PCA, 12 aAD, and nine fvAD patients. For
forward and reverse digit span, 117 observations were available from 21
lvPPA, 13 PCA, eight aAD, and eight fvAD patients. A total of 90 PBAC
observations, from which all other test measures were obtained, were
available for 17 lvPPA, 11 PCA, seven aAD, and seven fvAD patients. In
all cognitive domains except for social behavior, longitudinal cognition
was directly associated with GM volume change in one or more associated
brain areas, independent of volume at the time of initial MRI
(Supplementary Table 12). In the memory domain, volume loss in bilateral
hippocampi and left entorhinal predicted declines in recognition
discrimination. In the language domain, volume loss in left middle and
superior temporal gyri was associated with decreases in letter fluency
and forward digit span. In the visuospatial domain, Rey figure copy
performance over time was associated with volume loss in right angular,
middle temporal, and supramarginal gyrus as well as right precuneus.
Judgment of line orientations was likewise associated with right
precuneus volume change. No significant associations were found for
social behavior; however, left middle frontal gyrus, left anterior
insula, and right middle temporal gyrus predicted reverse digit span, a
measure of working memory and executive function (Kramer \emph{et al.},
2003).

\section*{Discussion}\label{discussion}
\addcontentsline{toc}{section}{Discussion}

Our previous work (Phillips \emph{et al.}, 2018) used cross-sectional
analyses to identify areas of frequent atrophy in several AD phenotypes,
which we hypothesized to be likely regions of disease onset. This
approach was inspired by pathological staging studies that have inferred
disease progression over time in AD (Braak and Braak, 1991), Parkinson's
disease (Braak \emph{et al.}, 2003), and amyotrophic lateral sclerosis
(Brettschneider \emph{et al.}, 2013) based on postmortem pathology.
However, this cross-sectional design did not allow us to directly
observe within-patient effects of earlier vs.~later disease progression
in each phenotype. The current study compared longitudinal disease
progression in multiple clinically-defined naAD phenotypes with autopsy
or CSF evidence of Alzheimer's disease pathology. We differentiated
earlier and later disease stages through a two-part approach. ROI-based
analysis allowed us to focus on GM volume change in the most likely
sites of disease onset for each phenotype (Phillips \emph{et al.},
2018). A second, exploratory analysis of whole-brain cortical thickness
values allowed us to examine disease spread outside this cluster of
\emph{a priori} ROIs. In each phenotype, we observed a combination of
local spread surrounding areas of early atrophy and distal spread to
brain areas that were not significantly atrophied at the beginning of
the follow-up period. Both patterns of initial atrophy and subsequent
progression differed between phenotypes. We further found that
longitudinal rates of neurodegeneration differed across patient groups
in phenotype-specific neocortical disease foci, a result which could at
least partially account for each phenotype's characteristic disease
distribution. In contrast, we observed no evidence of
phenotype-dependent differences in atrophy rates within the MTL,
although MTL atrophy appeared to begin later in naAD than in aAD.
Finally, we found that structural connectivity, assessed by node degree,
was a significant predictor of GM volume loss over time in both aAD and
naAD; this result supports brain connectivity as a general factor
mediating atrophy progression in AD.

\subsection*{Atrophy at initial MRI indicates possible sites of early
disease}\label{atrophy-at-initial-mri-indicates-possible-sites-of-early-disease}
\addcontentsline{toc}{subsection}{Atrophy at initial MRI indicates
possible sites of early disease}

We hypothesized that each naAD phenotype would be characterized by a
distinct pattern of early atrophy, observed through cross-sectional
contrasts of patients' first MRI scans. We consider significant baseline
atrophy an expected and necessary marker for identifying potential sites
of disease onset, although early atrophy alone is not sufficient to
determine these onset sites. \emph{A priori} ROIs for each phenotype
(including left temporal cortex in lvPPA, posterior temporoparietal
cortex in PCA, prefrontal cortex in fvAD, and the MTL in aAD)
demonstrated significant initial atrophy, consistent with hypotheses.
However, the lvPPA and PCA groups also exhibited lateralized hippocampal
atrophy vs.~controls at initial MRI; although this atrophy was mild
relative to the aAD group (Figure 1), we cannot rule out early,
lateralized hippocampal disease in these phenotypes. Longitudinal
imaging of patients from earlier disease stages, when atrophy will
presumably be more focal than in the current sample, is thus necessary
to conclusively determine whether focal neocortical disease precedes,
follows, or arises concurrently with MTL disease in these phenotypes.
Nevertheless, the current study narrows the field of brain areas where
disease is likely to originate in each naAD phenotype, providing a
valuable prior constraint on future hypothesis testing. Overall, we
propose that the current results are more consistent with the prevailing
hypothesis that naAD patients have disease originating in the neocortex,
as inferred by cross-sectional or single-group longitudinal imaging
studies (Rogalski \emph{et al.}, 2011; Lehmann \emph{et al.}, 2012;
Rohrer \emph{et al.}, 2013; Ossenkoppele \emph{et al.}, 2015a; Xia
\emph{et al.}, 2017; Phillips \emph{et al.}, 2018) as well as autopsy
studies of hippocampal-sparing AD (Giannakopoulos \emph{et al.}, 1994;
Murray \emph{et al.}, 2011; Ferreira \emph{et al.}, 2017). Phenotypic
variability in initial atrophy patterns (Table 2), including sparing of
primary motor cortex at the time of initial MRI, supports the regional
specificity of atrophy in naAD patients.

Interestingly, some areas of initial atrophy continued to change over
time, while others did not. From the data at our disposal, we cannot say
with certainty what differentiates these regions. One statistical
explanation is simply that variability prevented reliable detection of
longitudinal atrophy in some regions and phenotypes. An alternative,
biological explanation is that areas that failed to exhibit further
change over the follow-up period (blue areas, Table 2 and Figure 3) had
already undergone massive atrophy by the time of patients' first MRI,
reaching a plateau determined by the limited amount of remaining GM
tissue (Sabuncu \emph{et al.}, 2011; Schuff \emph{et al.}, 2012). The
right superior parietal lobule in PCA and left middle frontal gyrus in
fvAD may exemplify such slowing: in ROI-based analysis, both regions
were severely atrophied at initial MRI and did not significantly
progress over time in their respective phenotypes. Further research is
needed to determine why the pace of atrophy changes in some areas of
early degeneration but not others.

\subsection*{Differences between phenotypic groups in the neocortical
spread of
atrophy}\label{differences-between-phenotypic-groups-in-the-neocortical-spread-of-atrophy}
\addcontentsline{toc}{subsection}{Differences between phenotypic groups
in the neocortical spread of atrophy}

Areas that exhibited longitudinal atrophy in the absence of initial
cross-sectional differences provide a window onto disease spread in each
phenotype. In ROI-based analysis, the lvPPA group showed strong left
lateralization of atrophy at baseline, consistent with prior studies
(Rogalski \emph{et al.}, 2016; Phillips \emph{et al.}, 2018). This
pattern included left lateral temporal cortex, an area specifically
associated with language deficits in lvPPA (Gorno-Tempini \emph{et al.},
2011). ROI-based analysis also indicated early atrophy in left
prefrontal cortex, anterior insula, and hippocampus; and right lateral
temporal areas. Voxelwise analysis indicated atrophy in left precuneus
and right prefrontal cortex as well. Over the follow-up period, we
observed new progressive atrophy in brain areas both adjacent to and
distal from these areas of initial atrophy. Proximal disease spread was
observed throughout the left temporal and parietal lobes as well as
bilateral frontal lobes. This proximal atrophy may indicate diffusive
spread of pathology through the extracellular medium or along
short-distance axonal connections between neighboring cells in cortex
(Guo and Lee, 2014). However, we also observed progression through parts
of bilateral frontal lobes and right temporoparietal cortex distal from
foci of initial atrophy (Figure 3); diffusive spread from adjacent
disease areas appears insufficient to account for this progression. Two
possible explanations may account for new, distal atrophy progression.
First, pathology may have arisen independently in these areas; second,
pathogenic proteins may have been transmitted to these areas via
long-distance white matter projections, according to the transmission
hypothesis of neurodegenerative disease (Guo and Lee, 2014). It is
particularly interesting to consider these two possibilities with
respect to cross-sectional reports of right temporal atrophy in lvPPA,
which---if observed---tends to be much milder than left temporal
atrophy. In such cases, it is tempting to infer that right temporal
atrophy results from the spread of disease from left to right
hemispheres via callosal projections. However, this apparent
``progression'' may result from a subset of patients having bilateral
disease. The current study cannot rule out this possibility, as
ROI-based analysis indicated right temporal atrophy that pre-dated lvPPA
patients' first MRI (Table 2). Earlier recruitment and longitudinal
imaging of patients with language disturbances is thus necessary to
conclusively demonstrate interhemispheric disease spread in lvPPA.

The PCA group also exhibited a combination of proximal and distal
disease spread. Initial atrophy was observed in bilateral precunei and
temporoparietal regions (Figure 3) as well as right hippocampus (Table
2). These parietal areas, in particular, are important to visuospatial
processing (Astafiev \emph{et al.}, 2003; Greenberg \emph{et al.}, 2010;
Gmeindl \emph{et al.}, 2016) and are consistent with early disease
patterns observed in prior studies of PCA (Tang-Wai \emph{et al.}, 2004;
Lehmann \emph{et al.}, 2012). In voxelwise analysis, PCA patients had
newer atrophy extending from areas of early disease into inferior
parietal, posterior temporal, and insular/opercular cortex; they also
exhibited spread proximal to areas of prefrontal atrophy observed at
baseline. In addition, however, the PCA group exhibited atrophy
progression in the anterior temporal lobes distal from any cluster of
existing atrophy (Figure 3). This finding suggests testable hypotheses
regarding the diffusion of disease-causing agents along fiber pathways
that terminate in anterior temporal cortex. These pathways include
projections from MTL areas as well as more distal connections via the
inferior longitudinal fasciculus to striate and prestriate cortex, which
may in turn connect with parietal cortex (Nieuwenhuys \emph{et al.},
2008).

In the fvAD group, ROI-based and voxelwise analysis collectively
indicated GM volume loss at initial MRI in bilateral prefrontal,
temporal, and anterior insular cortex as well as right middle cingulate
and angular gyri. The involvement of the insula is particularly
interesting given this group's behavioral dysfunction, as anterior
insula is crucially implicated in primates' emotion (Phan \emph{et al.},
2002) as well as in empathy and social life (Singer, 2006). The anterior
insula is also implicated in behavioral-variant frontotemporal dementia
(bvFTD) (Seeley, 2010), and Ossenkoppele et al. (2015a) found that
insula was one of the few regions of atrophy specific to
behavioral-variant AD (bvAD) patients who were initially misdiagnosed as
bvFTD. While our findings suggest early involvement of frontal,
temporal, and limbic regions, previous studies of
behavioral/dysexecutive AD have shown either predominantly frontal
(Blennerhassett \emph{et al.}, 2014) or predominantly temporal
(Ossenkoppele \emph{et al.}, 2015a) disease. In ROI-based analysis, the
fvAD group demonstrated new atrophy progression only in left perisylvian
cortex; voxelwise analysis indicated additional disease progression in
bilateral insular/opercular cortex. These findings are located
proximally to atrophy clusters observed at first MRI and thus may
reflect local, diffusive spread of disease. Although more distal atrophy
progression was not observed, we emphasize that null results in this
group should be interpreted with extreme caution due to the small sample
size; while the reported foci may represent the areas of most robust
atrophy in the current sample, true disease progression may be missed
due to type II statistical error and may be more anatomically widespread
than reported here.

In the aAD group, ROI-based analysis showed new neocortical atrophy in
bilateral precentral gyri as well as right temporoparietal cortex (Table
2). Voxelwise analysis similarly indicated neocortical atrophy
progression throughout the right temporal lobe as well as in bilateral
parietal cortex and right prefrontal cortex. The slight lateralization
of disease progression (right hemisphere \textgreater{} left) may be
incidental to the current sample, and we do not propose that it is
characteristic of amnestic AD generally. However, the results are
broadly consistent with spreading neocortical disease in later Braak
stages (Braak and Braak, 1991). Clusters of newer atrophy in right
temporal cortex may indicate local, diffusive spread from right angular
gyrus, which was atrophied at initial MRI in the aAD group. However,
other areas of new progression observed in the voxelwise analysis
(Figure 3) are distal from sites of early atrophy and may result from
either white-matter-mediated disease spread or \emph{de novo}
accumulation of pathology. Notably, structural connectivity data from
healthy adults indicates that the superior parietal lobule is connected
to the hippocampus and angular gyrus (Supplementary Table 6), both of
which exhibited baseline atrophy in the aAD group; newer areas of
superior parietal atrophy in aAD may thus result from disease
transmission along white-matter pathways connecting these areas.

\subsection*{MTL atrophy in amnestic and non-amnestic
phenotypes}\label{mtl-atrophy-in-amnestic-and-non-amnestic-phenotypes}
\addcontentsline{toc}{subsection}{MTL atrophy in amnestic and
non-amnestic phenotypes}

At initial MRI, aAD patients demonstrated significant atrophy in
bilateral entorhinal cortex and hippocampi, as expected from Braak
staging (Braak and Braak, 1991). Bilateral parahippocampal gyri were not
atrophied, but they demonstrated significant change over the follow-up
period; this pattern of results is consistent with progression from
approximate Braak stages IV to V (Whitwell \emph{et al.}, 2008) among
our aAD sample. Based on well-characterized patterns of disease spread
in aAD, these MTL foci may be the source of disease spread to the
neocortex. The hippocampus has well-characterized white-matter
connections to posterior cortical areas via the posterior cingulate
(Nieuwenhuys \emph{et al.}, 2008; Teipel \emph{et al.}, 2010); these
pathways thus represent tracts of interest for investigating the spread
of pathogenic proteins to the neocortex.

NaAD patients, in turn, demonstrated relative sparing of MTL structures
at baseline. In ROI-based analysis, atrophy was limited to left
hippocampus in lvPPA and right hippocampus in PCA; these patterns of
lateralization were consistent with the general hemispheric bias
observed in both phenotypes. Over the follow-up period, naAD patients
demonstrated significant atrophy progression in the MTL (Table 2); in
the fvAD group, these changes were limited to bilateral entorhinal
cortex, although null findings in other MTL structures may reflect the
small size of this group. MTL progression in naAD patients suggests that
sparing of the hippocampus and surrounding MTL (a set of
clinically-defined syndromes) is a graded rather than an absolute
phenomenon, and that naAD patients may become increasingly susceptible
to hippocampal degeneration at older ages and in more advanced disease.
Indeed, age was a strong predictor of MTL atrophy, as evidenced by both
ROI-based results (see above, ``Effects of global cognition and age'')
and voxelwise results (Supplementary Figure 1). We note that while naAD
patients tend to be younger than typical aAD patients, the current study
controlled for this potential confound both by demographic balancing of
groups and by covarying for age in statistical models. Thus, baseline
differences between aAD and naAD patients in GM volume within the MTL
(Figure 1) were not attributable to age differences between these
patient groups. Seminal studies of hippocampal sparing in AD
(Giannakopoulos \emph{et al.}, 1994; Murray \emph{et al.}, 2011;
Whitwell \emph{et al.}, 2012) grouped patients based on postmortem
pathology findings; these studies may not have included patients who
initially presented with non-amnestic syndromes but developed
hippocampal pathology in later disease.

\subsection*{Differences between phenotypic groups in rates of atrophy
progression}\label{differences-between-phenotypic-groups-in-rates-of-atrophy-progression}
\addcontentsline{toc}{subsection}{Differences between phenotypic groups
in rates of atrophy progression}

The design of the current study not only allowed us to investigate
differences between phenotypic groups in the topographical distribution
of atrophy but also differences in the rate of atrophy within each
region. We reasoned that each phenotype might exhibit more rapid
degeneration within its associated disease foci, reflecting
phenotype-specific susceptibility to disease (Bergeron \emph{et al.},
2016; Mattsson \emph{et al.}, 2016) in that area. Among neocortical
areas associated with naAD phenotypes, we found evidence to support this
reasoning. LvPPA patients demonstrated more rapid atrophy in the left
temporal cortex than the PCA group (ROI-based analysis, Figure 2) and
the aAD group (voxelwise analysis, Figure 5). FvAD patients, in turn,
had more rapid atrophy in left anterior insula than both PCA and lvPPA
patients in ROI-based analysis (Figure 2); and voxelwise analysis
indicated additional prefrontal, temporal, and insular differences
between fvAD and lvPPA (Figure 5C). Contrary to our initial hypotheses,
we saw no difference in MTL atrophy rates between aAD and naAD patients,
even when all three naAD variants were combined to enhance statistical
power. Considered together with aAD patients' significant MTL atrophy at
initial MRI, this result suggests that relative MTL sparing in naAD may
result from a delayed onset of degeneration in these structures; but
that once neurodegeneration has begun, it proceeds at a similar rate as
in aAD. However, we caution that these findings warrant replication in
longitudinal studies involving larger sample sizes.

\subsection*{Associations between longitudinal atrophy and brain
connectivity}\label{associations-between-longitudinal-atrophy-and-brain-connectivity}
\addcontentsline{toc}{subsection}{Associations between longitudinal
atrophy and brain connectivity}

To investigate the possible role of brain connectivity in mediating
disease spread, we related patients' atrophy patterns to
population-average structural connectivity, as estimated from Human
Connectome Project white-matter imaging data (Yeh \emph{et al.}, 2018).
Several \emph{a priori} ROIs in the current study corresponded to hubs
in Yeh et al.`s structural connectivity matrix, as evidenced by their
high node degree. These findings replicate established functional
connectivity results that have related the neuroanatomy of AD to brain
network hubs including bilateral middle temporal, inferior parietal, and
superior parietal cortex (Buckner \emph{et al.}, 2009; Crossley \emph{et
al.}, 2014). Moreover, we found that node degree was a significant
predictor of regional GM volume loss over time in each of the patient
groups. We caution that this result is correlative in nature and does
not demonstrate long-distance disease spread along white-matter
pathways. Indeed, network influences on neurodegeneration need not be
limited to physical transport of pathogenic proteins along white-matter
tracts; rather, they may reflect effects such as diaschisis (Chételat,
2018), with disease in one area leading to metabolic and functional
disruptions in its network neighbors. Computational analysis of atrophy
patterns using network models such as those of Raj and colleagues (2012,
2015), Iturria-Medina et al. (2014), and Hu et al. (2016) offers a more
rigorous approach for testing hypotheses regarding disease spread in
brain networks. Nevertheless, associations between atrophy and node
degree provide supporting evidence for the hypothesis that atrophy
progression is mediated by structural connectivity. Importantly, the
structural connectivity analysis reported here does not address
potential hypotheses regarding connectivity differences in naAD and aAD.
Indeed, initial research on this question suggests that the connectivity
of specific brain networks may differ between typical and atypical
presentations of AD (Lehmann \emph{et al.}, 2015; Whitwell \emph{et
al.}, 2015). Because the current study relied on population-averaged
structural connectivity values, it was limited to showing a general
relationship between degree of connectivity and magnitude of
longitudinal change; future analysis of patients' specific connectivity
patterns remains a high priority.

\subsection*{Convergence of atrophy in advanced
disease}\label{convergence-of-atrophy-in-advanced-disease}
\addcontentsline{toc}{subsection}{Convergence of atrophy in advanced
disease}

The spread of disease along white-matter pathways may help explain the
reported convergence of atrophy patterns across AD phenotypes. For
example, Ossenkoppele et al. (2015a) proposed that the common
temporoparietal atrophy observed among lvPPA, PCA, early-onset AD, and
late-onset AD patients could result from convergent disease in nodes of
the posterior default mode network. In support of the convergence
hypothesis, we note that aAD and naAD variants alike had common early
atrophy and subsequent progression in bilateral temporal cortex (Table
2). These results replicate findings from our previous study (Phillips
\emph{et al.}, 2018) of substantial overlap in temporoparietal areas
among typical and atypical AD patients. Indeed, this study reported that
spatial distributions of atrophy become more similar across phenotypes
in later disease phases. At the same time, however, we found that simple
logistic regression models could effectively discriminate naAD and AD
phenotypes from one another based on atrophy patterns, even in advanced
disease (Phillips \emph{et al.}, 2018). Moreover, postmortem studies of
AD variants show that regional differences in pathology burden persist
among AD variants even until the end of life. One proposal for resolving
these apparently conflicting results is the proposal of Warren and
colleagues (Warren \emph{et al.}, 2012) that different clinical
presentations of AD involve a common temporal, parietal, and frontal
network, but that genetic variation or other factors cause the nodes of
these networks to be differentially engaged across syndromes.

\subsection*{Conclusions and
limitations}\label{conclusions-and-limitations}
\addcontentsline{toc}{subsection}{Conclusions and limitations}

Strengths of the current study include a novel comparison of
longitudinal anatomical changes in multiple clinically-defined naAD
phenotypes using both \emph{a priori} ROI-based and whole-brain
voxelwise analyses. The longitudinal study design allowed us to
differentiate areas of earlier and later atrophy and to compare these
patterns of disease progression across phenotypes. Moreover, we sought
to ensure the comparability of the heterogeneous patient groups included
here by controlling for demographic and clinical characteristics both
during sample selection and in statistical analysis. The relevance of
\emph{a priori} ROIs is supported by analyses showing that longitudinal
anatomical change is associated with concurrent domain-specific
cognitive decline.

However, one major limitation was the inability to evaluate non-linear
atrophy progression in Alzheimer's disease: prior evidence suggests that
an initial acceleration due to spreading cumulative damage is followed
by a deceleration due to the reduction of intact tissue (Sabuncu
\emph{et al.}, 2011; Schuff \emph{et al.}, 2012). Such nonlinearities
complicate study design and interpretation in ways that may not be fully
addressed by equating patient groups for chronological age and estimated
disease duration: for example, in the current study, it is possible that
areas of early atrophy in each phenotype (i.e., those exhibiting atrophy
at initial MRI) have entered the deceleration phase, while for other
phenotypes the same regions may have been imaged during the acceleration
phase. Investigating longitudinal change in earlier-stage patients may
allow us to observe a more complete trajectory of neurodegeneration, and
including a minimum of 3--4 imaging timepoints may allow us to
discriminate between linear, quadratic, and sigmoid models of
neurodegeneration. Another possible limitation in our findings is
statistical power, which is likely to have affected voxelwise analysis
more severely than ROI-based analysis due to the much stricter
multiple-comparisons correction of the former. Power limitations may
thus have resulted in underestimation of disease spread in the
relatively small fvAD and aAD groups; we thus emphasize the importance
of further longitudinal study, particularly of patients for whom
postmortem pathological diagnoses are available to rule out the
possibility of co-morbid FTLD or other pathologies. Relatedly, the PBAC
behavioral scale did not demonstrate expected worsening of behavioral
symptoms over time in the fvAD group; while it is possible that this
null result stems from successful treatment of behavioral symptoms
through psychiatric medications, future research should strive to
include more sensitive measures of behavioral dysfunction. An additional
limitation is that patients in the current sample were not selected
based on availability of white-matter imaging data, preventing us from
performing a structured white matter analysis to support interpretations
of disease spread along white matter pathways. The current study was
also not designed to investigate associations with the APOE genotype or
other genetic risk modifiers for Alzheimer's disease. We found that APOE
\(\epsilon4\) allele counts added little predictive power to our imaging
models after accounting for group effects; however, continued study of
the APOE genotype and other genetic risk modifiers in naAD remains an
important research aim. Finally, future studies should include patients
with corticobasal syndrome due to underlying Alzheimer's disease
pathology; insufficient longitudinal data prevented us from including
this uncommon naAD phenotype in the current study.

Understanding the neuropathological and clinical heterogeneity of
Alzheimer's disease is crucial to understanding the mechanisms of its
progression. The current study not only corroborated probable areas of
early disease for lvPPA, PCA, and fvAD but also showed that each
phenotype has a different pattern of atrophy progression across the
cortex. Moreover, we report novel evidence that the longitudinal rate of
neocortical atrophy varies by region and phenotype in naAD, reflecting
phenotype-specific cognitive decline. In contrast, the rate of MTL
atrophy in naAD was similar to that found in aAD, suggesting that early
sparing of these structures results from a later onset of MTL atrophy in
naAD. Finally, we observed associations between longitudinal atrophy and
structural brain connectivity, providing indirect support for models of
interregional disease spread in association with white-matter fiber
pathways in naAD.

\section*{Acknowledegments}\label{acknowledegments}
\addcontentsline{toc}{section}{Acknowledegments}

The authors would like to thank Dr.~Valeria Isella and Dr.~Carlo
Ferrarese for their valuable feedback on this project; and Dr.~Ilya
Nasrallah for assistance with visual reads of PET images. Additionally,
the authors thank Dr.~Fang-Cheng Yeh for making structural connectivity
results available at \url{http://brain.labsolver.org/}.

\section*{Funding}\label{funding}
\addcontentsline{toc}{section}{Funding}

This work was supported by grants from the Alzheimer's Association
(AARF-16-443681), National Institutes of Health (AG017586, AG010124,
AG043503, and NS088341), BrightFocus Foundation (A2016244S), Dana
Foundation, Newhouse Foundation, Wyncote Foundation, Arking Family
Foundation, and the Italian Ministry of Education, University, and
Research.

\section*{Competing interests}\label{competing-interests}
\addcontentsline{toc}{section}{Competing interests}

All authors report that they have no competing interests to disclose.

\section*{References}\label{references}
\addcontentsline{toc}{section}{References}

\hypertarget{refs}{}
\hypertarget{ref-ahmed_novel_2014}{}
Ahmed Z, Cooper J, Murray TK, Garn K, McNaughton E, Clarke H, et al. A
novel in vivo model of tau propagation with rapid and progressive
neurofibrillary tangle pathology: The pattern of spread is determined by
connectivity, not proximity. Acta Neuropathologica 2014; 127: 667--683.

\hypertarget{ref-astafiev_functional_2003}{}
Astafiev SV, Shulman GL, Stanley CM, Snyder AZ, Van Essen DC, Corbetta
M. Functional Organization of Human Intraparietal and Frontal Cortex for
Attending, Looking, and Pointing. The Journal of Neuroscience 2003; 23:
4689--4699.

\hypertarget{ref-avants_reproducible_2011}{}
Avants BB, Tustison NJ, Song G, Cook PA, Klein A, Gee JC. A reproducible
evaluation of ANTs similarity metric performance in brain image
registration. NeuroImage 2011; 54: 2033--2044.

\hypertarget{ref-avants_insight_2014}{}
Avants BB, Tustison NJ, Stauffer M, Song G, Wu B, Gee JC. The Insight
ToolKit image registration framework {[}Internet{]}. Frontiers in
Neuroinformatics 2014; 8{[}cited 2015 Jun 26{]} Available from:
\url{http://www.ncbi.nlm.nih.gov/pmc/articles/PMC4009425/}

\hypertarget{ref-bergeron_untangling_2016}{}
Bergeron D, Bensaïdane R, Laforce R. Untangling Alzheimer's Disease
Clinicoanatomical Heterogeneity Through Selective Network Vulnerability
- An Effort to Understand a Complex Disease. Current Alzheimer Research
2016; 13: 589--596.

\hypertarget{ref-blennerhassett_distribution_2014}{}
Blennerhassett R, Lillo P, Halliday GM, Hodges JR, Kril JJ. Distribution
of pathology in frontal variant Alzheimer's disease. Journal of
Alzheimer's disease: JAD 2014; 39: 63--70.

\hypertarget{ref-braak_neuropathological_1991}{}
Braak H, Braak E. Neuropathological stageing of Alzheimer-related
changes. Acta neuropathologica 1991; 82: 239--259.

\hypertarget{ref-braak_staging_2003}{}
Braak H, Del Tredici K, Rüb U, Vos RAI de, Jansen Steur ENH, Braak E.
Staging of brain pathology related to sporadic Parkinson's disease.
Neurobiology of Aging 2003; 24: 197--211.

\hypertarget{ref-brettschneider_stages_2013}{}
Brettschneider J, Del Tredici K, Toledo JB, Robinson JL, Irwin DJ,
Grossman M, et al. Stages of pTDP-43 pathology in amyotrophic lateral
sclerosis. Annals of neurology 2013; 74: 20--38.

\hypertarget{ref-buckner_cortical_2009}{}
Buckner RL, Sepulcre J, Talukdar T, Krienen FM, Liu H, Hedden T, et al.
Cortical Hubs Revealed by Intrinsic Functional Connectivity: Mapping,
Assessment of Stability, and Relation to Alzheimer's Disease. Journal of
Neuroscience 2009; 29: 1860--1873.

\hypertarget{ref-byun_heterogeneity_2015}{}
Byun MS, Kim SE, Park J, Yi D, Choe YM, Sohn BK, et al. Heterogeneity of
Regional Brain Atrophy Patterns Associated with Distinct Progression
Rates in Alzheimer's Disease. PloS One 2015; 10: e0142756.

\hypertarget{ref-chen_linear_2013}{}
Chen G, Saad ZS, Britton JC, Pine DS, Cox RW. Linear mixed-effects
modeling approach to FMRI group analysis. NeuroImage 2013; 73: 176--190.

\hypertarget{ref-chetelat_multimodal_2018}{}
Chételat G. Multimodal Neuroimaging in Alzheimer's Disease: Early
Diagnosis, Physiopathological Mechanisms, and Impact of Lifestyle.
Journal of Alzheimer's Disease 2018; 64: S199--S211.

\hypertarget{ref-corder_protective_1994}{}
Corder EH, Saunders AM, Risch NJ, Strittmatter WJ, Schmechel DE, Gaskell
PC, et al. Protective effect of apolipoprotein E type 2 allele for late
onset Alzheimer disease. Nature Genetics 1994; 7: 180--184.

\hypertarget{ref-cox_fmri_2017}{}
Cox RW, Chen G, Glen DR, Reynolds RC, Taylor PA. FMRI Clustering in
AFNI: False-Positive Rates Redux. Brain Connectivity 2017; 7: 152--171.

\hypertarget{ref-crossley_hubs_2014}{}
Crossley NA, Mechelli A, Scott J, Carletti F, Fox PT, McGuire P, et al.
The hubs of the human connectome are generally implicated in the anatomy
of brain disorders. Brain 2014; 137: 2382--2395.

\hypertarget{ref-crutch_posterior_2012}{}
Crutch SJ, Lehmann M, Schott JM, Rabinovici GD, Rossor MN, Fox NC.
Posterior cortical atrophy. The Lancet Neurology 2012; 11: 170--178.

\hypertarget{ref-crutch_consensus_2017}{}
Crutch SJ, Schott JM, Rabinovici GD, Murray M, Snowden JS, Flier WM van
der, et al. Consensus classification of posterior cortical atrophy
{[}Internet{]}. Alzheimer's \& Dementia 2017{[}cited 2017 Mar 7{]}
Available from:
\url{http://www.sciencedirect.com/science/article/pii/S1552526017300407}

\hypertarget{ref-dickerson_approach_2017}{}
Dickerson BC, McGinnis SM, Xia C, Price BH, Atri A, Murray ME, et al.
Approach to atypical Alzheimer's disease and case studies of the major
subtypes. CNS spectrums 2017; 22: 439--449.

\hypertarget{ref-duara_regional_2013}{}
Duara R, Loewenstein DA, Shen Q, Barker W, Greig MT, Varon D, et al.
Regional patterns of atrophy on MRI in Alzheimer's disease:
Neuropsychological features and progression rates in the ADNI cohort.
Advances in Alzheimer's Disease 2013; 02: 135--147.

\hypertarget{ref-dubois_advancing_2014}{}
Dubois B, Feldman HH, Jacova C, Hampel H, Molinuevo JL, Blennow K, et
al. Advancing research diagnostic criteria for Alzheimer's disease: The
IWG-2 criteria. The Lancet Neurology 2014; 13: 614--629.

\hypertarget{ref-ferreira_distinct_2017}{}
Ferreira D, Verhagen C, Hernández-Cabrera JA, Cavallin L, Guo C-J, Ekman
U, et al. Distinct subtypes of Alzheimer's disease based on patterns of
brain atrophy: Longitudinal trajectories and clinical applications.
Scientific Reports 2017; 7: 46263.

\hypertarget{ref-forman_improved_1995}{}
Forman SD, Cohen JD, Fitzgerald M, Eddy WF, Mintun MA, Noll DC. Improved
assessment of significant activation in functional magnetic resonance
imaging (fMRI): Use of a cluster-size threshold. Magnetic Resonance in
Medicine 1995; 33: 636--647.

\hypertarget{ref-galton_atypical_2000}{}
Galton CJ, Patterson K, Xuereb JH, Hodges JR. Atypical and typical
presentations of Alzheimer's disease: A clinical neuropsychological,
neuroimaging and pathological study of 13 cases. Brain: A Journal of
Neurology 2000; 123: 484--498.

\hypertarget{ref-giannakopoulos_alzheimers_1994}{}
Giannakopoulos P, Hof PR, Bouras C. Alzheimer's disease with asymmetric
atrophy of the cerebral hemispheres: Morphometric analysis of four
cases. Acta Neuropathologica 1994; 88: 440--447.

\hypertarget{ref-giannini_clinical_2017}{}
Giannini LAA, Irwin DJ, McMillan CT, Ash S, Rascovsky K, Wolk DA, et al.
Clinical marker for Alzheimer disease pathology in logopenic primary
progressive aphasia. Neurology 2017; 88: 2276--2284.

\hypertarget{ref-gmeindl_tracking_2016}{}
Gmeindl L, Chiu Y-C, Esterman MS, Greenberg AS, Courtney SM, Yantis S.
Tracking the Will to Attend: Cortical Activity Indexes Self-Generated,
Voluntary Shifts of Attention. Attention, perception \& psychophysics
2016; 78: 2176--2184.

\hypertarget{ref-gorno-tempini_classification_2011}{}
Gorno-Tempini ML, Hillis AE, Weintraub S, Kertesz A, Mendez M, Cappa SF,
et al. Classification of primary progressive aphasia and its variants.
Neurology 2011; 76: 1006--1014.

\hypertarget{ref-greenberg_control_2010}{}
Greenberg AS, Esterman M, Wilson D, Serences JT, Yantis S. Control of
Spatial and Feature-Based Attention in Frontoparietal Cortex. The
Journal of Neuroscience 2010; 30: 14330--14339.

\hypertarget{ref-gronenschild_effects_2012}{}
Gronenschild EHBM, Habets P, Jacobs HIL, Mengelers R, Rozendaal N, Os J
van, et al. The Effects of FreeSurfer Version, Workstation Type, and
Macintosh Operating System Version on Anatomical Volume and Cortical
Thickness Measurements. PLOS ONE 2012; 7: e38234.

\hypertarget{ref-guo_cell-to-cell_2014}{}
Guo JL, Lee VMY. Cell-to-cell transmission of pathogenic proteins in
neurodegenerative diseases. Nature Medicine 2014; 20: 130--138.

\hypertarget{ref-han_reliability_2006}{}
Han X, Jovicich J, Salat D, Kouwe A van der, Quinn B, Czanner S, et al.
Reliability of MRI-derived measurements of human cerebral cortical
thickness: The effects of field strength, scanner upgrade and
manufacturer. NeuroImage 2006; 32: 180--194.

\hypertarget{ref-hu_localizing_2016}{}
Hu C, Hua X, Ying J, Thompson PM, Fakhri GE, Li Q. Localizing Sources of
Brain Disease Progression with Network Diffusion Model. IEEE journal of
selected topics in signal processing 2016; 10: 1214--1225.

\hypertarget{ref-iba_synthetic_2013}{}
Iba M, Guo JL, McBride JD, Zhang B, Trojanowski JQ, Lee VM-Y. Synthetic
Tau Fibrils Mediate Transmission of Neurofibrillary Tangles in a
Transgenic Mouse Model of Alzheimer's-like Tauopathy. The Journal of
neuroscience : the official journal of the Society for Neuroscience
2013; 33: 1024--1037.

\hypertarget{ref-irwin_comparison_2012}{}
Irwin DJ, McMillan CT, Toledo JB, Arnold SE, Shaw LM, Wang L-S, et al.
Comparison of cerebrospinal fluid levels of tau and A\(\beta\) 1-42 in
Alzheimer disease and frontotemporal degeneration using 2 analytical
platforms. Archives of Neurology 2012; 69: 1018--1025.

\hypertarget{ref-iturria-medina_epidemic_2014}{}
Iturria-Medina Y, Sotero RC, Toussaint PJ, Evans AC, Alzheimer's Disease
Neuroimaging Initiative. Epidemic spreading model to characterize
misfolded proteins propagation in aging and associated neurodegenerative
disorders. PLoS computational biology 2014; 10: e1003956.

\hypertarget{ref-johnson_clinical_1999}{}
Johnson JK, Head E, Kim R, Starr A, Cotman CW. Clinical and pathological
evidence for a frontal variant of Alzheimer disease. Archives of
Neurology 1999; 56: 1233--1239.

\hypertarget{ref-klein_evaluation_2009}{}
Klein A, Andersson J, Ardekani BA, Ashburner J, Avants B, Chiang M-C, et
al. Evaluation of 14 nonlinear deformation algorithms applied to human
brain MRI registration. Neuroimage 2009; 46: 786--802.

\hypertarget{ref-klein_mindboggling_2017}{}
Klein A, Ghosh SS, Bao FS, Giard J, Häme Y, Stavsky E, et al.
Mindboggling morphometry of human brains. PLOS Computational Biology
2017; 13: e1005350.

\hypertarget{ref-klein_101_2012}{}
Klein A, Tourville J. 101 labeled brain images and a consistent human
cortical labeling protocol. Frontiers in Neuroscience 2012; 6: 171.

\hypertarget{ref-kramer_distinctive_2003}{}
Kramer JH, Jurik J, Sha SJ, Rankin KP, Rosen HJ, Johnson JK, et al.
Distinctive neuropsychological patterns in frontotemporal dementia,
semantic dementia, and Alzheimer disease. Cognitive and Behavioral
Neurology: Official Journal of the Society for Behavioral and Cognitive
Neurology 2003; 16: 211--218.

\hypertarget{ref-lee_clinicopathological_2011}{}
Lee SE, Rabinovici GD, Mayo MC, Wilson SM, Seeley WW, DeArmond SJ, et
al. Clinicopathological correlations in corticobasal degeneration.
Annals of neurology 2011; 70: 327--340.

\hypertarget{ref-lehmann_global_2012}{}
Lehmann M, Barnes J, Ridgway GR, Ryan NS, Warrington EK, Crutch SJ, et
al. Global gray matter changes in posterior cortical atrophy: A serial
imaging study. Alzheimer's \& Dementia 2012; 8: 502--512.

\hypertarget{ref-lehmann_loss_2015}{}
Lehmann M, Madison C, Ghosh PM, Miller ZA, Greicius MD, Kramer JH, et
al. Loss of functional connectivity is greater outside the default mode
network in nonfamilial early-onset Alzheimer's disease variants.
Neurobiology of Aging 2015; 36: 2678--2686.

\hypertarget{ref-libon_verbal_2011}{}
Libon DJ, Bondi MW, Price CC, Lamar M, Eppig J, Wambach DM, et al.
Verbal Serial List Learning in Mild Cognitive Impairment: A Profile
Analysis of Interference, Forgetting, and Errors. Journal of the
International Neuropsychological Society 2011a; 17: 905--914.

\hypertarget{ref-libon_philadelphia_2011}{}
Libon DJ, Rascovsky K, Gross RG, White MT, Xie SX, Dreyfuss M, et al.
The Philadelphia Brief Assessment of Cognition (PBAC): A Validated
Screening Measure for Dementia. The Clinical Neuropsychologist 2011b;
25: 1314--1330.

\hypertarget{ref-liu_trans-synaptic_2012}{}
Liu L, Drouet V, Wu JW, Witter MP, Small SA, Clelland C, et al.
Trans-synaptic spread of tau pathology in vivo. PloS One 2012; 7:
e31302.

\hypertarget{ref-marcus_open_2007}{}
Marcus DS, Wang TH, Parker J, Csernansky JG, Morris JC, Buckner RL. Open
Access Series of Imaging Studies (OASIS): Cross-sectional MRI data in
young, middle aged, nondemented, and demented older adults. Journal of
Cognitive Neuroscience 2007; 19: 1498--1507.

\hypertarget{ref-mattsson_selective_2016}{}
Mattsson N, Schott JM, Hardy J, Turner MR, Zetterberg H. Selective
vulnerability in neurodegeneration: Insights from clinical variants of
Alzheimer's disease. Journal of Neurology, Neurosurgery, and Psychiatry
2016; 87: 1000--1004.

\hypertarget{ref-mckhann_diagnosis_2011}{}
McKhann GM, Knopman DS, Chertkow H, Hyman BT, Jack Jr. CR, Kawas CH, et
al. The diagnosis of dementia due to Alzheimer's disease:
Recommendations from the National Institute on Aging-Alzheimer's
Association workgroups on diagnostic guidelines for Alzheimer's disease.
Alzheimer's \& Dementia 2011; 7: 263--269.

\hypertarget{ref-mcmillan_multimodal_2016}{}
McMillan CT, Irwin DJ, Nasrallah I, Phillips JS, Spindler M, Rascovsky
K, et al. Multimodal evaluation demonstrates in vivo 18F-AV-1451 uptake
in autopsy-confirmed corticobasal degeneration. Acta neuropathologica
2016; 132: 935--937.

\hypertarget{ref-medaglia_brain_2017}{}
Medaglia JD, Huang W, Segarra S, Olm C, Gee J, Grossman M, et al. Brain
network efficiency is influenced by the pathologic source of
corticobasal syndrome. Neurology 2017; 89: 1373--1381.

\hypertarget{ref-mesulam_primary_2014}{}
Mesulam M-M, Rogalski EJ, Wieneke C, Hurley RS, Geula C, Bigio EH, et
al. Primary progressive aphasia and the evolving neurology of the
language network. Nature Reviews. Neurology 2014a; 10: 554--569.

\hypertarget{ref-mesulam_asymmetry_2014}{}
Mesulam M-M, Weintraub S, Rogalski EJ, Wieneke C, Geula C, Bigio EH.
Asymmetry and heterogeneity of Alzheimer's and frontotemporal pathology
in primary progressive aphasia. Brain: A Journal of Neurology 2014b;
137: 1176--1192.

\hypertarget{ref-mezias_connectivity_2017}{}
Mezias C, LoCastro E, Xia C, Raj A. Connectivity, not region-intrinsic
properties, predicts regional vulnerability to progressive tau pathology
in mouse models of disease {[}Internet{]}. Acta Neuropathologica
Communications 2017; 5{[}cited 2017 Nov 14{]} Available from:
\url{https://www.ncbi.nlm.nih.gov/pmc/articles/PMC5556602/}

\hypertarget{ref-moller_different_2013}{}
Möller C, Vrenken H, Jiskoot L, Versteeg A, Barkhof F, Scheltens P, et
al. Different patterns of gray matter atrophy in early- and late-onset
Alzheimer's disease. Neurobiology of Aging 2013; 34: 2014--2022.

\hypertarget{ref-murray_neuropathologically_2011}{}
Murray ME, Graff-Radford NR, Ross OA, Petersen RC, Duara R, Dickson DW.
Neuropathologically defined subtypes of Alzheimer's disease with
distinct clinical characteristics: A retrospective study. The Lancet.
Neurology 2011; 10: 785--796.

\hypertarget{ref-nieuwenhuys_human_2008}{}
Nieuwenhuys R, Voogd J, Voogd J, Huijzen C van, Huijzen C van. The Human
Central Nervous System. 4th ed. Berlin: Springer; 2008.

\hypertarget{ref-ossenkoppele_atrophy_2015}{}
Ossenkoppele R, Cohn-Sheehy BI, La Joie R, Vogel JW, Möller C, Lehmann
M, et al. Atrophy Patterns in Early Clinical Stages Across Distinct
Phenotypes of Alzheimer's Disease. Human brain mapping 2015a; 36:
4421--4437.

\hypertarget{ref-ossenkoppele_behaviouralux2fdysexecutive_2015}{}
Ossenkoppele R, Pijnenburg YAL, Perry DC, Cohn-Sheehy BI, Scheltens NME,
Vogel JW, et al. The behavioural/dysexecutive variant of Alzheimer's
disease: Clinical, neuroimaging and pathological features. Brain 2015b;
138: 2732--2749.

\hypertarget{ref-peter_subgroups_2014}{}
Peter J, Abdulkadir A, Kaller C, Kümmerer D, Hüll M, Vach W, et al.
Subgroups of Alzheimer's disease: Stability of empirical clusters over
time. Journal of Alzheimer's disease: JAD 2014; 42: 651--661.

\hypertarget{ref-phan_functional_2002}{}
Phan KL, Wager T, Taylor SF, Liberzon I. Functional neuroanatomy of
emotion: A meta-analysis of emotion activation studies in PET and fMRI.
NeuroImage 2002; 16: 331--348.

\hypertarget{ref-phillips_neocortical_2018}{}
Phillips JS, Da Re F, Dratch L, Xie SX, Irwin DJ, McMillan CT, et al.
Neocortical origin and progression of gray matter atrophy in nonamnestic
Alzheimer's disease. Neurobiology of Aging 2018; 63: 75--87.

\hypertarget{ref-poulakis_heterogeneous_2018}{}
Poulakis K, Pereira JB, Mecocci P, Vellas B, Tsolaki M, Link to external
site this link will open in a new window, et al. Heterogeneous patterns
of brain atrophy in Alzheimer's disease. Neurobiology of Aging 2018; 65:
98--108.

\hypertarget{ref-raj_network_2012}{}
Raj A, Kuceyeski A, Weiner M. A Network Diffusion Model of Disease
Progression in Dementia. Neuron 2012; 73: 1204--1215.

\hypertarget{ref-raj_network_2015}{}
Raj A, LoCastro E, Kuceyeski A, Tosun D, Relkin N, Weiner M. Network
Diffusion Model of Progression Predicts Longitudinal Patterns of Atrophy
and Metabolism in Alzheimer's Disease. Cell Reports 2015; 10: 359--369.

\hypertarget{ref-ramanan_longitudinal_2017}{}
Ramanan S, Bertoux M, Flanagan E, Irish M, Piguet O, Hodges JR, et al.
Longitudinal Executive Function and Episodic Memory Profiles in
Behavioral-Variant Frontotemporal Dementia and Alzheimer's Disease.
Journal of the International Neuropsychological Society: JINS 2017; 23:
34--43.

\hypertarget{ref-rascovsky_sensitivity_2011}{}
Rascovsky K, Hodges JR, Knopman D, Mendez MF, Kramer JH, Neuhaus J, et
al. Sensitivity of revised diagnostic criteria for the behavioural
variant of frontotemporal dementia. Brain 2011; 134: 2456--2477.

\hypertarget{ref-rascovsky_disparate_2007}{}
Rascovsky K, Salmon DP, Hansen LA, Thal LJ, Galasko D. Disparate letter
and semantic category fluency deficits in autopsy-confirmed
frontotemporal dementia and Alzheimer's disease. Neuropsychology 2007;
21: 20--30.

\hypertarget{ref-rogalski_progression_2011}{}
Rogalski E, Cobia D, Harrison TM, Wieneke C, Weintraub S, Mesulam M-M.
Progression of language decline and cortical atrophy in subtypes of
primary progressive aphasia. Neurology 2011; 76: 1804--1810.

\hypertarget{ref-rogalski_aphasic_2016}{}
Rogalski E, Sridhar J, Rader B, Martersteck A, Chen K, Cobia D, et al.
Aphasic variant of Alzheimer disease: Clinical, anatomic, and genetic
features. Neurology 2016; 87: 1337--1343.

\hypertarget{ref-rohrer_patterns_2013}{}
Rohrer JD, Caso F, Mahoney C, Henry M, Rosen HJ, Rabinovici G, et al.
Patterns of longitudinal brain atrophy in the logopenic variant of
primary progressive aphasia. Brain and Language 2013; 127: 121--126.

\hypertarget{ref-sabuncu_dynamics_2011}{}
Sabuncu MR, Desikan RS, Sepulcre J, Yeo BTT, Liu H, Schmansky NJ, et al.
The Dynamics of Cortical and Hippocampal Atrophy in Alzheimer Disease.
Archives of Neurology 2011; 68: 1040--1048.

\hypertarget{ref-schuff_nonlinear_2012}{}
Schuff N, Tosun D, Insel PS, Chiang GC, Truran D, Aisen PS, et al.
Nonlinear time course of brain volume loss in cognitively normal and
impaired elders. Neurobiology of Aging 2012; 33: 845--855.

\hypertarget{ref-schwarz_large-scale_2016}{}
Schwarz CG, Gunter JL, Wiste HJ, Przybelski SA, Weigand SD, Ward CP, et
al. A large-scale comparison of cortical thickness and volume methods
for measuring Alzheimer's disease severity. NeuroImage: Clinical 2016;
11: 802--812.

\hypertarget{ref-seeley_anterior_2010}{}
Seeley WW. Anterior insula degeneration in frontotemporal dementia.
Brain Structure \& Function 2010; 214: 465--475.

\hypertarget{ref-shaw_cerebrospinal_2009}{}
Shaw LM, Vanderstichele H, Knapik-Czajka M, Clark CM, Aisen PS, Petersen
RC, et al. Cerebrospinal fluid biomarker signature in Alzheimer's
disease neuroimaging initiative subjects. Annals of Neurology 2009; 65:
403--413.

\hypertarget{ref-singer_neuronal_2006}{}
Singer T. The neuronal basis and ontogeny of empathy and mind reading:
Review of literature and implications for future research. Neuroscience
and Biobehavioral Reviews 2006; 30: 855--863.

\hypertarget{ref-tang-wai_clinical_2004}{}
Tang-Wai DF, Graff-Radford NR, Boeve BF, Dickson DW, Parisi JE, Crook R,
et al. Clinical, genetic, and neuropathologic characteristics of
posterior cortical atrophy. Neurology 2004; 63: 1168--1174.

\hypertarget{ref-teipel_white_2010}{}
Teipel SJ, Bokde ALW, Meindl T, Amaro E, Soldner J, Reiser MF, et al.
White matter microstructure underlying default mode network connectivity
in the human brain. NeuroImage 2010; 49: 2021--2032.

\hypertarget{ref-toledo_csf_2012}{}
Toledo JB, Brettschneider J, Grossman M, Arnold SE, Hu WT, Xie SX, et
al. CSF biomarkers cutoffs: The importance of coincident
neuropathological diseases. Acta neuropathologica 2012; 124: 23--35.

\hypertarget{ref-tustison_n4itk:_2010}{}
Tustison NJ, Avants BB, Cook PA, Zheng Y, Egan A, Yushkevich PA, et al.
N4ITK: Improved N3 bias correction. IEEE transactions on medical imaging
2010; 29: 1310--1320.

\hypertarget{ref-tustison_explicit_2013}{}
Tustison NJ, Avants BB. Explicit B-spline regularization in
diffeomorphic image registration {[}Internet{]}. Frontiers in
Neuroinformatics 2013; 7{[}cited 2018 Oct 5{]} Available from:
\url{https://www.ncbi.nlm.nih.gov/pmc/articles/PMC3870320/}

\hypertarget{ref-tustison_large-scale_2014}{}
Tustison NJ, Cook PA, Klein A, Song G, Das SR, Duda JT, et al.
Large-scale evaluation of ANTs and FreeSurfer cortical thickness
measurements. NeuroImage 2014; 99: 166--179.

\hypertarget{ref-tzourio-mazoyer_automated_2002}{}
Tzourio-Mazoyer N, Landeau B, Papathanassiou D, Crivello F, Etard O,
Delcroix N, et al. Automated anatomical labeling of activations in SPM
using a macroscopic anatomical parcellation of the MNI MRI
single-subject brain. NeuroImage 2002; 15: 273--289.

\hypertarget{ref-wang_multi-atlas_2013}{}
Wang H, Suh JW, Das SR, Pluta J, Craige C, Yushkevich PA. Multi-Atlas
Segmentation with Joint Label Fusion. IEEE transactions on pattern
analysis and machine intelligence 2013; 35: 611--623.

\hypertarget{ref-warren_paradox_2012}{}
Warren JD, Fletcher PD, Golden HL. The paradox of syndromic diversity in
Alzheimer disease. Nature Reviews Neurology 2012; 8: 451--464.

\hypertarget{ref-whitwell_neuroimaging_2012}{}
Whitwell JL, Dickson DW, Murray ME, Weigand SD, Tosakulwong N, Senjem
ML, et al. Neuroimaging correlates of pathologically defined subtypes of
Alzheimer's disease: A case-control study. The Lancet Neurology 2012;
11: 868--877.

\hypertarget{ref-whitwell_temporoparietal_2011}{}
Whitwell JL, Jack CR, Przybelski SA, Parisi JE, Senjem ML, Boeve BF, et
al. Temporoparietal atrophy: A marker of AD pathology independent of
clinical diagnosis. Neurobiology of Aging 2011; 32: 1531--1541.

\hypertarget{ref-whitwell_working_2015}{}
Whitwell JL, Jones DT, Duffy JR, Strand EA, Machulda MM, Przybelski SA,
et al. Working memory and language network dysfunction in logopenic
aphasia: A task-free fMRI comparison to Alzheimer's dementia.
Neurobiology of aging 2015; 36: 1245--1252.

\hypertarget{ref-whitwell_mri_2008}{}
Whitwell JL, Josephs KA, Murray ME, Kantarci K, Przybelski SA, Weigand
SD, et al. MRI correlates of neurofibrillary tangle pathology at
autopsy. Neurology 2008; 71: 743--749.

\hypertarget{ref-xia_association_2017}{}
Xia C, Makaretz SJ, Caso C, McGinnis S, Gomperts SN, Sepulcre J, et al.
Association of In Vivo {[}18F{]}AV-1451 Tau PET Imaging Results With
Cortical Atrophy and Symptoms in Typical and Atypical Alzheimer Disease
{[}Internet{]}. JAMA Neurology 2017{[}cited 2017 Mar 15{]} Available
from:
\url{http://jamanetwork.com/journals/jamaneurology/fullarticle/2604134}

\hypertarget{ref-yeh_population-averaged_2018}{}
Yeh F-C, Panesar S, Fernandes D, Meola A, Yoshino M, Fernandez-Miranda
JC, et al. Population-averaged atlas of the macroscale human structural
connectome and its network topology. NeuroImage 2018; 178: 57--68.


\end{document}
