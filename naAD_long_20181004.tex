\documentclass[]{article}
\usepackage{lmodern}
\usepackage{amssymb,amsmath}
\usepackage{ifxetex,ifluatex}
\usepackage{fixltx2e} % provides \textsubscript
\ifnum 0\ifxetex 1\fi\ifluatex 1\fi=0 % if pdftex
  \usepackage[T1]{fontenc}
  \usepackage[utf8]{inputenc}
\else % if luatex or xelatex
  \ifxetex
    \usepackage{mathspec}
  \else
    \usepackage{fontspec}
  \fi
  \defaultfontfeatures{Ligatures=TeX,Scale=MatchLowercase}
\fi
% use upquote if available, for straight quotes in verbatim environments
\IfFileExists{upquote.sty}{\usepackage{upquote}}{}
% use microtype if available
\IfFileExists{microtype.sty}{%
\usepackage{microtype}
\UseMicrotypeSet[protrusion]{basicmath} % disable protrusion for tt fonts
}{}
\usepackage[margin=1in]{geometry}
\usepackage{hyperref}
\hypersetup{unicode=true,
            pdftitle={Longitudinal progression of grey matter atrophy in non-amnestic Alzheimer's disease},
            pdfauthor={*Fulvio Da Re, MD1,2,3, *Jeffrey S. Phillips, PhD1,4, David J. Irwin, MD1,4, Corey T. McMillan, PhD1,4, Sanjeev N. Vaishnavi, MD, PhD4, Sharon X. Xie, PhD5, Edward B. Lee, MD, PhD6, Philip A. Cook, PhD7, James C. Gee, PhD7, Leslie M. Shaw, PhD6, John Q. Trojanowski, MD, PhD6, David A. Wolk, MD4,8 and Murray Grossman, MD, EdD1,4},
            pdfborder={0 0 0},
            breaklinks=true}
\urlstyle{same}  % don't use monospace font for urls
\usepackage{graphicx,grffile}
\makeatletter
\def\maxwidth{\ifdim\Gin@nat@width>\linewidth\linewidth\else\Gin@nat@width\fi}
\def\maxheight{\ifdim\Gin@nat@height>\textheight\textheight\else\Gin@nat@height\fi}
\makeatother
% Scale images if necessary, so that they will not overflow the page
% margins by default, and it is still possible to overwrite the defaults
% using explicit options in \includegraphics[width, height, ...]{}
\setkeys{Gin}{width=\maxwidth,height=\maxheight,keepaspectratio}
\IfFileExists{parskip.sty}{%
\usepackage{parskip}
}{% else
\setlength{\parindent}{0pt}
\setlength{\parskip}{6pt plus 2pt minus 1pt}
}
\setlength{\emergencystretch}{3em}  % prevent overfull lines
\providecommand{\tightlist}{%
  \setlength{\itemsep}{0pt}\setlength{\parskip}{0pt}}
\setcounter{secnumdepth}{5}
% Redefines (sub)paragraphs to behave more like sections
\ifx\paragraph\undefined\else
\let\oldparagraph\paragraph
\renewcommand{\paragraph}[1]{\oldparagraph{#1}\mbox{}}
\fi
\ifx\subparagraph\undefined\else
\let\oldsubparagraph\subparagraph
\renewcommand{\subparagraph}[1]{\oldsubparagraph{#1}\mbox{}}
\fi

%%% Use protect on footnotes to avoid problems with footnotes in titles
\let\rmarkdownfootnote\footnote%
\def\footnote{\protect\rmarkdownfootnote}

%%% Change title format to be more compact
\usepackage{titling}

% Create subtitle command for use in maketitle
\newcommand{\subtitle}[1]{
  \posttitle{
    \begin{center}\large#1\end{center}
    }
}

\setlength{\droptitle}{-2em}

  \title{Longitudinal progression of grey matter atrophy in non-amnestic
Alzheimer's disease}
    \pretitle{\vspace{\droptitle}\centering\huge}
  \posttitle{\par}
    \author{*Fulvio Da Re, MD\textsuperscript{1,2,3}, *Jeffrey S. Phillips,
PhD\textsuperscript{1,4}, David J. Irwin, MD\textsuperscript{1,4}, Corey
T. McMillan, PhD\textsuperscript{1,4}, Sanjeev N. Vaishnavi, MD,
PhD\textsuperscript{4}, Sharon X. Xie, PhD\textsuperscript{5}, Edward B.
Lee, MD, PhD\textsuperscript{6}, Philip A. Cook, PhD\textsuperscript{7},
James C. Gee, PhD\textsuperscript{7}, Leslie M. Shaw,
PhD\textsuperscript{6}, John Q. Trojanowski, MD, PhD\textsuperscript{6},
David A. Wolk, MD\textsuperscript{4,8} and Murray Grossman, MD,
EdD\textsuperscript{1,4}}
    \preauthor{\centering\large\emph}
  \postauthor{\par}
      \predate{\centering\large\emph}
  \postdate{\par}
    \date{October 05, 2018}

\usepackage{graphicx} \usepackage{morefloats} \usepackage{caption}
\usepackage{lscape} \usepackage{siunitx} \usepackage{xcolor, colortbl}
\newcommand{\blandscape}{\begin{landscape}}
\newcommand{\elandscape}{\end{landscape}}

\begin{document}
\maketitle

\pagenumbering{gobble}

* These authors contributed equally to this work.

\textbf{Short title:} Longitudinal Atrophy in Non-Amnestic AD

\textbf{Affiliations:}

\textsuperscript{1}Penn Frontotemporal Degeneration Center, University
of Pennsylvania, Philadelphia, PA, USA, \textsuperscript{2}PhD Program
in Neuroscience, University of Milano-Bicocca, Milan, Italy,
\textsuperscript{3}School of Medicine and Surgery, Milan Center for
Neuroscience (NeuroMI), \textsuperscript{4}Department of Neurology,
Perelman School of Medicine, University of Pennsylvania, Philadelphia,
PA, USA, University of Milano-Bicocca, Milan, Italy,
\textsuperscript{5}Department of Biostatistics, Epidemiology, and
Informatics, Perelman School of Medicine, University of Pennsylvania,
Philadelphia, PA, USA, \textsuperscript{6}Center for Neurodegenerative
Disease Research, University of Pennsylvania, Philadelphia, PA, USA,
\textsuperscript{7}Penn Image Computing and Science Laboratory,
Department of Radiology, University of Pennsylvania, Philadelphia, PA,
USA, \textsuperscript{8}Penn Memory Center, University of Pennsylvania,
Philadelphia, PA, USA

\textbf{Corresponding author:}

Jeffrey S. Phillips, PhD\\
\href{mailto:jefphi@pennmedicine.upenn.edu}{\nolinkurl{jefphi@pennmedicine.upenn.edu}}\\
Penn Frontotemporal Degeneration Center\\
Department of Neurology\\
University of Pennsylvania\\
3400 Spruce St., 3 Gibson\\
Philadelphia, PA 19104

\newpage

\pagenumbering{arabic}

\section*{Abstract}\label{abstract}
\addcontentsline{toc}{section}{Abstract}

Non-amnestic Alzheimer's disease is associated with domain-specific
cognitive deficits and spared episodic memory. However, longitudinal
anatomical change in this condition remains understudied. We
investigated whether non-amnestic Alzheimer's disease patients exhibit
focal neocortical atrophy that spreads longitudinally depending on
patients' phenotype. Additionally, we asked whether atrophy spreads more
rapidly throughout neocortical regions and more slowly in the medial
temporal lobes in non-amnestic than amnestic Alzheimer's disease. We
analyzed longitudinal atrophy in 50 non-amnestic patients---including
logopenic-variant primary progressive aphasia (n=25), posterior cortical
atrophy (n=15), and frontal-variant Alzheimer's disease (n=10)---and 15
amnestic patients, diagnosed by published clinical research criteria and
matched for age, disease severity, and disease duration at initial MRI.
Patients were compared to 37 demographically-matched controls. All
patients had autopsy (n=8) or cerebrospinal fluid (n=57) evidence of
Alzheimer's disease pathology. We compared grey matter volume at initial
MRI and longitudinal change in regions associated with each phenotype,
adjusting for age and severity of cognitive impairment. We observed
unique patterns of longitudinal neocortical disease spread in each
non-amnestic phenotype. Non-amnestic patients had more gradual atrophy
than amnestic patients in right temporoparietal areas. Additionally,
logopenic-variant patients exhibited both greater initial atrophy and
more rapid longitudinal change in left lateral temporal areas than other
groups. Frontal-variant patients had more rapid frontal and slower
temporoparietal atrophy than other non-amnestic groups. At the time of
initial scan, logopenic-variant and frontal-variant patients had less
atrophy than amnestic patients in the medial temporal lobes, but
longitudinal atrophy rates in this area did not differ between
non-amnestic and amnestic patients. Older age was associated with medial
temporal atrophy independent of group. Volume change in
phenotype-specific regions of interest was associated with longitudinal
performance on measures of episodic memory, language, visuospatial
function, and executive function. The distribution and rate of
neocortical atrophy differed by phenotype in non-amnestic Alzheimer's
disease; these anatomic patterns differ from amnestic patients and may
explain phenotype-specific cognitive change. Medial temporal sparing in
non-amnestic Alzheimer's disease may be due to later onset of medial
temporal degeneration than in the amnestic syndrome rather than
different rates of atrophy over time. Despite this relative early
sparing, non-amnestic patients were as susceptible as amnestic patients
to normal, age-related atrophy of the medial temporal lobes.

\section*{Keywords}\label{keywords}
\addcontentsline{toc}{section}{Keywords}

Non-amnestic Alzheimer's disease, logopenic-variant primary progressive
aphasia, posterior cortical atrophy, frontal-variant Alzheimer's
disease, longitudinal progression

\section*{Abbreviations}\label{abbreviations}
\addcontentsline{toc}{section}{Abbreviations}

A\(\beta\) = \(\beta\)-amyloid; aAD= amnestic Alzheimer's disease; naAD
= non-amnestic Alzheimer's disease; CBS = corticobasal syndrome; fvAD =
frontal-variant Alzheimer's disease; lvPPA = logopenic-variant primary
progressive aphasia; PCA = posterior cortical atrophy; GM = grey matter;
MTL = medial temporal lobes; ROI = region of interest; MMSE = Mini
Mental Status Exam; LME = linear mixed effects; PVLT = Philadelphia
Verbal Learning Test; PBAC = Philadelphia Brief Assessment of Cognition

\newpage

\section*{Introduction}\label{introduction}
\addcontentsline{toc}{section}{Introduction}

Recent observations suggest that non-amnestic syndromes with underlying
Alzheimer's disease pathology are more prevalent than previously thought
(Peter \emph{et al.}, 2014; Dickerson \emph{et al.}, 2017). These
non-amnestic syndromes are associated with a combination of prominent
neocortical disease and relative hippocampal sparing (Galton \emph{et
al.}, 2000; Murray \emph{et al.}, 2011; Whitwell \emph{et al.}, 2012;
Mesulam \emph{et al.}, 2014). Moreover, patients with atypical disease
distributions may have different rates of clinical progression than
typical amnestic Alzheimer's disease (aAD) patients (Duara \emph{et
al.}, 2013; Byun \emph{et al.}, 2015; Poulakis \emph{et al.}, 2018).
However, relatively little research has examined the longitudinal
anatomical spread of disease in non-amnestic Alzheimer's disease (naAD).
In a previous MRI analysis (Phillips \emph{et al.}, 2018), we inferred
patterns of atrophy spread in four naAD phenotypes: logopenic-variant
primary progressive aphasia (lvPPA), characterized by primary language
deficits (Gorno-Tempini \emph{et al.}, 2011); posterior cortical atrophy
(PCA), characterized by visuospatial deficits (Crutch \emph{et al.},
2017); frontal-variant Alzheimer's disease (fvAD), defined by deficits
in executive function and/or social behavior (Dubois \emph{et al.},
2014); and corticobasal syndrome (CBS), which can present with a
constellation of lateralized motor and cognitive deficits (Medaglia
\emph{et al.}, 2017). These results corroborated the hypothesis that
each naAD phenotype has a distinct neocortical origin with relative
sparing of the medial temporal lobes (MTL). Moreover, this study
suggested that each naAD phenotype has a distinct pattern of disease
spread that differs from aAD. However, this study used a cross-sectional
design and did not directly compare anatomical disease progression
between aAD and naAD phenotypes.

In the present study, we used longitudinal MRI to investigate whether
aAD and naAD differ in longitudinal rate and anatomic distribution of
grey matter (GM) atrophy over time. We reasoned that differences in the
longitudinal rate and distribution of degeneration could explain
differences in clinical progression across amnestic and non-amnestic
Alzheimer's disease variants. In a hypothesis-driven analysis based on
anatomic areas implicated in our prior work (Phillips \emph{et al.},
2018), we investigated differences in GM volume at the time of initial
MRI as well as volume change over time in regions-of-interest (ROIs)
associated with lvPPA, PCA, fvAD, and aAD. Additionally, we performed an
exploratory voxelwise analysis of cortical thickness to identify group
differences not captured by these ROIs. We sought to identify group
differences in atrophy distribution and progression independent of age,
which has been previously reported to differ between typical and
atypical forms of Alzheimer's disease (Murray \emph{et al.}, 2011).
Based on the high neocortical disease burden and domain-specific
cognitive deficits that we previously observed in naAD, we predicted
that naAD patients would exhibit faster rates of atrophy in
phenotype-specific neocortical ROIs relative to aAD. Additionally, we
tested the hypothesis that naAD patients would exhibit slower atrophy
than aAD patients in the hippocampus and surrounding MTL areas, as a
possible explanation for the relative memory sparing associated with
these structures in naAD.

\section*{Materials and methods}\label{materials-and-methods}
\addcontentsline{toc}{section}{Materials and methods}

\subsection*{Patients}\label{patients}
\addcontentsline{toc}{subsection}{Patients}

\begin{table}[ht]
\centering
\caption{Participant characteristics at time of first scan. Median values and interquartile ranges (square brackets) are given for all continuous variables. Education, disease duration, and age at MRI are expressed in years. For each cognitive score, numbers in parentheses indicate the number of observations per group. P-values reflect the results of a chi-squared test for sex and Kruskal-Wallis tests for all other variables.} 
\scalebox{0.67}{
\begin{tabular}{lrrrrrr}
  \hline
 & Control & aAD & lvPPA & PCA & fvAD & P \\ 
  \hline
N &   37 &   15 &   25 &   15 &   10 &  \\ 
  Male &   16 (43.2\%)  &    6 (40.0\%)  &   10 (40.0\%)  &    5 (33.3\%)  &    7 (70.0\%)  &  0.450 \\ 
  Education & 16.0 [16.0, 18.0] & 16.0 [14.0, 18.0] & 16.0 [14.0, 19.0] & 16.0 [16.0, 17.5] & 17.0 [14.5, 18.0] &  0.995 \\ 
  Age at MRI (years) & 61.9 [57.9, 65.6] & 60.8 [56.7, 71.7] & 60.7 [57.7, 67.9] & 57.7 [54.9, 61.9] & 67.7 [60.2, 74.8] &  0.130 \\ 
  MMSE (0--30) & 29.2 [29.0, 30.0] & 24.0 [20.0, 25.5] & 24.0 [22.0, 27.0] & 24.0 [18.5, 25.5] & 26.0 [23.0, 26.8] & <0.001 \\ 
  Inter-scan interval (years) & 1.2 [0.9, 1.7] & 1.1 [0.9, 1.5] & 1.0 [0.9, 1.2] & 1.0 [0.8, 1.1] & 1.1 [0.9, 1.3] & 0.274 \\ 
  Disease duration (years) & --- &  3.0 [2.1, 4.4] &  2.5 [1.7, 3.7] &  2.5 [1.4, 4.5] &  2.2 [1.5, 4.8] &  0.882 \\ 
  Recognition memory (discrimination, 0--1) & --- &  0.5 [0.4, 0.6] (8) &  0.8 [0.8, 1.0] (25) &  0.7 [0.6, 0.9] (13) &  0.7 [0.6, 0.8] (9) &  0.025 \\ 
  Speech (0--4) & --- &  2.5 [2.2, 2.8] (7) &  2.5 [2.0, 3.0] (19) &  3.5 [3.0, 4.0] (9) &  3.5 [3.2, 4.0] (8) &  0.007 \\ 
  Letter fluency (\# words/60 s) & --- &  7.0 [3.0, 9.0] (7) &  7.0 [5.0, 10.0] (19) & 15.0 [7.0, 17.5] (11) & 11.0 [8.5, 12.5] (7) &  0.038 \\ 
  Forward digit span (length correct) & --- &  5.5 [4.2, 6.2] (4) &  5.0 [4.0, 6.0] (23) &  6.0 [5.0, 6.5] (15) &  5.5 [4.8, 6.5] (8) &  0.531 \\ 
  Rey figure copy (0--12) & --- &  4.0 [2.0, 11.5] (7) & 12.0 [11.0, 12.0] (19) &  1.0 [0.0, 6.5] (8) & 10.5 [8.2, 11.2] (8) &  0.003 \\ 
  Judgment of line orientation (0--6) & --- &  2.5 [0.2, 4.8] (6) &  5.0 [4.0, 6.0] (19) &  1.0 [0.0, 2.5] (7) &  4.0 [3.5, 5.5] (7) &  0.002 \\ 
  Social behavior (0--18) & --- & 17.5 [16.2, 18.0] (6) & 18.0 [17.0, 18.0] (19) & 17.0 [16.0, 18.0] (9) & 12.0 [9.2, 16.2] (8) &  0.005 \\ 
  Oral trail-making test (0--6) & --- &  3.0 [1.5, 3.0] (3) &  2.0 [0.5, 3.0] (11) &  2.0 [1.0, 3.0] (5) &  3.5 [2.5, 4.2] (4) &  0.576 \\ 
  Backward digit span (length correct) & --- &  3.5 [2.2, 4.0] (6) &  3.0 [3.0, 4.0] (23) &  3.0 [2.0, 3.8] (14) &  3.0 [3.0, 4.0] (8) &  0.532 \\ 
   \hline
\end{tabular}
}
\end{table}

Participants were recruited through the Penn Frontotemporal Degeneration
Center (FTDC) and the Penn Memory Center (PMC) at the University of
Pennsylvania, including 37 elderly controls, 15 patients with aAD, and
50 patients with naAD (25 with lvPPA, 15 with PCA, and 10 with fvAD).
All procedures were approved by the University of Pennsylvania's
Institutional Review Board, and all patients and/or their caregivers
gave written informed consent according to the principles established by
the Declaration of Helsinki. All patients were clinically diagnosed by
experienced neurologists (MG, DJI, DW, and SV), and diagnoses were
confirmed by consensus after patients' initial visit by clinicians with
expertise in dementia. All patients were at least 50 years of age and
had no major cerebrovascular disease, stroke, head trauma, or comorbid
psychiatric, neurodegenerative, medical, or developmental disorders.
Clinical criteria for each phenotype were as follows: for aAD, primary
memory impairment plus deficits in one or more additional cognitive
domains (McKhann \emph{et al.}, 2011); for lvPPA, primary language
impairment including deficits in repetition and/or naming (Gorno-Tempini
\emph{et al.}, 2011; Giannini \emph{et al.}, 2017); for PCA,
visuospatial deficits (e.g., in object/spatial perception, neglect, or
oculomotor apraxia) (Crutch \emph{et al.}, 2017); and for fvAD, deficits
in executive function or social behavior. NaAD patients had relatively
preserved abilities in all cognitive domains except their domain of
primary impairment at initial presentation. All patients had evidence of
Alzheimer's disease pathology based on autopsy results (n=9) or a CSF
total tau/beta-amyloid 1--42 ratio greater than 0.34 (n=57) according to
methods previously described (Shaw \emph{et al.}, 2009; Irwin \emph{et
al.}, 2012). APOE genotyping was performed on 62 of 66 patients. One
patient (white male, aAD, age 51 at onset) with an APOE
\(\epsilon3\)/\(\epsilon4\) genotype was found to have a mutation in the
PSEN1 gene.

Shapiro-Wilks tests indicated non-normal distributions for education and
disease duration, age, and MMSE score at initial MRI (all
p\textless{}0.05). Kruskal-Wallis tests of group differences were
non-significant, with the exception of MMSE {[}\(\chi^2\)(4)=56.1,
p\textless{}0.001{]}, reflecting patients' cognitive deficits relative
to controls. Mann-Whitney tests confirmed that all patient groups
exhibited significantly lower MMSE scores than controls (all
U\textgreater{}=4.0, p\textless{}0.001); all other pairwise comparisons
were non-significant. To corroborate naAD patients' domain-specific
cognitive impairment, we analyzed neuropsychological performance on
assessments independent of those used in clinical diagnosis, including
performance on specific items of the Philadelphia Brief Assessment of
Cognition (PBAC) (Libon \emph{et al.}, 2011b). Language was assessed in
terms of speech features (with lower scores indicating speech and
language impairment), forward digit span as a measure of repetition
(Giannini \emph{et al.}, 2017), and letter fluency, which is sensitive
to deficits in executive-mediated lexical retrieval (Rascovsky \emph{et
al.}, 2007; Ramanan \emph{et al.}, 2017). Visuospatial function was
assessed by patients' ability to copy a modified version of the Rey
complex figure as well as the judgment of line orientation. Social
behavior was assessed on an 18-point scale evaluating social
comportment, apathy, disinhibition, agitation, empathy, and ritualistic
behaviors. Executive function was evaluated through an oral version of
the trail-making test as well as backward digit span. Finally, episodic
memory was assessed by recognition on the Philadelphia Verbal Learning
Test (PVLT) (Libon \emph{et al.}, 2011a) or the PBAC verbal memory test,
as available. All neuropsychological assessments were acquired within 1
year of the initial MRI scan (PVLT and letter fluency: mean=0.18 years,
SD=0.25; PBAC: mean=0.27 years, SD=0.29; digit span: mean=0.16 years,
SD=0.32). Results were consistent with each phenotype's primary
impairment in all domains except for executive function (Table 1).
Post-hoc comparisons between all patient groups are reported in
Supplementary Table 1.

Participants were required to have longitudinal T1-weighted structural
MRI data. We retrospectively selected 163 T1-weighted anatomical MRI
scans from patients and 85 scans from controls; all scans were collected
on the same 3.0-Tesla Siemens TIM Trio scanner at the Hospital of the
University of Pennsylvania. A majority of participants (39/66 patients
and 29/37 controls) had only 2 available scans; the remaining
participants contributed 3--4 scans each. We included scans acquired
with a minimum inter-scan interval of 6 months up to 3.5 years from the
initial MRI; beyond this window, there were insufficient observations
for a valid analysis. Due to the limited number of observations, only
linear associations between atrophy and time were assessed. MRI scans
were screened for signs of cerebrovascular disease, hydrocephalus, or
white matter lesions; those with a Fazekas scale score\textgreater{}1
were excluded. The current study included 54 patients from our previous,
cross-sectional study (Phillips \emph{et al.}, 2018) (aAD, n=9; lvPPA,
n=24; PCA, n=15; and fvAD, n=6).

\subsection*{Neuroimaging methods}\label{neuroimaging-methods}
\addcontentsline{toc}{subsection}{Neuroimaging methods}

T1-weighted MR images were acquired axially with 0.98 mm x 0.98 mm x 1
mm voxels, a 256 x 192 matrix, a repetition time of 1620 ms, an
inversion time of 950 ms, and a flip angle of 15\(^\circ\). Scans were
visually inspected for quality by two authors (JP and FDR). Advanced
Normalization Tools (ANTs) (Avants \emph{et al.}, 2014; Tustison
\emph{et al.}, 2014) was used to process each image using a prior-based
approach. Images underwent intensity normalization (Tustison \emph{et
al.}, 2010) and were spatially normalized to a template based on healthy
controls from the Open Access Series of Imaging Studies (OASIS) dataset
(Marcus \emph{et al.}, 2007) using a symmetric diffeomorphic algorithm
(Klein \emph{et al.}, 2009; Avants \emph{et al.}, 2011). Images were
segmented into 6 tissue classes (cortical grey matter, subcortical grey
matter, deep white matter, CSF, brainstem, and cerebellum) using
template-based priors; this tissue segmentation was then used to
estimate cortical thickness (Tustison \emph{et al.}, 2014). We used a
joint label fusion approach (Wang \emph{et al.}, 2013) to align the
Mindboggle-101 labels (based on the Desikan-Killainy-Tourville label
scheme) (Klein and Tourville, 2012) with each image using
pseudo-geodesic registration (Tustison and Avants, 2013) and calculated
the volume of GM voxels within each label, normalized by intracranial
volume and converted to a z-score relative to controls' initial scans.
To perform voxelwise group analyses, we warped cortical thickness images
to the template using the previously-computed spatial transforms; these
images were then spatially smoothed with a 2-sigma Gaussian kernel and
downsampled to 2 mm isotropic voxels.

\subsection*{Statistical analysis}\label{statistical-analysis}
\addcontentsline{toc}{subsection}{Statistical analysis}

In a hypothesis-driven analysis, we analyzed GM volumes in
phenotype-specific ROIs from our previous study of disease progression
in naAD (Phillips \emph{et al.}, 2018). This study identified the
regions most commonly atrophied in each naAD phenotype, reflecting the
likely anatomical origin of disease. These ROIs included left middle and
superior temporal gyri in lvPPA; right precuneus, superior parietal
lobule, and angular, supramarginal, and middle temporal gyri in PCA; and
left anterior insula and middle frontal gyrus as well as right middle
temporal gyrus in fvAD (Table 2). Each ROI was expected to exhibit lower
volume at the time of participants' initial MRI scan as well as more
rapid volume loss over time in its associated patient group(s) relative
to other groups. We additionally hypothesized that the aAD group would
demonstrate selective atrophy in the MTL, including bilateral
hippocampi, parahippocampal gyri, and entorhinal cortex. Atrophy at the
time of initial MRI was analyzed using multiple linear regression models
with a factor of group and covariates for age and MMSE score at the time
of initial MRI as well as the interval between MMSE administration and
MRI; controls formed the reference group in these models. Longitudinal
atrophy was assessed using linear mixed effects (LME) models with fixed
factors of group, time since first scan, and the interaction of group x
time. As in the baseline model, covariates included age and MMSE score
at initial MRI as well as the interval between MMSE and initial MRI. A
subject-specific random intercept was included to account for
intra-individual correlations in imaging measures. Post-hoc comparisons
were performed for the effect of group at initial MRI as well as the
group x time interaction in longitudinal models; values of
p\textless{}0.05, FDR-corrected, were considered significant.

We used LME models to relate GM volume change to neuropsychological
performance within 1 year of each imaging session. The mean interval
between test and MRI was 0.19 years (SD=0.24) for recognition memory;
0.22 years (SD=0.24) for letter fluency; 0.18 years (SD=0.23) for digit
span; and 0.19 years (SD=0.24) for all other longitudinal
neuropsychological measures. Separate LME models were computed for each
measure and change in associated ROIs. Thus, recognition performance was
related to GM volume in each of the 6 MTL ROIs; language measures were
compared to volume change in left middle and superior temporal gyrus;
visuospatial measures were related to change in the right superior
parietal lobule, precuneus, and angular, supramarginal, and middle
temporal gyri; and behavioral and executive measures were related to
left anterior insula and middle frontal gyrus as well as right middle
temporal gyrus. Neuropsychological performance formed the outcome in
each model; predictors treated as fixed effects included regional GM
volume at initial MRI and subsequent volume change. Additionally, a
subject-specific random intercept was included in the LME model. Due to
limited neuropsychological data, controls were omitted from these
models, and lvPPA patients formed the reference group. The association
with regional volume change in each model was assessed at a significance
level of p\textless{}0.05, FDR-corrected. To assess the regional
specificity of cognitive associations, we also related each
neuropsychological measure to longitudinal change in whole cortical
volume.

Additionally, we performed exploratory voxelwise analysis to investigate
differences in cortical thickness that were not assessed by a priori
ROIs. ROI-based and voxelwise analyses both present distinct advantages
and weaknesses. Voxelwise analysis is not constrained by the borders of
anatomically-defined ROIs, and it allows more precise anatomical
localization of effects. However, voxelwise parametric tests depend on
patients' displaying neurodegeneration at the same precise point within
a brain area. Thus, ROI-based volumetric analysis may be more sensitive
to atrophy if the precise focus of atrophy within a region differs
across individuals. Voxelwise analysis did not include hippocampus,
where cortical thickness is not well estimated, but did include
entorhinal cortex and parahippocampal gyri. As in ROI-based analysis, we
used multiple regression to assess group differences at initial MRI and
an LME model to investigate longitudinal atrophy. These voxelwise models
used the same regression formulae as ROI-based models, and the LME was
implemented in the 3dLME (Chen \emph{et al.}, 2013) function from the
Analysis of Functional NeuroImaging (AFNI) software suite. Multiple
comparisons correction was performed by first thresholding voxelwise
results at p\textless{}0.001 (uncorrected), then applying a cluster
extent threshold corresponding to a cluster-wise alpha value of 0.05. To
calculate cluster extent thresholds, we first estimated spatial
auto-correlation from the model residuals using AFNI's 3dFWHMx. We then
used the 3dClustSim function, which is based on a Monte Carlo approach
(Forman \emph{et al.}, 1995; Cox \emph{et al.}, 2017), to determine the
cluster size corresponding to a false-positive rate of 0.05 at
p\textless{}0.001 (uncorrected). These simulations indicated a cluster
threshold of 71 voxels (i.e., \SI{568}{\micro\litre}) for the baseline
MRI model and a threshold of 70 voxels (\SI{560}{\micro\litre}) for the
longitudinal LME model. For both the baseline effect of group and the
group x time interaction, we performed post-hoc contrasts between all
groups, which were corrected to cluster-wise p\textless{}0.05 using the
same method. In the Supplementary Material, we additionally display
voxelwise contrasts vs.~controls at a lenient threshold of
p\textless{}0.01, uncorrected for multiple comparisons.

\subsection*{Data and availability}\label{data-and-availability}
\addcontentsline{toc}{subsection}{Data and availability}

Computer code for the current manuscript (including all text, analysis,
and visualization of results) is available in the form of Rmarkdown and
LaTeX scripts in a public GitHub repository
(\url{https://github.com/jeffrey-phillips/naAD-longitudinal.git}).
Rmarkdown code requires \href{https://cran.r-project.org/}{R} version
3.4.4 or higher. Investigators who wish to access imaging and clinical
data may submit a direct request to the corresponding author.

\section*{Results}\label{results}
\addcontentsline{toc}{section}{Results}

\begin{table}[ht]
\centering
\caption{Differences in grey matter volume at initial MRI and longitudinal atrophy in hypothesis-driven analysis of regional brain volumes, relative to matched controls. Hypotheses included selective atrophy of neocortical areas associated with early disease in naAD\textsuperscript{10} and of the MTL (hippocampus, entorhinal cortex, and parahippocampal gyrus) in aAD patients. F-statistics indicate the main effect of group at initial MRI scan and the group x time interaction across all scans. Additional columns report z-statistics for pairwise contrasts of each patient group vs. controls. Blue cells indicate significant differences in volume only at initial MRI; red cells indicate significant differences in longitudinal atrophy rates; and green cells indicate differences in both initial volume and longitudinal atrophy, based on a threshold of p<0.05, FDR-corrected. *p<0.05; **p<0.01; ***p<0.001.} 
\scalebox{0.67}{
\begin{tabular}{rlrrrrrrrrrr}
  \hline
A priori association & Region & $F_{First~MRI}$(4,94) & aAD & lvPPA & PCA & fvAD & $F_{Group~x~Time}$(4,141) & aAD & lvPPA & PCA & fvAD \\ 
  \hline
aAD & L entorhinal & 9.1*** & \cellcolor{blue}\textcolor{white}{-3.7} & -1.7 & -1.5 & -0.5 & 6.4*** & -1.7 & \cellcolor{red}\textcolor{white}{-4.2} & -2.0 & \cellcolor{red}\textcolor{white}{-3.6} \\ 
   & R entorhinal & 6.1*** & \cellcolor{green}\textcolor{black}{-2.9} & 0.6 & -1.7 & -0.4 & 6.8*** & \cellcolor{green}\textcolor{black}{-3.9} & \cellcolor{red}\textcolor{white}{-3.6} & \cellcolor{red}\textcolor{white}{-3.6} & \cellcolor{red}\textcolor{white}{-2.6} \\ 
   & L hippocampus & 9.1*** & \cellcolor{green}\textcolor{black}{-4.6} & \cellcolor{green}\textcolor{black}{-3.2} & -2.3 & -2.2 & 5.1** & \cellcolor{green}\textcolor{black}{-4.0} & \cellcolor{green}\textcolor{black}{-3.0} & \cellcolor{red}\textcolor{white}{-2.5} & -0.5 \\ 
   & R hippocampus & 9.7*** & \cellcolor{green}\textcolor{black}{-4.5} & -1.4 & \cellcolor{green}\textcolor{black}{-3.1} & -2.3 & 4.5** & \cellcolor{green}\textcolor{black}{-3.3} & \cellcolor{red}\textcolor{white}{-3.0} & \cellcolor{green}\textcolor{black}{-3.0} & -1.1 \\ 
   & L parahippocampal & 3.8*** & \cellcolor{green}\textcolor{black}{-2.5} & \cellcolor{green}\textcolor{black}{-2.7} & -0.9 & 0.1 & 7.4*** & \cellcolor{green}\textcolor{black}{-3.0} & \cellcolor{green}\textcolor{black}{-5.2} & -1.6 & -2.2 \\ 
   & R parahippocampal & 3.5** & \cellcolor{green}\textcolor{black}{-3.0} & -0.8 & -1.5 & -1.3 & 5.8*** & \cellcolor{green}\textcolor{black}{-3.3} & \cellcolor{red}\textcolor{white}{-3.5} & \cellcolor{red}\textcolor{white}{-3.7} & -0.6 \\ 
  lvPPA & L middle temporal & 28.6*** & -2.4 & \cellcolor{green}\textcolor{black}{-6.8} & \cellcolor{green}\textcolor{black}{-2.9} & \cellcolor{green}\textcolor{black}{-3.2} & 33.2*** & \cellcolor{red}\textcolor{white}{-8.9} & \cellcolor{green}\textcolor{black}{-9.4} & \cellcolor{green}\textcolor{black}{-5.9} & \cellcolor{green}\textcolor{black}{-5.0} \\ 
   & L superior temporal & 21.2*** & -1.8 & \cellcolor{green}\textcolor{black}{-6.1} & -1.5 & -2.2 & 20.4*** & \cellcolor{red}\textcolor{white}{-4.4} & \cellcolor{green}\textcolor{black}{-8.7} & \cellcolor{red}\textcolor{white}{-4.7} & \cellcolor{red}\textcolor{white}{-3.1} \\ 
  PCA & R angular & 12.7*** & -1.9 & -0.2 & \cellcolor{green}\textcolor{black}{-2.6} & \cellcolor{blue}\textcolor{white}{-2.7} & 3.1* & -1.0 & \cellcolor{red}\textcolor{white}{-3.1} & \cellcolor{green}\textcolor{black}{-2.5} & -0.8 \\ 
   & R precuneus & 9.2*** & -0.6 & -0.2 & \cellcolor{green}\textcolor{black}{-3.9} & -1.6 & 15.1*** & \cellcolor{red}\textcolor{white}{-6.4} & \cellcolor{red}\textcolor{white}{-5.2} & \cellcolor{green}\textcolor{black}{-4.9} & -0.5 \\ 
   & R superior parietal lobule & 12.7*** & 1.0 & -0.1 & \cellcolor{green}\textcolor{black}{-4.4} & -1.9 & 4.8** & \cellcolor{red}\textcolor{white}{-3.8} & \cellcolor{red}\textcolor{white}{-2.6} & \cellcolor{green}\textcolor{black}{-2.4} & 0.1 \\ 
   & R supramarginal & 3.0** & 0.6 & 0.8 & -1.4 & -0.3 & 10.5*** & \cellcolor{red}\textcolor{white}{-6.2} & \cellcolor{red}\textcolor{white}{-3.2} & \cellcolor{red}\textcolor{white}{-3.1} & -0.5 \\ 
  fvAD & L anterior insula & 9.1*** & -2.1 & \cellcolor{blue}\textcolor{white}{-3.3} & -0.4 & \cellcolor{green}\textcolor{black}{-2.8} & 3.5* & -1.8 & -1.8 & -0.6 & \cellcolor{green}\textcolor{black}{-3.4} \\ 
   & L middle frontal & 11.7*** & -1.9 & \cellcolor{green}\textcolor{black}{-3.1} & -1.8 & \cellcolor{blue}\textcolor{white}{-4.3} & 7.7*** & \cellcolor{red}\textcolor{white}{-4.1} & \cellcolor{green}\textcolor{black}{-4.2} & \cellcolor{red}\textcolor{white}{-4.0} & -1.8 \\ 
  PCA, fvAD & R middle temporal & 22.0*** & -1.8 & -1.6 & \cellcolor{green}\textcolor{black}{-4.3} & \cellcolor{green}\textcolor{black}{-3.9} & 42.6*** & \cellcolor{red}\textcolor{white}{-10.9} & \cellcolor{red}\textcolor{white}{-9.5} & \cellcolor{green}\textcolor{black}{-6.8} & \cellcolor{green}\textcolor{black}{-5.9} \\ 
   \hline
\end{tabular}
}
\end{table}

\begin{figure}

{\centering \includegraphics{./fig/base_diff_fig-1} 

}

\caption{Patient group differences at time of initial MRI in normalized volumes for a priori regions of interest. Plotted values are the regression coefficients representing the difference in regional grey matter volume between each patient group and the healthy control reference group, expressed in z-score units. Error bars represent the standard error of each regression coefficient. More negative values on the x-axis indicate greater atrophy. Black brackets indicate significant pairwise group differences (p<0.05 after false discovery rate correction).}\label{fig:base_diff_fig}
\end{figure}

\subsection*{Hypothesis-driven analysis of ROI
volumes}\label{hypothesis-driven-analysis-of-roi-volumes}
\addcontentsline{toc}{subsection}{Hypothesis-driven analysis of ROI
volumes}

We first assessed both regional volume at initial MRI and longitudinal
volume change in each group relative to matched controls. This analysis
indicated regions that displayed initial atrophy with no longitudinal
progression (Table 2, blue cells), both initial atrophy and longitudinal
progression (green cells), and progression in the absence of inital
atrophy (red cells). In cross-sectional analysis of participants'
initial MRI scans, all ROIs exhibited a main effect of group,
independent of age and MMSE. The aAD patients exhibited initial atrophy
relative to controls in all six MTL regions but in none of the other
regions tested (Table 2). LvPPA patients exhibited atrophy relative to
controls in left anterior insula and hippocampus as well as left
parahippocampal, superior temporal, middle temporal, and middle frontal
gyri. PCA patients, in turn, had lower volumes than controls in right
hippocampus, angular gyrus, precuneus, superior parietal lobule, and
middle temporal gyrus. FvAD patients had significant atrophy in left
anterior insula and middle frontal gyrus; right angular gyrus; and
bilateral middle temporal gyri.

In longitudinal models, all ROIs tested also exhibited a significant
group x time interaction (Table 2). Post-hoc contrasts indicated areas
where each patient group had significantly more rapid atrophy than
controls. The aAD group differed from controls in all regions tested
except for left entorhinal cortex, left anterior insula, and right
angular gyrus. LvPPA patients similarly exhibited significant change
over time in all regions except left anterior insula; and PCA patients
differed from controls in all regions except left entorhinal cortex,
parahipppocampal gyrus, and anterior insula. FvAD patients were the only
group to exhibit significant atrophy over time in left anterior insula;
they also had more rapid atrophy than controls in bilateral entorhinal
cortex and middle temporal gyri as well as left superior temporal gyrus.

Post-hoc comparisons between patient groups reflected established
phenotype-specific patterns of atrophy (Figure 1). LvPPA patients had
lower volumes in left middle and superior temporal gyri than aAD
patients (all z\textgreater{}=2.5, p\textless{}0.05); conversely, they
had larger volumes in bilateral entorhinal cortex, right hippocampus,
and right parahippocampal gyrus than aAD patients. PCA patients had
lower volumes than aAD patients in the right precuneus and superior
parietal lobule (z=3.2, p\textless{}0.02 and z=5.1, p\textless{}0.001,
respectively); additionally, they exhibited marginally greater volumes
than aAD patients in left hippocampus (z=2.3, p\textless{}0.08) and
entorhinal cortex (z=2.2, p\textless{}0.09) and marginally lower volumes
in right middle temporal gyrus (z=2.4, p\textless{}0.06). FvAD patients
had more severe atrophy than aAD patients in the right superior parietal
lobule (z=2.6, p\textless{}0.04) and marginally lower volumes in left
middle frontal (z=2.5, p\textless{}0.06) and right middle temporal gyri
(z=2.2, p\textless{}0.09). Conversely, like the lvPPA group, fvAD
patients had greater volumes than the aAD group in left entorhinal
cortex (z=2.7, p\textless{}0.04); they also had marginally greater
volumes than aAD patients in right entorhinal cortex (z=2.1,
p\textless{}0.09) and left parahippocampal gyrus (z=2.3,
p\textless{}0.08). Comparisons between naAD groups further corroborated
the association of each naAD phenotype with focal neocortical disease
patterns. LvPPA patients had more severe initial atrophy than both PCA
and fvAD patients in left superior temporal gyrus; and more severe
atrophy than PCA patients in left middle temporal gyrus and anterior
insula (all z\textgreater{}=2.5, p\textless{}0.05). PCA patients had
lower volumes than lvPPA patients in right angular gyrus, precuneus,
superior parietal lobule, and middle temporal gyrus, reflecting a
phenotypic difference in lateralization of disease (all
z\textgreater{}=2.6, p\textless{}0.05). Finally, fvAD patients had more
severe atrophy than PCA patients in left middle frontal gyrus (z=2.6,
p\textless{}0.05) and relative to lvPPA patients in right middle
temporal gyrus and angular gyrus (z=2.8 and z=2.7, respectively,
p\textless{}0.05).

\begin{figure}

{\centering \includegraphics{./fig/roi_time_diffs-1} 

}

\caption{Patient group differences in the effect of time for a priori regions of interest. The plot displays regression coefficients for annualized change in regional grey matter volume in each group, expressed in z-score units relative to the healthy control reference group. Error bars represent the standard error of each regression coefficient. More negative values on the x-axis indicate more rapid atrophy over time. Black brackets indicate significant pairwise group differences (p<0.05 after false discovery rate correction).}\label{fig:roi_time_diffs}
\end{figure}

Pairwise contrasts of group x time interaction terms (Figure 2) revealed
differences in rates of longitudinal atrophy. We had hypothesized that
each phenotype would exhibit faster atrophy in its assocaited
neocortical ROIs than other patient groups; this hypothesis was upheld
in left superior temporal gyrus, where lvPPA patients had more rapid
atrophy than aAD patients (z=-2.6, p\textless{}0.03). In left anterior
insula, fvAD patients had a faster rate of atrophy than both PCA
(z=-2.7, p\textless{}0.03) and lvPPA (z=-2.5, p\textless{}0.05)
patients. Longitudinal rates of atrophy did not differ between lvPPA and
PCA in any of the regions investigated. Conversely, fvAD patients had
slower atrophy than PCA in right precuneus (z=2.6, p\textless{}0.04).
Contrary to hypotheses, naAD patients exhibited more gradual atrophy
than aAD patients in several areas. PCA and lvPPA groups both had more
gradual atrophy than aAD patients in right middle temporal gyrus (z=2.8,
p\textless{}0.02 and z=3.0, p\textless{}0.02, respectively), and fvAD
patients had slower atrophy relative to aAD in the right superior
parietal lobule (z=2.3, p\textless{}0.05) and right precuneus (z=3.3,
p\textless{}0.005). Additionally, the lvPPA and fvAD groups had slower
atrophy than aAD patients in right supramarginal gyrus (all
z\textgreater{}=3.2, p\textless{}0.01). MTL atrophy rates did not differ
between naAD and aAD, except in left entorhinal cortex, where fvAD
patients had more rapid atrophy than aAD patients (z=-2.4,
p\textless{}0.05).

\subsubsection*{Effects of global cognition and
age}\label{effects-of-global-cognition-and-age}
\addcontentsline{toc}{subsubsection}{Effects of global cognition and
age}

Initial MMSE score (which was included as a measure of global cognitive
impairment) was positively associated with GM volume in all cortical
ROIs outside the MTL, including bilateral middle temporal gyri; left
anterior insula, middle frontal gyrus, and superior temporal gyrus; and
right supramarginal gyrus, angular gyrus, precuneus, and superior
parietal lobule {[}all t(94)\textgreater{}=2.6, p\textless{}0.02{]}. In
the MTL, MMSE score was associated only with left and right entorhinal
cortex volume {[}t(94)=3.4, p\textless{}0.01 and t(94)=2.2,
p\textless{}0.05, respectively{]}. In contrast, age at initial MRI was
inversely associated with volume in all 6 MTL regions investigated,
including left and right hippocampus {[}t(94)=-2.4, p\textless{}0.05 and
t(94)=-2.9, p\textless{}0.02, respectively, after FDR correction{]},
left and right parahippocampal gyri {[}t(94)=-4.1, p\textless{}0.001 and
t(94)=-4.5, p\textless{}0.001{]}, and left and right entorhinal cortex
{[}both t(94)=-3.4, p\textless{}0.001{]}; no other ROIs displayed an
effect of age. To determine whether this age effect differed by group,
we performed secondary analyses on MTL volumes at the time of first scan
using multiple regression models with predictors of group, age, and
their interaction, covarying for MMSE score and the interval between
MMSE and MRI. After FDR correction, no MTL regions showed a significant
group x age interaction {[}all F(4,90)\textless{}2.8,
p\textgreater{}0.1{]}, suggesting that the association of increased age
with MTL atrophy was similar across groups.

\subsubsection*{Effects of APOE
genotype}\label{effects-of-apoe-genotype}
\addcontentsline{toc}{subsubsection}{Effects of APOE genotype}

We additionally assessed the distribution of APOE genotypes among aAD
and naAD patients. Genotyping data were unavailable for one lvPPA, one
fvAD, and two aAD patients. One aAD and two PCA patients each carried
one copy of the \(\epsilon2\) allele, which is associated with lower
risk for Alzheimer's disease (Corder \emph{et al.}, 1994). The
proportions of lvPPA, PCA, fvAD, and aAD patients carrying 1--2 copies
of the APOE \(\epsilon4\) allele were 37.5\%, 38.4\%, 66.7\%, and
84.6\%, respectively. The frequency of individuals carrying zero, one,
or two copies of the \(\epsilon4\) allele significantly differed across
patient groups {[}\(\chi^2\)(6)=38.7, p\textless{}0.001{]}. These
frequencies significantly differed between all phenotype pairs in
post-hoc comparisons (all \(\chi^2\)(2)\textgreater{}=7.1,
p\textless{}0.03). Because APOE genotypes were unavailable for control
participants, we recomputed LMEs for a priori ROIs using lvPPA (the
largest group) as the reference group and number of \(\epsilon4\)
alleles as a covariate. No ROIs exhibited an association with number of
\(\epsilon4\) alleles independent of group.

\subsubsection*{Longitudinal associations between neuropsychological
performance and grey matter
volume}\label{longitudinal-associations-between-neuropsychological-performance-and-grey-matter-volume}
\addcontentsline{toc}{subsubsection}{Longitudinal associations between
neuropsychological performance and grey matter volume}

\begin{table}[ht]
\centering
\caption{Associations between neuropsychological performance and grey matter volume change in task-specific ROIs. P-values are corrected for multiple comparisons using the false discovery rate method; values<0.05 are considered statistically significant and shown in bold.} 
\begin{tabular}{llll}
  \hline
Task & Region & T & P \\ 
  \hline
Recognition memory & \textbf{L entorhinal} & \textbf{t(62)=2.9} & \textbf{0.02} \\ 
   & \textbf{L hippocampus} & \textbf{t(62)=5.1} & \textbf{0.0002} \\ 
   & \textbf{L parahippocampal} & \textbf{t(62)=3.5} & \textbf{0.005} \\ 
   & \textbf{R entorhinal} & \textbf{t(62)=2.8} & \textbf{0.03} \\ 
   & \textbf{R hippocampus} & \textbf{t(62)=4.1} & \textbf{0.0010} \\ 
   & R parahippocampal & t(62)=2.1 & 0.07 \\ 
  Speech & L middle temporal & t(40)=0.0 & 1 \\ 
   & L superior temporal & t(40)=0.5 & 0.7 \\ 
  Letter fluency & \textbf{L middle temporal} & \textbf{t(57)=2.9} & \textbf{0.02} \\ 
   & L superior temporal & t(57)=2.1 & 0.08 \\ 
  Forward digit span & \textbf{L middle temporal} & \textbf{t(61)=4.4} & \textbf{0.0006} \\ 
   & \textbf{L superior temporal} & \textbf{t(61)=4.9} & \textbf{0.0002} \\ 
  Rey copy & \textbf{R angular} & \textbf{t(36)=2.4} & \textbf{0.05} \\ 
   & \textbf{R middle temporal} & \textbf{t(36)=3.0} & \textbf{0.02} \\ 
   & \textbf{R precuneus} & \textbf{t(36)=2.5} & \textbf{0.05} \\ 
   & R superior parietal lobule & t(36)=2.0 & 0.09 \\ 
   & \textbf{R supramarginal} & \textbf{t(36)=2.5} & \textbf{0.05} \\ 
  Judgment of line orientation & R angular & t(33)=1.4 & 0.3 \\ 
   & R middle temporal & t(33)=1.9 & 0.1 \\ 
   & \textbf{R precuneus} & \textbf{t(33)=2.5} & \textbf{0.04} \\ 
   & R superior parietal lobule & t(33)=2.3 & 0.06 \\ 
   & R supramarginal & t(33)=1.9 & 0.1 \\ 
  Social behavior & L anterior insula & t(38)=1.4 & 0.3 \\ 
   & L middle frontal & t(38)=0.8 & 0.6 \\ 
   & R middle temporal & t(38)=0.8 & 0.6 \\ 
  Oral Trails & L anterior insula & t(14)=-0.7 & 0.6 \\ 
   & L middle frontal & t(14)=0.6 & 0.7 \\ 
   & R middle temporal & t(14)=0.8 & 0.6 \\ 
  Reverse digit span & L anterior insula & t(60)=2.1 & 0.08 \\ 
   & \textbf{L middle frontal} & \textbf{t(60)=2.9} & \textbf{0.02} \\ 
   & \textbf{R middle temporal} & \textbf{t(60)=2.7} & \textbf{0.03} \\ 
   \hline
\end{tabular}
\end{table}

Associations between longitudinal neuropsychological performance and
atrophy were evaluated in patients that had two or more assessments,
each within one year of a structural MRI scan. For recognition memory,
this yielded 109 observations from 45 patients, including 21 lvPPA, nine
PCA, seven aAD, and eight fvAD patients. For letter fluency, 99
observations were available from 18 lvPPA, nine PCA, six aAD, and eight
fvAD patients. For forward and reverse digit span, 107 observations were
available from 22 lvPPA, 10 PCA, six aAD, and seven fvAD patients. A
total of 75 PBAC observations, from which all other test measures were
obtained, were available for 17 lvPPA, six PCA, five aAD, and six fvAD
patients. In all cognitive domains except for social behavior,
longitudinal cognition was directly associated with GM volume change in
one or more associated brain areas, independent of volume at the time of
initial MRI (Table 3). In the memory domain, volume loss in bilateral
hippocampi and entorhinal cortex as well as left parahippocampal gyrus
predicted declines in recognition discrimination. In the language
domain, volume loss in left middle temporal gyrus was associated with
decreases in letter fluency; and volume loss in both left middle and
superior temporal gyri predicted declines in forward digit span. In the
visuospatial domain, Rey figure copy performance over time was
associated with volume loss in right angular, middle temporal, and
supramarginal gyrus as well as right precuneus. Judgment of line
orientations was associated with right precuneus volume. No significant
associations were found for social behavior; however, left middle
frontal and right middle temporal gyri predicted reverse digit span, a
measure of working memory and executive function (Kramer \emph{et al.},
2003).

\subsection*{Exploratory analysis}\label{exploratory-analysis}
\addcontentsline{toc}{subsection}{Exploratory analysis}

\begin{figure}[]
% Voxelwise cortical thickness versus controls.

{\centering \includegraphics[width=7in]{./fig/me_time.pdf} 

}

\caption[GM time effects]{Voxelwise differences in cortical thickness relative to matched controls. Image overlays are binarized t-statistic maps for simple contrasts of controls minus each patient group. Blue: simple effect of group (patients<controls) from cross-sectional analysis of participants' initial MRI scans; red: group x time interaction from longitudinal LME models, indicating where patients have more rapid cortical thinning than controls; green: overlap between group and group x time effects. All results were calculated from linear mixed effects (LME) models and thresholded at voxelwise p<0.001 with a minimum cluster volume of \SI{560}{\micro\litre}, corresponding to a corrected cluster-wise threshold of p<0.05. \label{gm_me_time}}

\end{figure}

Exploratory analysis highlighted regions where each phenotype had lower
cortical thickness than controls at the time of their initial MRI scan;
regions that displayed both initial atrophy as well as atrophy
progression over time; and regions without initial atrophy where
patients had a higher longitudinal rate of atrophy than controls.
Supplementary Table 6 reports the peak t-statistic (local minimum) and
volume occupied in each of the Mindboggle cortical regions of interest
for contrasts of initial cortical thickness vs.~controls, and
Supplementary Table 7 similarly reports regional peaks for contrasts of
longitudinal rate of change vs.~controls.

At the time of initial MRI, the lvPPA group exhibited lower cortical
thickness vs.~controls in left middle and superior temporal gyri, our
hypothesized disease focus for lvPPA, corroborating ROI volume analysis.
Additionally, we observed additional cortical thinning in the left
central and parietal opercula; planum temporale; planum polare; and left
inferior temporal, fusiform, supramarginal, angular, inferior occipital,
and middle occipital gyri. In prefrontal cortex, lvPPA patients had
cortical thinning in left anterior insula and frontal operculum as well
as left anterior orbital, left medial orbital, and bilateral middle and
superior frontal gyri. Focal thinning of left parahippocampal gyrus,
which was detected in ROI-based analysis, was observed only at an
uncorrected threshold of p\textless{}0.01 (Supplementary Figure 3) and
did not survive multiple comparisons correction.

Several voxel clusters in which lvPPA patients displayed initial
thinning failed to exhibit further progression in longitudinal analysis,
including portions of left perisylvian cortex, posterior insula, and
inferior occipital and occipital fusiform gyri as well as bilateral
middle and superior frontal gyri (Figure 3A, blue regions). The absence
of longitudinal progression in these areas suggests that
neurodegeneration had reached a plateau during the longitudinal course
of these observations. In contrast, a number of left temporal and
inferior parietal voxel clusters displayed both initial and longitudinal
atrophy, including inferior, middle, and superior temporal gyri;
perisylvian cortex; and supramarginal, angular, and anterior fusiform
gyri (Figure 3A, green regions). A similar pattern of effects was
observed throughout left insular and opercular cortex and in bilateral
middle and superior frontal gyri. Finally, longitudinal progression in
the absence of initial atrophy occurred throughout bilateral prefrontal,
medial parietal, and anterior temporal cortex as well as right inferior
parietal/posterior temporal cortex (Figure 3A, red regions), suggesting
spread of disease to these areas following patients' initial scans.

PCA patients' initial neocortical atrophy corroborated ROI-based results
(Table 2) and additionally indicated homologous atrophy in left
precuneus, superior parietal lobule and angular gyrus (Figure 3B).
Initial cortical thinning was also observed in bilateral inferior
occipital, middle occipital, and posterior cingulate gyri; right cuneus
as well as precentral, middle frontal, and superior frontal gyri, and
inferior occipitotemporal cortex; and left superior occipital and
superior temporal gyri. Areas of initial atrophy that continued to
progress longitudinally included bilateral posterior cingulate gyri as
well as right precentral gyrus, left precuneus, and left supramarginal
and superior temporal gyri. Progressive atrophy was observed in several
areas that were not atrophied at the time of initial MRI, including
bilateral postcentral gyri, transverse temporal gyri, plana temporalia,
central opercula, and temporal poles; left frontal pole, middle/superior
frontal gyri, inferior frontal gyrus (pars opercularis and pars
triangularis), and middle cingulate gyrus; and right parietal operculum
and planum polare.

As in ROI-based analysis, fvAD patients exhibited initial atrophy
relative to controls in left anterior insula and middle frontal gyrus,
right angular gyrus, and bilateral middle temporal gyri. Additionally,
the fvAD group had cortical thinning in left supplementary motor area
and supramarginal gyrus as well as right anterior insula, middle frontal
gyrus, medial prefrontal cortex, posterior cingulate, and posterior
insula. Cortical thinning was also observed bilaterally in the inferior
and superior frontal gyri; anterior, middle, and posterior cingulate
gyri; inferior and superior temporal gyri; frontal opercula; and
temporal poles (Figure 3C). The fvAD group also had initial atrophy
relative to controls in the anterior and dorsal portion of right
entorhinal cortex, a finding that was not captured by ROI-based
analysis. Differences in the rate of atrophy between fvAD patients and
controls were limited to the junction of right anterior insula and
frontal/central operculum, which also exhibited atrophy relative to
controls at initial MRI.

At initial MRI, the aAD group exhibited lower cortical thickness than
controls in bilateral middle and posterior cingulate gyri; left angular
and supramarginal gyri as well as inferior, middle, and superior
temporal gyri; and right frontal lobe areas including supplementary
motor cortex and anterior orbital, middle frontal, superior frontal, and
medial precentral gyri (Figure 3D). Additionally, we observed cortical
thinning in the left parahippocampal gyrus and bilateral entorhinal
cortex at p\textless{}0.01, uncorrected (Supplementary Figure 3);
however, these results were not significant at cluster-wise
p\textless{}0.05. Regions where aAD patients displayed a faster rate of
atrophy over time in the absence of initial cortical thinning included
the bilateral insulae, angular gyri, precunei, supramarginal gyri, and
superior parietal lobules as well as portions of right prefrontal,
lateral temporal, inferior occipito-temporal, and occipital cortex. In
the aAD group, regions that differed from controls in rates of
longitudinal atrophy did not overlap with regions displaying atrophy at
first scan.

\begin{figure}[]
% Voxelwise group differences in cortical thickness.

{\centering \includegraphics[width=7in]{./fig/vox_base_diff.pdf} 

}

\caption[GM mean differences]{Voxelwise differences between patient groups in cortical thickness at time of initial MRI scan. Results are thresholded at voxelwise p<0.001 with a minimum cluster volume of \SI{560}{\micro\litre}, corresponding to a corrected cluster-wise threshold of p<0.05. Warm colors indicate thinner cortical grey matter in the second group than the first; cool colors indicate thinner cortical grey matter in the first group than the second. \label{gm_mean_diff}}

\end{figure}

Figure 4 presents contrasts between patient groups of initial cortical
thickness. Relative to aAD, lvPPA patients exhibited more severe atrophy
than aAD patients in left fusiform and inferior temporal gyri,
consistent with the expected left-lateralization of disease in these
patients (Figure 4A). Conversely, lvPPA patients had less atrophy in
right entorhinal cortex. Lateralization was also evident in PCA
patients, who had lower cortical thickness vs.~the lvPPA group
throughout right but not left occipitotemporal and parietal cortex
(Figure 4D), including the superior parietal lobule as well as the
posterior cingulate, fusiform, occipital fusiform, and inferior and
superior occipital gyri. PCA patients had greater atrophy than aAD
patients in bilateral superior parietal lobules and precunei as well as
right fusiform and inferior temporal gyri (Figure 4B); conversely, PCA
patients had less severe atrophy than aAD patients in ventromedial
prefrontal cortex. In contrast to all other groups, fvAD patients
exhibited more severe atrophy in frontal and anterior temporal areas.
Relative to aAD patients, the fvAD group had more severe atrophy in
right medial superior frontal gyrus, supplementary motor area, and
medial orbital gyrus (Figure 4C). Relative to lvPPA patients, the fvAD
group had more severe atrophy in right entorhinal cortex, posterior
insula, and middle temporal and middle cingulate gyri (Figure 4D).
Finally, fvAD patients had more severe atrophy than PCA patients in
ventromedial prefrontal regions including anterior cingulate and
subcallosal cortex as well as the gyri recti; conversely, fvAD patients
had less severe atrophy than PCA patients in left middle and superior
occipital gyri. Collectively, these results replicate initial volume
differences from ROI-based analysis and highlight additional
phenotype-specific areas of atrophy reported in prior studies of lvPPA
(Rogalski \emph{et al.}, 2016), PCA (Lehmann \emph{et al.}, 2012), and
fvAD (Whitwell \emph{et al.}, 2011).

\begin{figure}[]
% Voxelwise group differences in rates of longitudinal GM change.

{\centering \includegraphics[width=7in]{./fig/vox_group_time.pdf} 

}

\caption[GM group x time differences]{Voxelwise differences between patient groups in rates of cortical thinning over time. Image overlays are t-statistic maps for the interaction of each group with time, calculated from linear mixed effects (LME) models and thresholded at voxelwise p<0.001 with a minimum cluster volume of \SI{560}{\micro\litre}, corresponding to a corrected cluster-wise threshold of p<0.05. Warm colors indicate that cortical thinning over time is more rapid in the second group than the first; cool colors indicate that cortical thinning is more rapid in the first group than the second. No differences in rates of cortical thinning were observed between aAD and PCA patients or between lvPPA and PCA patients. \label{gm_group_time}}

\end{figure}

Finally, we assessed group differences in the rate of cortical thinning
over time. In general, we found more rapid rates of progressive atrophy
in core neocortical regions and also at the margins of core regions
associated with each naAD subgroup. Left posterior superior temporal
gyrus, which exhibited faster atrophy for lvPPA than aAD patients in
ROI-based analysis, demonstrated the same pattern in voxelwise analysis
(Figure 5A). FvAD patients exhibited focal differences relative to the
other patient groups in prefrontal, parietal, and temporal cortex.
Atrophy was faster in fvAD than aAD in right anterior cingulate cortex
but more gradual than the aAD group in right supramarginal and middle
occipital gyri (Figure 5B). Relative to fvAD patients, lvPPA patients
had more rapid atrophy in bilateral inferior and middle occipital gyri
as well as left precuneus and right supramarginal gyri (Figure 5C). PCA
patients also exhibited more rapid atrophy than fvAD patients
posteriorly in right middle occipital gyrus (Figure 5D). In entorhinal
cortex and parahippocampal gyri, rates of cortical thinning did not
differ in aAD compared to naAD.

Independent of group differences in longitudinal atrophy, age at initial
MRI was inversely associated with cortical thickness, including large
portions of the bilateral medial and anterior temporal lobes, including
bilateral temporal poles, entorhinal cortex, parahippocampal gyri, and
anterior fusiform gyri; bilateral precentral, postcentral, and medial
orbital gyri; and right lingual gyrus, calcarine cortex, cuneus, and
middle temporal gyrus (Supplementary Figure 2, top). Additionally, age
was positively associated with cortical thickness in the precuneus,
which exhibits greater atrophy in earlier-onset than later-onset
Alzheimer's disease (Möller \emph{et al.}, 2013). Associations with
baseline MMSE score were distributed throughout all lobes of the brain
(Supplementary Figure 2, bottom).

\section*{Discussion}\label{discussion}
\addcontentsline{toc}{section}{Discussion}

Few studies to date have addressed the longitudinal spread of disease in
naAD. Previous longitudinal imaging studies of non-amnestic Alzheimer's
disease have focused on single variants such as lvPPA (Rogalski \emph{et
al.}, 2011; Rohrer \emph{et al.}, 2013) and PCA (Lehmann \emph{et al.},
2012). The current study compares longitudinal disease progression in
multiple clinically-defined naAD phenotypes with autopsy or CSF evidence
of Alzheimer's disease pathology. The longitudinal design allowed us to
investigate not only areas of initial atrophy in each phenotype, but how
atrophy progressed as a result of spreading pathology. Furthermore, we
asked whether longitudinal rates of neurodegeneration differed across
patient groups in phenotype-specific disease foci, a finding which could
at least partially account for each phenotype's characteristic disease
distribution. Below, we review areas of initial and progressive atrophy
in each phenotype as well as differences between groups in the
longitudinal progression of atrophy in the neocortex and MTL.
Collectively, the current results provide valuable support for the
hypothesis that pathology originates in and emanates from
phenotype-specific neocortical areas rather than the MTL in naAD and
corroborate previous autopsy studies of hippocampal-sparing Alzheimer's
disease that have been limited by sparse clinical characterization of
patients and the cross-sectional nature of postmortem studies
(Giannakopoulos \emph{et al.}, 1994; Murray \emph{et al.}, 2011;
Ferreira \emph{et al.}, 2017).

Contrasts relative to controls illustrate the anatomical progression of
disease in each patient group. In all phenotypes, some brain areas
exhibited significant atrophy at the initial scan without a temporal
progression. Following the logic of pathology studies, we reasoned that
such regions are likely involved early in the disease process; because
they exhibited minimal progression over the study period, we infer that
these regions have already undergone advanced neurodegeneration. We also
observed other areas that exhibited both atrophy at initial MRI and
significant change over time. These areas represent an important window
on progressive neurodegeneration, where atrophy is ongoing and has not
yet reached a plateau. Finally, areas that demonstrated change over time
but whose initial values did not differ from controls may have begun to
degenerate only later in the disease process.

ROI-based and voxelwise analyses provided converging and complementary
evidence regarding atrophy patterns at initial MRI. Overall, lvPPA
patients exhibited strong left lateralization of atrophy, consistent
with prior studies (Rogalski \emph{et al.}, 2016; Phillips \emph{et
al.}, 2018). In ROI-based analysis, lvPPA patients had neocortical
volume loss relative to controls in left middle and superior temporal
gyri, anterior insula, and middle frontal gyrus. Voxelwise analysis
corroborated these findings and indicated additional atrophy in left
occipitotemporal and posterior temporal/inferior parietal cortex as well
as left orbitofrontal cortex and bilateral superior frontal gyri.
Additionally, ROI-based analysis indicated that lvPPA patients had
volume loss in left hippocampus and parahippocampal gyrus. In the PCA
group, ROI-based analysis indicated initial atrophy in right
hippocampus, precuneus, superior parietal lobule, and angular gyrus as
well as in bilateral middle temporal gyri. Voxelwise analysis indicated
that this atrophy extended to homologous left-hemisphere areas as well
as bilateral occipital cortex; right cuneus, prefrontal cortex, and
inferior occipitotemporal cortex; and left superior temporal gyrus. In
the fvAD group, ROI-based analysis indicated initial volume loss in left
anterior insula and middle frontal gyrus, right angular gyrus, and
bilateral middle temporal gyri. Voxelwise analysis corroborated these
findings and indicated additional atrophy throughout bilateral medial
and lateral prefrontal cortex. Additionally, fvAD patients exhibited
initial cortical thinning in right posterior cingulate gyrus, entorhinal
cortex, and anterior and posterior insula; left supramarginal gyrus; and
bilateral temporal lobes, including the temporal poles, perisylvian
cortex, and inferior and superior temporal gyri. Finally, ROI-based
analysis indicated that the aAD group had initial volume loss relative
to controls in all six MTL regions but in none of the other regions
tested. Voxelwise cortical thickness was lower in the aAD group than in
controls in bilateral middle and posterior cingulate gyri; left inferior
parietal and lateral temporal cortex; and right medial and lateral
prefrontal areas. Cortical thinning in the left parahippocampal gyrus
and bilateral entorhinal cortex was observed at a lenient statistical
threshold but did not survive multiple comparisons correction. As
discussed previously, this discrepancy between ROI-based and voxelwise
results may indicate that the precise focus of atrophy varied across
patients in these structures, or it may result from the difficulty of
segmenting MTL structures. The presence of some MTL atrophy at baseline
imaging in lvPPA and PCA differs from our previous cross-sectional
analysis of atrophy in naAD (Phillips \emph{et al.}, 2018), which
concluded that MTL areas were not significantly atrophied relative to
controls in early disease. This discrepancy may be due to differences in
analysis approaches: while our previous study used a non-parametric,
frequency-based algorithm inspired by pathological staging studies, the
current study uses a more traditional parametric statistical approach.
Additional work is needed to resolve these differences. Nevertheless,
both ROI-based and voxelwise analyses showed that lvPPA patients had
significantly less atrophy than aAD patients in MTL regions. Overall,
cross-sectional comparisons of atrophy at initial MRI largely
corroborated prior conclusions regarding the anatomical origin of
disease in each phenotype (Ossenkoppele \emph{et al.}, 2015; Xia
\emph{et al.}, 2017; Phillips \emph{et al.}, 2018).

A subset of regions that exhibited initial atrophy failed to exhibit
further longitudinal change during the study period. These regions may
represent areas where atrophy has slowed or ceased as the volume of
remaining tissue decreases (Sabuncu \emph{et al.}, 2011; Schuff \emph{et
al.}, 2012), even as continued progression is observed in other regions.
This pattern of effects is likely to indicate foci of earliest disease
in each phenotype. In ROI-based analysis of the PCA group, all regions
that exhibited initial atrophy also displayed significant longitudinal
change; however, voxelwise analysis revealed clusters of early,
non-progressing atrophy in bilateral calcarine cortex, cunei, occipital
poles, and inferior, middle, and superior occipital gyri; right superior
parietal lobule, supplementary motor cortex, and right fusiform,
lingual, and occipital fusiform gyri; and left inferior temporal gyrus.
These areas are largely consistent with the suspected anatomical origin
of disease in PCA based on prior studies (Tang-Wai \emph{et al.}, 2004;
Lehmann \emph{et al.}, 2012). In fvAD, nearly all of the areas that
exibited initial atrophy failed to display longitudinal progression,
including large portions of bilateral ventral and dorsal medial and
lateral prefrontal cortex as well as more anterior portions of the
temporal lobes, as described above; these findings support the
hypothesis that patients with behavioral and executive function deficits
due to underlying Alzheimer's disease have disease originating in the
frontal lobe. In aAD, initial atrophy without subsequent progression was
observed in left entorhinal cortex (Table 2), corresponding to early
Braak stages (Braak and Braak, 1991). Initial, non-progressing atrophy
was also observed in bilateral middle and posterior cingulate gyri; left
inferior, middle, and superior temporal gyri; right anterior orbital and
medial precentral gyri; and right supplementary motor cortex (Figure 3,
Supplementary Table 6). In the context of aAD, many of these neocortical
areas are unlikely to reflect early involvement in the disease process,
since they correspond to Braak stages of V--VI in aAD; we may have
failed to detect longitudinal change in these areas due to statistical
power issues or because the areas in question are on the lower end of
normal variation in cortical thickness (Von Economo, 1929; Fischl and
Dale, 2000; Sowell \emph{et al.}, 2004) and exhibit smaller changes over
time. In contrast to other groups, lvPPA patients exhibited longitudinal
progression in some form in all areas that displayed initial atrophy.

Many of the areas that exhibited initial atrophy relative to controls
also exhibited longitudinal change, suggesting that neurodegeneration
had begun prior to initial MRI and did not reach a plateau during the
study period. In lvPPA, these areas included frontal, central, and
parietal opercula; inferior, middle, and superior temporal gyri;
angular, supramarginal, fusiform, and middle occipital gyri; anterior
insula; planum temporale (Figure 3, Supplementary Tables 6--7); and left
hippocampus and parahippocampal gyrus (Table 2). In several larger ROIs,
voxel clusters that exhibited cortical thinning at initial MRI did not
exhibit subsequent progression, but adjacent clusters within the same
ROI did. Thus, the lvPPA group showed early atrophy within the left
middle frontal gyrus that remained stable over the follow-up period
(Figure 3, top row, blue), but atrophy appeared to spread to neighboring
voxels within the middle frontal gyrus (Figure 3, red); ROI analysis
corroborated longitudinal change in left middle frontal gyrus among
lvPPA patients. Similar within-region spread of atrophy was observed in
bilateral prefrontal cortex as well as left posterior insula,
perisylvian cortex, occipital fusiform gyrus, and orbitofrontal cortex.
In the PCA group, ROI-based analysis indicated overlapping effects of
initial and progressive atrophy in bilateral middle temporal gyri as
well as right hippocampus, precuneus, superior parietal lobule, and
angular gyrus. Voxelwise analysis corroborated each of these findings
and additionally indicated overlapping initial and longitudinal atrophy
in bilateral supramarginal and posterior cingulate gyri; left parietal
operculum as well as angular and superior temporal gyri; and right
middle cingulate, middle frontal, and precentral gyri. Within-region
spread of atrophy was additionally observed in voxelwise analysis
throughout the right temporal lobe and left superior parietal cortex. In
the fvAD group, ROI-based analysis indicated both initial and
longitudinal atrophy in bilateral middle temporal gyri and left anterior
insula. In voxelwise analysis, the overlap between initial and
longitudinal atrophy was restricted to right anterior insula, frontal
operculum, and central operculum; the left anterior insula was also
detected at a more lenient statistical threshold (Supplementary Figure
3). ROI-based analysis in the aAD group indicated that five of six MTL
regions (excluding left entorhinal cortex) exhibited both initial and
longitudinal atrophy relative to controls. Additionally, voxelwise
analysis suggested within-region spread of atrophy in left angular and
supramarginal gyri as well as right middle and superior frontal gyri, as
evidenced clusters of initial atrophy (but no longitudinal change)
located proximally to clusters that exhibited longitudinal change (but
no initial atrophy).

Finally, both ROI-based and voxelwise analysis detected several areas in
each phenotype that displayed longitudinal progression relative to
controls in the absence of initial atrophy. This profile of effects
reflects brain areas that developed significant atrophy after the start
of the study period and are likely to represent areas of later
involvement in each phenotype. In lvPPA, these areas of latest
involvement were widespread and included bilateral entorhinal cortex as
well as right hippocampus and parahippocampal gyrus (Table 2); and
bilateral atrophy in the temporal poles, inferior frontal gyri,
occipitotemporal cortex, precunei, posterior cingulate gyri, superior
parietal lobules, precentral gyri, and medial aspects of the superior
frontal gyri (Figure 3, Supplementary Table 7). Additional areas of
later atrophy included the left frontal pole and postcentral gyrus.
Areas of later right hemisphere atrophy in lvPPA included perisylvian
and lateral temporal cortex, middle and angular and supramarginal gyri,
ventral occipitotemporal areas, and inferior frontal gyrus; these
findings appear to represent spread of disease from homologous left
hemisphere areas affected early in the disease process, as reported in
previous longitudinal imaging of PPA (Rogalski \emph{et al.}, 2011;
Rohrer \emph{et al.}, 2013). In the PCA group, ROI-based analysis
indicated later longitudinal atrophy in right entorhinal cortex and
parahipppocampal gyrus as well as left hippocampus; left superior
temporal and middle frontal gyri; and right supramarginal gyrus (Table
2). Voxelwise analysis additionally indicated later atrophy in bilateral
postcentral gyri, perisylvian cortex, central opercula, and temporal
poles; left middle cingulate, inferior frontal (pars opercularis),
precentral, and superior frontal gyri; and right parietal operculum
(Figure 3, Supplementary Table 7). These results are consistent with the
spread of PCA into more anterior portions of the temporal lobes and into
prefrontal cortex (Lehmann \emph{et al.}, 2012). In the fvAD group,
ROI-based analysis indicated later atrophy in bilateral entorhinal
cortex as well as left superior temporal gyrus, while voxelwise analysis
indicated later cortical thinning in the right anterior insula and
frontal operculum; discrepancies between ROI-based and voxelwise results
in this case may result from differences in the sensitivity of the two
methods for atrophy detection or to the difficulty of segmenting MTL
regions, as previously discussed. In the aAD group, ROI-based analysis
indicated later atrophy in bilateral middle temporal gyri, left middle
frontal and superior temporal gyri, and right angular gyrus,
supramarginal gyrus, precuneus, and superior parietal lobule (Table 2).
Voxelwise results similarly indicated clusters of longitudinal cortical
thinning in each of these areas (Figure 3, Supplementary Figure 3) and
also indicated clusters of later atrophy in bilateral anterior and
posterior insulae, and posterior orbital gyri; left frontal operculum,
inferior frontal gyrus (pars opercularis), precuneus, and superior
parietal lobule; and right central operculum, parietal operculum,
cuneus, temporal pole, lateral temporal cortex (inferior and superior
temporal gyri), and occipitotemporal areas (fusiform, middle occipital,
and superior occipital gyri). These findings in aAD are consistent with
inferences about disease staging in pathology studies of aAD (Braak and
Braak, 1991), and autopsy studies of regional pathologic burden in naAD
(Josephs \emph{et al.}, 2008) would help verify our findings in these
less common variants of Alzheimer's disease. Collectively, areas
exhibiting longitudinal atrophy relative to controls in the absence of
initial cross-sectional differences provide a window onto how disease
spreads within each phenotype in more advanced disease.

Direct comparisons between patient groups further corroborated
established phenotype-specific patterns of atrophy (Ossenkoppele
\emph{et al.}, 2015; Phillips \emph{et al.}, 2018), including sparing of
the MTL relative to aAD (Murray \emph{et al.}, 2011). At initial MRI,
ROI analysis showed that lvPPA patients exhibited significantly greater
GM volume than aAD patients in bilateral entorhinal cortex, right
hippocampus, and right parahippocampal gyrus (Figure 1). FvAD patients
similarly exhibited greater GM volume than the aAD group in left
entorhinal cortex, and PCA patients exhibited marginally greater volumes
than aAD patients in left hippocampus and entorhinal cortex (both
p\textless{}0.1, corrected). These results indicate relative sparing of
MTL structures in naAD and demonstrate that aAD patients had undergone
more extensive atrophy in MTL areas than naAD patients prior to their
initial MRI scans. In the neocortex, ROI analysis showed that lvPPA
patients exhibited reduced GM volumes relative to other patient groups
in left lateral temporal cortex and relative to PCA patients in left
anterior insula. Voxelwise results additionally indicated that lvPPA
patients had lower initial cortical thickness than aAD patients in left
inferior temporal and fusiform gyri. ROI analysis in PCA patients, in
turn, indicated lower GM volumes than lvPPA and aAD patients in the
right precuneus and superior parietal lobule and lower volumes than the
lvPPA group in right angular and middle temporal gyri. Voxelwise
analysis provided converging evidence for a right parietal disease focus
in the current PCA sample: initial cortical thickness was lower in PCA
relative to both aAD and lvPPA patients in both right occipitotemporal
(fusiform, occipital fusiform, and middle and superior occipital gyri)
and parietal (precuneus and superior parietal lobule) areas. ROI
analysis in fvAD patients indicated more severe initial atrophy than PCA
patients in left middle frontal gyrus; greater atrophy relative to lvPPA
patients in right angular and middle temporal gyri; and atrophy relative
to aAD patients in the right superior parietal lobule. Voxelwise
analysis provided converging evidence that fvAD patients had more severe
atrophy in medial prefrontal and right anterior temporal areas than
other patient groups; the finding of right temporal degeneration, in
particular, is consistent with patients' behavioral symptoms (Seeley
\emph{et al.}, 2005).

The current study also reports novel comparisons of longitudinal atrophy
rates in naAD and aAD. LvPPA patients had more rapid atrophy than aAD
patients in left superior temporal (Figure 2) and supramarginal gyri
(Figure 5); these differential rates of change are independent of
baseline atrophy. Rates of atrophy also differed between fvAD and other
phenotypes along an anterior-posterior axis. ROI-based analysis
indicated more rapid atrophy for fvAD than lvPPA and PCA patients in
left anterior insula; and in voxelwise analysis, fvAD patients had more
rapid atrophy than aAD patients in right anterior cingulate cortex.
Conversely, fvAD patients exhibited more gradual volume loss than PCA
and aAD patients in right precuneus and relative to the aAD group in
right supramarginal gyrus (Figure 2); additionally, fvAD patients had
more gradual cortical thinning than all other groups in right angular
and middle occipital gyri and relative to aAD and lvPPA in right
supramarginal gyrus. These neocortical differences may contribute to the
rapid, domain-specific clinical progression observed in naAD (Lam
\emph{et al.}, 2013). Contrary to hypotheses, we observed more rapid
neocortical atrophy for aAD than naAD patients in multiple areas,
including right supramarginal gyrus (relative to lvPPA and fvAD
patients) and right middle temporal gyrus (relative to lvPPA and PCA
patients). While we had not hypothesized that these two regions would
have specific involvement in aAD, functional connectivity between the
hippocampus and each is disrupted in aAD (Park \emph{et al.}, 2017).
Collectively, longitudinal analysis demonstrates that rates of
neurodegeneration vary between clinical presentations of Alzheimer's
disease as well as between brain areas. These results underscore the
importance of examining anatomical disease markers that are specifically
associated with patients' primary cognitive deficits rather than
simplified measures such as mean cortical thickness or volume.

The importance of phenotype-specific ROIs in our hypothesis-driven
analysis is validated by associations between GM volume change in these
regions and longitudinal neuropsychological performance. In the memory
domain, bilateral hippocampi were the most sensitive anatomical markers
of memory change; parahippocampal and entorhinal areas showed lesser but
still significant associations. In the language domain, patients'
declines in letter fluency, a measure that requires guided retrieval of
lexical representations, were associated with reductions in left middle
temporal gyrus volume. Moreover, forward digit span had robust
associations with left temporal volume change; we have previously
reported this measure as a sensitive behavioral marker of lvPPA
(Giannini \emph{et al.}, 2017). The 4-point speech scale administered as
part of the PBAC exhibited no associations with longitudinal GM atrophy,
likely due to the limited range of this scale; furthermore, this speech
scale was developed to be sensitive to the entire range of deficits
found in PPA and is not specifically tuned to deficits found in lvPPA
(Libon \emph{et al.}, 2011b). Decline in visuospatial function was
associated with volume loss in several right-hemisphere parietal and
temporal areas. Finally, decline in reverse digit span, a measure of
working memory and executive function, was associated with ROIs specific
to the fvAD group (left middle frontal and right middle temporal gyri).

In the MTL, we had expected naAD patients to exhibit slower rates of
atrophy than aAD patients, providing a potential explanation for their
relative hippocampal sparing and preserved memory. However, although
naAD patients---particularly lvPPA cases---had less MTL atrophy at
initial MRI than aAD patients, rates of change largely did not differ by
group. In fact, in the left entorhinal cortex, fvAD patients displayed
more rapid atrophy than aAD patients, a finding which we attribute to
slowing of left entorhinal atrophy in aAD due to its already-severe
atrophy at initial MRI. The absence of hypothesized differences in rates
of MTL atrophy suggests that MTL sparing in naAD takes the form of a
delayed onset of MTL atrophy rather than a similar onset with a
different rate of change. Thus, MTL sparing in autopsy studies of
Alzheimer's disease patients with non-amnestic presentations (Murray
\emph{et al.}, 2011) could be due to later involvement in naAD than in
aAD rather than to a slower rate of atrophy. Moreover, secondary
analysis found that age effects did not differ between aAD and naAD,
suggesting that naAD patients are not protected from normal
age-associated degeneration of the MTL (Apostolova \emph{et al.}, 2012).
In all six MTL ROIs tested---but no other ROIs---higher age at the
initial scan was associated with more severe atrophy. Voxelwise analysis
provided converging evidence for robust age effects throughout the
medial and anterior temporal lobes. These findings are consistent with
previous studies showing that hippocampal-sparing Alzheimer's disease
cases have a younger age at onset (Murray \emph{et al.}, 2011). These
results are also in line with a recent study of Eckerström and
colleagues (Eckerström \emph{et al.}, 2018), where early-
(age\textless{}65) and late-onset Alzheimer's disease
(age\textgreater{}65) showed no differences in hippocampal and cortical
atrophy after correcting for age. Finally, such age effects may help
explain the occurrence of cases with atypical clinical phenotypes but
typical distributions of Alzheimer's disease neuropathology at autopsy
(Murray \emph{et al.}, 2011; Boon \emph{et al.}, 2018). Collectively,
these findings suggest that the focal MTL atrophy in aAD may be at least
partially due to age effects, independently of clinical phenotype (Pol
\emph{et al.}, 2006\}). More longitudinal studies comparing aAD and naAD
are necessary to address this theory, which raises interesting questions
about the association between aging and regional susceptibility to
disease. Although poor tolerance of MRI makes it difficult to image
patients with advanced disease, such late-stage imaging would be useful
in determining whether patients with an initial non-amnestic syndrome
eventually exhibit spread of atrophy to MTL.

Strengths of the current study include a novel comparison of
longitudinal anatomical changes in multiple clinically-defined naAD
phenotypes using both a priori ROI-based and whole-brain voxelwise
analyses. We sought to ensure the comparability of the heterogeneous
patient groups included here by controlling for demographic and clinical
characteristics both during sample selection and in statistical
analysis. However, one major limitation was the inability to evaluate
non-linear atrophy progression in Alzheimer's disease: prior evidence
suggests that an initial acceleration due to spreading cumulative damage
is followed by a deceleration due to the reduction of intact tissue
(Sabuncu \emph{et al.}, 2011; Schuff \emph{et al.}, 2012). Such
non-linearities complicate study design and interpretation in ways that
may not be fully addressed by equating patient groups for chronological
age and estimated disease duration: for example, in the current study,
it is possible that areas of early atrophy in each phenotype (i.e.,
those exhibiting atrophy at initial MRI) have entered the deceleration
phase, while for other phenotypes the same regions may have been imaged
during the acceleration phase. Investigating longitudinal change in
earlier-stage patients may allow us to observe a more complete
trajectory of neurodegeneration, and including a minimum of 3--4 imaging
timepoints may allow us to discriminate between linear, quadratic, and
sigmoid models of neurodegeneration. Additionally, the current study was
limited in its ability to investigate associations with the APOE
genotype or other genetic risk modifiers for Alzheimer's disease. We
found that APOE \(\epsilon4\) allele counts added little predictive
power to our imaging models after accounting for group effects; however,
continued study of the APOE genotype and other genetic risk modifiers in
naAD remains an important research aim. Finally, future studies should
include patients with corticobasal syndrome due to underlying
Alzheimer's disease pathology; insufficient longitudinal data prevented
us from including this uncommon naAD phenotype in the current study.

Understanding the neuropathological and clinical heterogeneity of
Alzheimer's disease is crucial to understanding the mechanisms of its
progression. The current results indicate that the onset and rate of
neocortical atrophy varies by region and phenotype in naAD, reflecting
longitudinal clinical progression. Moreover, the rate of MTL atrophy in
naAD appears to be similar to that found in aAD, although onset of MTL
atrophy in naAD is considerably delayed.

\section*{Acknowledegments}\label{acknowledegments}
\addcontentsline{toc}{section}{Acknowledegments}

The authors would like to thank Dr.~Valeria Isella and Dr.~Carlo
Ferrarese for their valuable feedback on this project.

\section*{Funding}\label{funding}
\addcontentsline{toc}{section}{Funding}

This work was supported by grants from the Alzheimer's Association
(AARF-16-443681), National Institutes of Health (AG017586, AG010124,
AG043503, and NS088341), BrightFocus Foundation (A2016244S), Dana
Foundation, Newhouse Foundation, Wyncote Foundation, Arking Family
Foundation, and the Italian Ministry of Education, University, and
Research.

\section*{Competing interests}\label{competing-interests}
\addcontentsline{toc}{section}{Competing interests}

All authors report that they have no competing interests to disclose.

\section*{References}\label{references}
\addcontentsline{toc}{section}{References}

\hypertarget{refs}{}
\hypertarget{ref-apostolova_hippocampal_2012}{}
Apostolova LG, Green AE, Babakchanian S, Hwang KS, Chou Y-Y, Toga AW, et
al. Hippocampal atrophy and ventricular enlargement in normal aging,
mild cognitive impairment (MCI), and Alzheimer Disease. Alzheimer
Disease and Associated Disorders 2012; 26: 17--27.

\hypertarget{ref-avants_reproducible_2011}{}
Avants BB, Tustison NJ, Song G, Cook PA, Klein A, Gee JC. A reproducible
evaluation of ANTs similarity metric performance in brain image
registration. NeuroImage 2011; 54: 2033--2044.

\hypertarget{ref-avants_insight_2014}{}
Avants BB, Tustison NJ, Stauffer M, Song G, Wu B, Gee JC. The Insight
ToolKit image registration framework {[}Internet{]}. Frontiers in
Neuroinformatics 2014; 8{[}cited 2015 Jun 26{]} Available from:
\url{http://www.ncbi.nlm.nih.gov/pmc/articles/PMC4009425/}

\hypertarget{ref-boon_neuroinflammation_2018}{}
Boon BDC, Hoozemans JJM, Lopuhaä B, Eigenhuis KN, Scheltens P, Kamphorst
W, et al. Neuroinflammation is increased in the parietal cortex of
atypical Alzheimer's disease {[}Internet{]}. Journal of
Neuroinflammation 2018; 15{[}cited 2018 Jun 7{]} Available from:
\url{https://www.ncbi.nlm.nih.gov/pmc/articles/PMC5975447/}

\hypertarget{ref-braak_neuropathological_1991}{}
Braak H, Braak E. Neuropathological stageing of Alzheimer-related
changes. Acta neuropathologica 1991; 82: 239--259.

\hypertarget{ref-byun_heterogeneity_2015}{}
Byun MS, Kim SE, Park J, Yi D, Choe YM, Sohn BK, et al. Heterogeneity of
Regional Brain Atrophy Patterns Associated with Distinct Progression
Rates in Alzheimer's Disease. PloS One 2015; 10: e0142756.

\hypertarget{ref-chen_linear_2013}{}
Chen G, Saad ZS, Britton JC, Pine DS, Cox RW. Linear mixed-effects
modeling approach to FMRI group analysis. NeuroImage 2013; 73: 176--190.

\hypertarget{ref-corder_protective_1994}{}
Corder EH, Saunders AM, Risch NJ, Strittmatter WJ, Schmechel DE, Gaskell
PC, et al. Protective effect of apolipoprotein E type 2 allele for late
onset Alzheimer disease. Nature Genetics 1994; 7: 180--184.

\hypertarget{ref-cox_fmri_2017}{}
Cox RW, Chen G, Glen DR, Reynolds RC, Taylor PA. FMRI Clustering in
AFNI: False-Positive Rates Redux. Brain Connectivity 2017; 7: 152--171.

\hypertarget{ref-crutch_consensus_2017}{}
Crutch SJ, Schott JM, Rabinovici GD, Murray M, Snowden JS, Flier WM van
der, et al. Consensus classification of posterior cortical atrophy
{[}Internet{]}. Alzheimer's \& Dementia 2017{[}cited 2017 Mar 7{]}
Available from:
\url{http://www.sciencedirect.com/science/article/pii/S1552526017300407}

\hypertarget{ref-dickerson_approach_2017}{}
Dickerson BC, McGinnis SM, Xia C, Price BH, Atri A, Murray ME, et al.
Approach to atypical Alzheimer's disease and case studies of the major
subtypes. CNS spectrums 2017; 22: 439--449.

\hypertarget{ref-duara_regional_2013}{}
Duara R, Loewenstein DA, Shen Q, Barker W, Greig MT, Varon D, et al.
Regional patterns of atrophy on MRI in Alzheimer's disease:
Neuropsychological features and progression rates in the ADNI cohort.
Advances in Alzheimer's Disease 2013; 02: 135--147.

\hypertarget{ref-dubois_advancing_2014}{}
Dubois B, Feldman HH, Jacova C, Hampel H, Molinuevo JL, Blennow K, et
al. Advancing research diagnostic criteria for Alzheimer's disease: The
IWG-2 criteria. The Lancet Neurology 2014; 13: 614--629.

\hypertarget{ref-eckerstrom_similar_2018}{}
Eckerström C, Klasson N, Olsson E, Selnes P, Rolstad S, Wallin A.
Similar pattern of atrophy in early- and late-onset Alzheimer's disease.
Alzheimer's \& Dementia: Diagnosis, Assessment \& Disease Monitoring
2018; 10: 253--259.

\hypertarget{ref-ferreira_distinct_2017}{}
Ferreira D, Verhagen C, Hernández-Cabrera JA, Cavallin L, Guo C-J, Ekman
U, et al. Distinct subtypes of Alzheimer's disease based on patterns of
brain atrophy: Longitudinal trajectories and clinical applications.
Scientific Reports 2017; 7: 46263.

\hypertarget{ref-fischl_measuring_2000}{}
Fischl B, Dale AM. Measuring the thickness of the human cerebral cortex
from magnetic resonance images. Proceedings of the National Academy of
Sciences 2000; 97: 11050--11055.

\hypertarget{ref-forman_improved_1995}{}
Forman SD, Cohen JD, Fitzgerald M, Eddy WF, Mintun MA, Noll DC. Improved
assessment of significant activation in functional magnetic resonance
imaging (fMRI): Use of a cluster-size threshold. Magnetic Resonance in
Medicine 1995; 33: 636--647.

\hypertarget{ref-galton_atypical_2000}{}
Galton CJ, Patterson K, Xuereb JH, Hodges JR. Atypical and typical
presentations of Alzheimer's disease: A clinical neuropsychological,
neuroimaging and pathological study of 13 cases. Brain: A Journal of
Neurology 2000; 123: 484--498.

\hypertarget{ref-giannakopoulos_alzheimers_1994}{}
Giannakopoulos P, Hof PR, Bouras C. Alzheimer's disease with asymmetric
atrophy of the cerebral hemispheres: Morphometric analysis of four
cases. Acta Neuropathologica 1994; 88: 440--447.

\hypertarget{ref-giannini_clinical_2017}{}
Giannini LAA, Irwin DJ, McMillan CT, Ash S, Rascovsky K, Wolk DA, et al.
Clinical marker for Alzheimer disease pathology in logopenic primary
progressive aphasia. Neurology 2017; 88: 2276--2284.

\hypertarget{ref-gorno-tempini_classification_2011}{}
Gorno-Tempini ML, Hillis AE, Weintraub S, Kertesz A, Mendez M, Cappa SF,
et al. Classification of primary progressive aphasia and its variants.
Neurology 2011; 76: 1006--1014.

\hypertarget{ref-irwin_comparison_2012}{}
Irwin DJ, McMillan CT, Toledo JB, Arnold SE, Shaw LM, Wang L-S, et al.
Comparison of cerebrospinal fluid levels of tau and A\(\beta\) 1-42 in
Alzheimer disease and frontotemporal degeneration using 2 analytical
platforms. Archives of Neurology 2012; 69: 1018--1025.

\hypertarget{ref-josephs_progressive_2008}{}
Josephs KA, Whitwell JL, Duffy JR, Vanvoorst WA, Strand EA, Hu WT, et
al. Progressive aphasia secondary to Alzheimer disease pathology: A
clinicopathologic and MRI study. Neurology 2008; 70: 25--34.

\hypertarget{ref-klein_evaluation_2009}{}
Klein A, Andersson J, Ardekani BA, Ashburner J, Avants B, Chiang M-C, et
al. Evaluation of 14 nonlinear deformation algorithms applied to human
brain MRI registration. Neuroimage 2009; 46: 786--802.

\hypertarget{ref-klein_101_2012}{}
Klein A, Tourville J. 101 labeled brain images and a consistent human
cortical labeling protocol. Frontiers in Neuroscience 2012; 6: 171.

\hypertarget{ref-kramer_distinctive_2003}{}
Kramer JH, Jurik J, Sha SJ, Rankin KP, Rosen HJ, Johnson JK, et al.
Distinctive neuropsychological patterns in frontotemporal dementia,
semantic dementia, and Alzheimer disease. Cognitive and Behavioral
Neurology: Official Journal of the Society for Behavioral and Cognitive
Neurology 2003; 16: 211--218.

\hypertarget{ref-lam_clinical_2013}{}
Lam B, Masellis M, Freedman M, Stuss DT, Black SE. Clinical, imaging,
and pathological heterogeneity of the Alzheimer's disease syndrome.
Alzheimer's Research \& Therapy 2013; 5: 1.

\hypertarget{ref-lehmann_global_2012}{}
Lehmann M, Barnes J, Ridgway GR, Ryan NS, Warrington EK, Crutch SJ, et
al. Global gray matter changes in posterior cortical atrophy: A serial
imaging study. Alzheimer's \& Dementia 2012; 8: 502--512.

\hypertarget{ref-libon_verbal_2011}{}
Libon DJ, Bondi MW, Price CC, Lamar M, Eppig J, Wambach DM, et al.
Verbal Serial List Learning in Mild Cognitive Impairment: A Profile
Analysis of Interference, Forgetting, and Errors. Journal of the
International Neuropsychological Society 2011a; 17: 905--914.

\hypertarget{ref-libon_philadelphia_2011}{}
Libon DJ, Rascovsky K, Gross RG, White MT, Xie SX, Dreyfuss M, et al.
The Philadelphia Brief Assessment of Cognition (PBAC): A Validated
Screening Measure for Dementia. The Clinical Neuropsychologist 2011b;
25: 1314--1330.

\hypertarget{ref-marcus_open_2007}{}
Marcus DS, Wang TH, Parker J, Csernansky JG, Morris JC, Buckner RL. Open
Access Series of Imaging Studies (OASIS): Cross-sectional MRI data in
young, middle aged, nondemented, and demented older adults. Journal of
Cognitive Neuroscience 2007; 19: 1498--1507.

\hypertarget{ref-mckhann_diagnosis_2011}{}
McKhann GM, Knopman DS, Chertkow H, Hyman BT, Jack Jr. CR, Kawas CH, et
al. The diagnosis of dementia due to Alzheimer's disease:
Recommendations from the National Institute on Aging-Alzheimer's
Association workgroups on diagnostic guidelines for Alzheimer's disease.
Alzheimer's \& Dementia 2011; 7: 263--269.

\hypertarget{ref-medaglia_brain_2017}{}
Medaglia JD, Huang W, Segarra S, Olm C, Gee J, Grossman M, et al. Brain
network efficiency is influenced by the pathologic source of
corticobasal syndrome. Neurology 2017; 89: 1373--1381.

\hypertarget{ref-mesulam_primary_2014}{}
Mesulam M-M, Rogalski EJ, Wieneke C, Hurley RS, Geula C, Bigio EH, et
al. Primary progressive aphasia and the evolving neurology of the
language network. Nature Reviews. Neurology 2014; 10: 554--569.

\hypertarget{ref-moller_different_2013}{}
Möller C, Vrenken H, Jiskoot L, Versteeg A, Barkhof F, Scheltens P, et
al. Different patterns of gray matter atrophy in early- and late-onset
Alzheimer's disease. Neurobiology of Aging 2013; 34: 2014--2022.

\hypertarget{ref-murray_neuropathologically_2011}{}
Murray ME, Graff-Radford NR, Ross OA, Petersen RC, Duara R, Dickson DW.
Neuropathologically defined subtypes of Alzheimer's disease with
distinct clinical characteristics: A retrospective study. The Lancet.
Neurology 2011; 10: 785--796.

\hypertarget{ref-ossenkoppele_atrophy_2015}{}
Ossenkoppele R, Cohn-Sheehy BI, La Joie R, Vogel JW, Möller C, Lehmann
M, et al. Atrophy Patterns in Early Clinical Stages Across Distinct
Phenotypes of Alzheimer's Disease. Human brain mapping 2015; 36:
4421--4437.

\hypertarget{ref-park_functional_2017}{}
Park KH, Noh Y, Choi EJ, Kim H, Chun S, Son YD. Functional Connectivity
of the Hippocampus in Early- and vs. Late-Onset Alzheimer's Disease.
Journal of Clinical Neurology (Seoul, Korea) 2017; 13: 387--393.

\hypertarget{ref-peter_subgroups_2014}{}
Peter J, Abdulkadir A, Kaller C, Kümmerer D, Hüll M, Vach W, et al.
Subgroups of Alzheimer's disease: Stability of empirical clusters over
time. Journal of Alzheimer's disease: JAD 2014; 42: 651--661.

\hypertarget{ref-phillips_neocortical_2018}{}
Phillips JS, Da Re F, Dratch L, Xie SX, Irwin DJ, McMillan CT, et al.
Neocortical origin and progression of gray matter atrophy in nonamnestic
Alzheimer's disease. Neurobiology of Aging 2018; 63: 75--87.

\hypertarget{ref-van_de_pol_hippocampal_2006}{}
Pol LA van de, Hensel A, Barkhof F, Gertz HJ, Scheltens P, Flier WM van
der. Hippocampal atrophy in Alzheimer disease: Age matters. Neurology
2006; 66: 236--238.

\hypertarget{ref-poulakis_heterogeneous_2018}{}
Poulakis K, Pereira JB, Mecocci P, Vellas B, Tsolaki M, Link to external
site this link will open in a new window, et al. Heterogeneous patterns
of brain atrophy in Alzheimer's disease. Neurobiology of Aging 2018; 65:
98--108.

\hypertarget{ref-ramanan_longitudinal_2017}{}
Ramanan S, Bertoux M, Flanagan E, Irish M, Piguet O, Hodges JR, et al.
Longitudinal Executive Function and Episodic Memory Profiles in
Behavioral-Variant Frontotemporal Dementia and Alzheimer's Disease.
Journal of the International Neuropsychological Society: JINS 2017; 23:
34--43.

\hypertarget{ref-rascovsky_disparate_2007}{}
Rascovsky K, Salmon DP, Hansen LA, Thal LJ, Galasko D. Disparate letter
and semantic category fluency deficits in autopsy-confirmed
frontotemporal dementia and Alzheimer's disease. Neuropsychology 2007;
21: 20--30.

\hypertarget{ref-rogalski_progression_2011}{}
Rogalski E, Cobia D, Harrison TM, Wieneke C, Weintraub S, Mesulam M-M.
Progression of language decline and cortical atrophy in subtypes of
primary progressive aphasia. Neurology 2011; 76: 1804--1810.

\hypertarget{ref-rogalski_aphasic_2016}{}
Rogalski E, Sridhar J, Rader B, Martersteck A, Chen K, Cobia D, et al.
Aphasic variant of Alzheimer disease: Clinical, anatomic, and genetic
features. Neurology 2016; 87: 1337--1343.

\hypertarget{ref-rohrer_patterns_2013}{}
Rohrer JD, Caso F, Mahoney C, Henry M, Rosen HJ, Rabinovici G, et al.
Patterns of longitudinal brain atrophy in the logopenic variant of
primary progressive aphasia. Brain and Language 2013; 127: 121--126.

\hypertarget{ref-sabuncu_dynamics_2011}{}
Sabuncu MR, Desikan RS, Sepulcre J, Yeo BTT, Liu H, Schmansky NJ, et al.
The Dynamics of Cortical and Hippocampal Atrophy in Alzheimer Disease.
Archives of Neurology 2011; 68: 1040--1048.

\hypertarget{ref-schuff_nonlinear_2012}{}
Schuff N, Tosun D, Insel PS, Chiang GC, Truran D, Aisen PS, et al.
Nonlinear time course of brain volume loss in cognitively normal and
impaired elders. Neurobiology of Aging 2012; 33: 845--855.

\hypertarget{ref-seeley_natural_2005}{}
Seeley W, Bauer A, Miller B, Gorno-Tempini M, Kramer J, Weiner M, et al.
The natural history of temporal variant frontotemporal dementia.
Neurology 2005; 64: 1384--1390.

\hypertarget{ref-shaw_cerebrospinal_2009}{}
Shaw LM, Vanderstichele H, Knapik-Czajka M, Clark CM, Aisen PS, Petersen
RC, et al. Cerebrospinal fluid biomarker signature in Alzheimer's
disease neuroimaging initiative subjects. Annals of Neurology 2009; 65:
403--413.

\hypertarget{ref-sowell_longitudinal_2004}{}
Sowell ER, Thompson PM, Leonard CM, Welcome SE, Kan E, Toga AW.
Longitudinal mapping of cortical thickness and brain growth in normal
children. The Journal of Neuroscience: The Official Journal of the
Society for Neuroscience 2004; 24: 8223--8231.

\hypertarget{ref-tang-wai_clinical_2004}{}
Tang-Wai DF, Graff-Radford NR, Boeve BF, Dickson DW, Parisi JE, Crook R,
et al. Clinical, genetic, and neuropathologic characteristics of
posterior cortical atrophy. Neurology 2004; 63: 1168--1174.

\hypertarget{ref-tustison_n4itk:_2010}{}
Tustison NJ, Avants BB, Cook PA, Zheng Y, Egan A, Yushkevich PA, et al.
N4ITK: Improved N3 bias correction. IEEE transactions on medical imaging
2010; 29: 1310--1320.

\hypertarget{ref-tustison_explicit_2013}{}
Tustison NJ, Avants BB. Explicit B-spline regularization in
diffeomorphic image registration {[}Internet{]}. Frontiers in
Neuroinformatics 2013; 7{[}cited 2018 Oct 5{]} Available from:
\url{https://www.ncbi.nlm.nih.gov/pmc/articles/PMC3870320/}

\hypertarget{ref-tustison_large-scale_2014}{}
Tustison NJ, Cook PA, Klein A, Song G, Das SR, Duda JT, et al.
Large-scale evaluation of ANTs and FreeSurfer cortical thickness
measurements. NeuroImage 2014; 99: 166--179.

\hypertarget{ref-von_economo_cytoarchitectonics_1929}{}
Von Economo C. The cytoarchitectonics of the human cerebral cortex. H.
Milford Oxford University Press; 1929.

\hypertarget{ref-wang_multi-atlas_2013}{}
Wang H, Suh JW, Das SR, Pluta J, Craige C, Yushkevich PA. Multi-Atlas
Segmentation with Joint Label Fusion. IEEE transactions on pattern
analysis and machine intelligence 2013; 35: 611--623.

\hypertarget{ref-whitwell_neuroimaging_2012}{}
Whitwell JL, Dickson DW, Murray ME, Weigand SD, Tosakulwong N, Senjem
ML, et al. Neuroimaging correlates of pathologically defined subtypes of
Alzheimer's disease: A case-control study. The Lancet Neurology 2012;
11: 868--877.

\hypertarget{ref-whitwell_temporoparietal_2011}{}
Whitwell JL, Jack CR, Przybelski SA, Parisi JE, Senjem ML, Link to
external site this link will open in a new window, et al.
Temporoparietal atrophy: A marker of AD pathology independent of
clinical diagnosis. Neurobiology of Aging 2011; 32: 1531--1541.

\hypertarget{ref-xia_association_2017}{}
Xia C, Makaretz SJ, Caso C, McGinnis S, Gomperts SN, Sepulcre J, et al.
Association of In Vivo {[}18F{]}AV-1451 Tau PET Imaging Results With
Cortical Atrophy and Symptoms in Typical and Atypical Alzheimer Disease
{[}Internet{]}. JAMA Neurology 2017{[}cited 2017 Mar 15{]} Available
from:
\url{http://jamanetwork.com/journals/jamaneurology/fullarticle/2604134}


\end{document}
